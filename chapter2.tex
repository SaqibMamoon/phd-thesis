\documentclass[thesis]{subfiles}

\begin{document}
%*******************************************************************************
%****************************** Second Chapter *********************************
%*******************************************************************************

\chapter{Background}
\label{background}
\ifpdf
    \graphicspath{{Figs/Raster/}{Figs/PDF/}{Figs/}}
\else
    \graphicspath{{Figs/Vector/}{Figs/}}
\fi

\section{Classifiers}
\subsection{Neural Networks}
Artificial Neural Networks (or simply Neural Networks) are a broad range of statistical models characterized by consisting of a set of inter-connected nodes with non-linear \emph{activation functions} with learnable parameters, or \emph{weights}). Although initially biologically inspired, Neural Networks (NN) within the field of machine learning have moved away from biologically-plausible models and towards practical models for applications as diverse as computer vision, speech recognition, and general classifiers.

Neural networks are a universal approximator - a network with one \emph{hidden} layer can theoretically represent any function. However, in practice less nodes are required when more hidden layers are used, and thus in practice networks may have many hidden layers.
\subsection{Decision Forests}

\section{The relationship between Decision Forests and Neural Networks}
There has been work in the past exploring the relationship between decision forests and neural networks. This work has usually focused on exploiting the equivalence of decision trees and a subset of neural networks.

\subsection{Entropy Nets}
This equivalence was introduced by Sethi~\cite{Sethi1990}, describing how the decision boundaries which are explicitly expressed in a decision tree can be represented by a three-layer neural network, where the decision nodes of the tree are on the first layer, the leaf nodes on the second layer)
\subsection{Random Forests as Artificial Neural Networks}
~\cite{Welbl2014casting}

\section{Object Recognition Datasets}
\subsection{CIFAR 10/100}
\subsection{Imagenet}

\section{Object Category Recognition}
\section[Bag of Words]{Bag of Words Approaches to Object Class Recognition}

\section{Convolutional Neural Networks}
\subsection{Imagenet}
\end{document}