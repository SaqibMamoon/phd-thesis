\documentclass[thesis]{subfiles}

\begin{document}
	\chapter{The Unreasonable Affect of Structure on Learning in Neural Networks}
	\label{motivation}
	
	It is well known that the design of a neural network architecture can have a large effect on the generalization of a learned model. And yet, despite this, network design itself remains poorly understood, and is considered somewhat of a black magic, with intuition and experience being the cited motivation behind most common architectures, rather than any theory. This, more than perhaps any other factor, has been a barrier to access for the field.
	
	Beyond the simpler hyper-parameters used for tuning the optimization method, such as learning rate, momentum and weight decay, the architecture of a network has a profound effect on the learning. Nowhere is this affect more pronounced than in the case of using neural networks with image inputs, which pose unique challenges in learning. 
	
	
	\section{Network Architecture and Generalization}
    A persistent question in training artificial neural networks has been in the design of the networks. Specifically the question of how many parameters should be learned, and in what way they should be connected,  so as to be suitable for good generalization from a given size dataset. Notable steps in the theoretical answers to this question include findings showing the limitations of single layer networks~\citep{minsky1988perceptrons}, information-theoretic measures of the representational capacity of a network~\citep{vapnik2015uniform}, the proof that single hidden-layer networks are universal approximators~\citep{hornik89a}, and the theoretical number of nodes required for generalization from a dataset of given size~\citep{baum1989size}. 
    
    Empirical results have, however, shown that the realities of training neural networks do not match what theory predicts. Deep networks of many hidden layers have been shown time and again to out-perform shallow networks~\citep{Krizhevsky2012,Simonyan2014verydeep,He2015,He2016}, perhaps due to our limited method of optimization~\citep{NIPS2014_5484}. Networks with many more parameters than training samples, that use early-stopping or are regularized strongly, generalize better in practice than networks with the theoretically sufficient capacity~\citep{caruana2001overfitting, Krizhevsky2012, HintonTalk2015}. Networks designed with a specialized connectivity structure closer reflecting the underlying solution have consistently generalized better than fully-connected networks with higher learning capacity~\citep{lecun1989backpropagation,He2016}. In fact these seemingly theoretically defying design strategies can claim to have been responsible for recent breakthroughs in generalization on previously difficult tasks such as image class recognition~\citep{Krizhevsky2012, HintonTalk2015}.
    
    \todo{Include references to \citet{caruana2001overfitting, baum1989size, rethinking2016}}
    
    \section{Model Capacity and Representational Power}
    \citep{vapnik2015uniform}
    \citep{hornik89a}
    \citep{baum1989size}
    
	\section{Structural Priors}

	\begin{figure}[tb]
		\centering
		\begin{subfigure}[t]{0.49\textwidth}
			\resizebox{\linewidth}{!}{%% Creator: Matplotlib, PGF backend
%%
%% To include the figure in your LaTeX document, write
%%   \input{<filename>.pgf}
%%
%% Make sure the required packages are loaded in your preamble
%%   \usepackage{pgf}
%%
%% Figures using additional raster images can only be included by \input if
%% they are in the same directory as the main LaTeX file. For loading figures
%% from other directories you can use the `import` package
%%   \usepackage{import}
%% and then include the figures with
%%   \import{<path to file>}{<filename>.pgf}
%%
%% Matplotlib used the following preamble
%%   \usepackage[utf8x]{inputenc}
%%   \usepackage[T1]{fontenc}
%%
\begingroup%
\makeatletter%
\begin{pgfpicture}%
\pgfpathrectangle{\pgfpointorigin}{\pgfqpoint{4.296389in}{2.655314in}}%
\pgfusepath{use as bounding box, clip}%
\begin{pgfscope}%
\pgfsetbuttcap%
\pgfsetmiterjoin%
\definecolor{currentfill}{rgb}{1.000000,1.000000,1.000000}%
\pgfsetfillcolor{currentfill}%
\pgfsetlinewidth{0.000000pt}%
\definecolor{currentstroke}{rgb}{1.000000,1.000000,1.000000}%
\pgfsetstrokecolor{currentstroke}%
\pgfsetdash{}{0pt}%
\pgfpathmoveto{\pgfqpoint{0.000000in}{0.000000in}}%
\pgfpathlineto{\pgfqpoint{4.296389in}{0.000000in}}%
\pgfpathlineto{\pgfqpoint{4.296389in}{2.655314in}}%
\pgfpathlineto{\pgfqpoint{0.000000in}{2.655314in}}%
\pgfpathclose%
\pgfusepath{fill}%
\end{pgfscope}%
\begin{pgfscope}%
\pgfsetbuttcap%
\pgfsetmiterjoin%
\definecolor{currentfill}{rgb}{1.000000,1.000000,1.000000}%
\pgfsetfillcolor{currentfill}%
\pgfsetlinewidth{0.000000pt}%
\definecolor{currentstroke}{rgb}{0.000000,0.000000,0.000000}%
\pgfsetstrokecolor{currentstroke}%
\pgfsetstrokeopacity{0.000000}%
\pgfsetdash{}{0pt}%
\pgfpathmoveto{\pgfqpoint{0.548317in}{0.386884in}}%
\pgfpathlineto{\pgfqpoint{4.124652in}{0.386884in}}%
\pgfpathlineto{\pgfqpoint{4.124652in}{2.488647in}}%
\pgfpathlineto{\pgfqpoint{0.548317in}{2.488647in}}%
\pgfpathclose%
\pgfusepath{fill}%
\end{pgfscope}%
\begin{pgfscope}%
\pgfsetbuttcap%
\pgfsetroundjoin%
\definecolor{currentfill}{rgb}{0.150000,0.150000,0.150000}%
\pgfsetfillcolor{currentfill}%
\pgfsetlinewidth{1.003750pt}%
\definecolor{currentstroke}{rgb}{0.150000,0.150000,0.150000}%
\pgfsetstrokecolor{currentstroke}%
\pgfsetdash{}{0pt}%
\pgfsys@defobject{currentmarker}{\pgfqpoint{0.000000in}{-0.083333in}}{\pgfqpoint{0.000000in}{0.000000in}}{%
\pgfpathmoveto{\pgfqpoint{0.000000in}{0.000000in}}%
\pgfpathlineto{\pgfqpoint{0.000000in}{-0.083333in}}%
\pgfusepath{stroke,fill}%
}%
\begin{pgfscope}%
\pgfsys@transformshift{0.548317in}{0.386884in}%
\pgfsys@useobject{currentmarker}{}%
\end{pgfscope}%
\end{pgfscope}%
\begin{pgfscope}%
\definecolor{textcolor}{rgb}{0.150000,0.150000,0.150000}%
\pgfsetstrokecolor{textcolor}%
\pgfsetfillcolor{textcolor}%
\pgftext[x=0.548317in,y=0.206329in,,top]{\color{textcolor}\sffamily\fontsize{10.000000}{12.000000}\selectfont \(\displaystyle 0.0\)}%
\end{pgfscope}%
\begin{pgfscope}%
\pgfsetbuttcap%
\pgfsetroundjoin%
\definecolor{currentfill}{rgb}{0.150000,0.150000,0.150000}%
\pgfsetfillcolor{currentfill}%
\pgfsetlinewidth{1.003750pt}%
\definecolor{currentstroke}{rgb}{0.150000,0.150000,0.150000}%
\pgfsetstrokecolor{currentstroke}%
\pgfsetdash{}{0pt}%
\pgfsys@defobject{currentmarker}{\pgfqpoint{0.000000in}{-0.083333in}}{\pgfqpoint{0.000000in}{0.000000in}}{%
\pgfpathmoveto{\pgfqpoint{0.000000in}{0.000000in}}%
\pgfpathlineto{\pgfqpoint{0.000000in}{-0.083333in}}%
\pgfusepath{stroke,fill}%
}%
\begin{pgfscope}%
\pgfsys@transformshift{1.263584in}{0.386884in}%
\pgfsys@useobject{currentmarker}{}%
\end{pgfscope}%
\end{pgfscope}%
\begin{pgfscope}%
\definecolor{textcolor}{rgb}{0.150000,0.150000,0.150000}%
\pgfsetstrokecolor{textcolor}%
\pgfsetfillcolor{textcolor}%
\pgftext[x=1.263584in,y=0.206329in,,top]{\color{textcolor}\sffamily\fontsize{10.000000}{12.000000}\selectfont \(\displaystyle 0.2\)}%
\end{pgfscope}%
\begin{pgfscope}%
\pgfsetbuttcap%
\pgfsetroundjoin%
\definecolor{currentfill}{rgb}{0.150000,0.150000,0.150000}%
\pgfsetfillcolor{currentfill}%
\pgfsetlinewidth{1.003750pt}%
\definecolor{currentstroke}{rgb}{0.150000,0.150000,0.150000}%
\pgfsetstrokecolor{currentstroke}%
\pgfsetdash{}{0pt}%
\pgfsys@defobject{currentmarker}{\pgfqpoint{0.000000in}{-0.083333in}}{\pgfqpoint{0.000000in}{0.000000in}}{%
\pgfpathmoveto{\pgfqpoint{0.000000in}{0.000000in}}%
\pgfpathlineto{\pgfqpoint{0.000000in}{-0.083333in}}%
\pgfusepath{stroke,fill}%
}%
\begin{pgfscope}%
\pgfsys@transformshift{1.978851in}{0.386884in}%
\pgfsys@useobject{currentmarker}{}%
\end{pgfscope}%
\end{pgfscope}%
\begin{pgfscope}%
\definecolor{textcolor}{rgb}{0.150000,0.150000,0.150000}%
\pgfsetstrokecolor{textcolor}%
\pgfsetfillcolor{textcolor}%
\pgftext[x=1.978851in,y=0.206329in,,top]{\color{textcolor}\sffamily\fontsize{10.000000}{12.000000}\selectfont \(\displaystyle 0.4\)}%
\end{pgfscope}%
\begin{pgfscope}%
\pgfsetbuttcap%
\pgfsetroundjoin%
\definecolor{currentfill}{rgb}{0.150000,0.150000,0.150000}%
\pgfsetfillcolor{currentfill}%
\pgfsetlinewidth{1.003750pt}%
\definecolor{currentstroke}{rgb}{0.150000,0.150000,0.150000}%
\pgfsetstrokecolor{currentstroke}%
\pgfsetdash{}{0pt}%
\pgfsys@defobject{currentmarker}{\pgfqpoint{0.000000in}{-0.083333in}}{\pgfqpoint{0.000000in}{0.000000in}}{%
\pgfpathmoveto{\pgfqpoint{0.000000in}{0.000000in}}%
\pgfpathlineto{\pgfqpoint{0.000000in}{-0.083333in}}%
\pgfusepath{stroke,fill}%
}%
\begin{pgfscope}%
\pgfsys@transformshift{2.694118in}{0.386884in}%
\pgfsys@useobject{currentmarker}{}%
\end{pgfscope}%
\end{pgfscope}%
\begin{pgfscope}%
\definecolor{textcolor}{rgb}{0.150000,0.150000,0.150000}%
\pgfsetstrokecolor{textcolor}%
\pgfsetfillcolor{textcolor}%
\pgftext[x=2.694118in,y=0.206329in,,top]{\color{textcolor}\sffamily\fontsize{10.000000}{12.000000}\selectfont \(\displaystyle 0.6\)}%
\end{pgfscope}%
\begin{pgfscope}%
\pgfsetbuttcap%
\pgfsetroundjoin%
\definecolor{currentfill}{rgb}{0.150000,0.150000,0.150000}%
\pgfsetfillcolor{currentfill}%
\pgfsetlinewidth{1.003750pt}%
\definecolor{currentstroke}{rgb}{0.150000,0.150000,0.150000}%
\pgfsetstrokecolor{currentstroke}%
\pgfsetdash{}{0pt}%
\pgfsys@defobject{currentmarker}{\pgfqpoint{0.000000in}{-0.083333in}}{\pgfqpoint{0.000000in}{0.000000in}}{%
\pgfpathmoveto{\pgfqpoint{0.000000in}{0.000000in}}%
\pgfpathlineto{\pgfqpoint{0.000000in}{-0.083333in}}%
\pgfusepath{stroke,fill}%
}%
\begin{pgfscope}%
\pgfsys@transformshift{3.409385in}{0.386884in}%
\pgfsys@useobject{currentmarker}{}%
\end{pgfscope}%
\end{pgfscope}%
\begin{pgfscope}%
\definecolor{textcolor}{rgb}{0.150000,0.150000,0.150000}%
\pgfsetstrokecolor{textcolor}%
\pgfsetfillcolor{textcolor}%
\pgftext[x=3.409385in,y=0.206329in,,top]{\color{textcolor}\sffamily\fontsize{10.000000}{12.000000}\selectfont \(\displaystyle 0.8\)}%
\end{pgfscope}%
\begin{pgfscope}%
\pgfsetbuttcap%
\pgfsetroundjoin%
\definecolor{currentfill}{rgb}{0.150000,0.150000,0.150000}%
\pgfsetfillcolor{currentfill}%
\pgfsetlinewidth{1.003750pt}%
\definecolor{currentstroke}{rgb}{0.150000,0.150000,0.150000}%
\pgfsetstrokecolor{currentstroke}%
\pgfsetdash{}{0pt}%
\pgfsys@defobject{currentmarker}{\pgfqpoint{0.000000in}{-0.083333in}}{\pgfqpoint{0.000000in}{0.000000in}}{%
\pgfpathmoveto{\pgfqpoint{0.000000in}{0.000000in}}%
\pgfpathlineto{\pgfqpoint{0.000000in}{-0.083333in}}%
\pgfusepath{stroke,fill}%
}%
\begin{pgfscope}%
\pgfsys@transformshift{4.124652in}{0.386884in}%
\pgfsys@useobject{currentmarker}{}%
\end{pgfscope}%
\end{pgfscope}%
\begin{pgfscope}%
\definecolor{textcolor}{rgb}{0.150000,0.150000,0.150000}%
\pgfsetstrokecolor{textcolor}%
\pgfsetfillcolor{textcolor}%
\pgftext[x=4.124652in,y=0.206329in,,top]{\color{textcolor}\sffamily\fontsize{10.000000}{12.000000}\selectfont \(\displaystyle 1.0\)}%
\end{pgfscope}%
\begin{pgfscope}%
\pgfsetbuttcap%
\pgfsetroundjoin%
\definecolor{currentfill}{rgb}{0.150000,0.150000,0.150000}%
\pgfsetfillcolor{currentfill}%
\pgfsetlinewidth{1.003750pt}%
\definecolor{currentstroke}{rgb}{0.150000,0.150000,0.150000}%
\pgfsetstrokecolor{currentstroke}%
\pgfsetdash{}{0pt}%
\pgfsys@defobject{currentmarker}{\pgfqpoint{-0.083333in}{0.000000in}}{\pgfqpoint{0.000000in}{0.000000in}}{%
\pgfpathmoveto{\pgfqpoint{0.000000in}{0.000000in}}%
\pgfpathlineto{\pgfqpoint{-0.083333in}{0.000000in}}%
\pgfusepath{stroke,fill}%
}%
\begin{pgfscope}%
\pgfsys@transformshift{0.548317in}{0.386884in}%
\pgfsys@useobject{currentmarker}{}%
\end{pgfscope}%
\end{pgfscope}%
\begin{pgfscope}%
\definecolor{textcolor}{rgb}{0.150000,0.150000,0.150000}%
\pgfsetstrokecolor{textcolor}%
\pgfsetfillcolor{textcolor}%
\pgftext[x=0.082267in,y=0.336742in,left,base]{\color{textcolor}\sffamily\fontsize{10.000000}{12.000000}\selectfont \(\displaystyle -1.0\)}%
\end{pgfscope}%
\begin{pgfscope}%
\pgfsetbuttcap%
\pgfsetroundjoin%
\definecolor{currentfill}{rgb}{0.150000,0.150000,0.150000}%
\pgfsetfillcolor{currentfill}%
\pgfsetlinewidth{1.003750pt}%
\definecolor{currentstroke}{rgb}{0.150000,0.150000,0.150000}%
\pgfsetstrokecolor{currentstroke}%
\pgfsetdash{}{0pt}%
\pgfsys@defobject{currentmarker}{\pgfqpoint{-0.083333in}{0.000000in}}{\pgfqpoint{0.000000in}{0.000000in}}{%
\pgfpathmoveto{\pgfqpoint{0.000000in}{0.000000in}}%
\pgfpathlineto{\pgfqpoint{-0.083333in}{0.000000in}}%
\pgfusepath{stroke,fill}%
}%
\begin{pgfscope}%
\pgfsys@transformshift{0.548317in}{0.912325in}%
\pgfsys@useobject{currentmarker}{}%
\end{pgfscope}%
\end{pgfscope}%
\begin{pgfscope}%
\definecolor{textcolor}{rgb}{0.150000,0.150000,0.150000}%
\pgfsetstrokecolor{textcolor}%
\pgfsetfillcolor{textcolor}%
\pgftext[x=0.082267in,y=0.862183in,left,base]{\color{textcolor}\sffamily\fontsize{10.000000}{12.000000}\selectfont \(\displaystyle -0.5\)}%
\end{pgfscope}%
\begin{pgfscope}%
\pgfsetbuttcap%
\pgfsetroundjoin%
\definecolor{currentfill}{rgb}{0.150000,0.150000,0.150000}%
\pgfsetfillcolor{currentfill}%
\pgfsetlinewidth{1.003750pt}%
\definecolor{currentstroke}{rgb}{0.150000,0.150000,0.150000}%
\pgfsetstrokecolor{currentstroke}%
\pgfsetdash{}{0pt}%
\pgfsys@defobject{currentmarker}{\pgfqpoint{-0.083333in}{0.000000in}}{\pgfqpoint{0.000000in}{0.000000in}}{%
\pgfpathmoveto{\pgfqpoint{0.000000in}{0.000000in}}%
\pgfpathlineto{\pgfqpoint{-0.083333in}{0.000000in}}%
\pgfusepath{stroke,fill}%
}%
\begin{pgfscope}%
\pgfsys@transformshift{0.548317in}{1.437766in}%
\pgfsys@useobject{currentmarker}{}%
\end{pgfscope}%
\end{pgfscope}%
\begin{pgfscope}%
\definecolor{textcolor}{rgb}{0.150000,0.150000,0.150000}%
\pgfsetstrokecolor{textcolor}%
\pgfsetfillcolor{textcolor}%
\pgftext[x=0.190292in,y=1.387624in,left,base]{\color{textcolor}\sffamily\fontsize{10.000000}{12.000000}\selectfont \(\displaystyle 0.0\)}%
\end{pgfscope}%
\begin{pgfscope}%
\pgfsetbuttcap%
\pgfsetroundjoin%
\definecolor{currentfill}{rgb}{0.150000,0.150000,0.150000}%
\pgfsetfillcolor{currentfill}%
\pgfsetlinewidth{1.003750pt}%
\definecolor{currentstroke}{rgb}{0.150000,0.150000,0.150000}%
\pgfsetstrokecolor{currentstroke}%
\pgfsetdash{}{0pt}%
\pgfsys@defobject{currentmarker}{\pgfqpoint{-0.083333in}{0.000000in}}{\pgfqpoint{0.000000in}{0.000000in}}{%
\pgfpathmoveto{\pgfqpoint{0.000000in}{0.000000in}}%
\pgfpathlineto{\pgfqpoint{-0.083333in}{0.000000in}}%
\pgfusepath{stroke,fill}%
}%
\begin{pgfscope}%
\pgfsys@transformshift{0.548317in}{1.963207in}%
\pgfsys@useobject{currentmarker}{}%
\end{pgfscope}%
\end{pgfscope}%
\begin{pgfscope}%
\definecolor{textcolor}{rgb}{0.150000,0.150000,0.150000}%
\pgfsetstrokecolor{textcolor}%
\pgfsetfillcolor{textcolor}%
\pgftext[x=0.190292in,y=1.913065in,left,base]{\color{textcolor}\sffamily\fontsize{10.000000}{12.000000}\selectfont \(\displaystyle 0.5\)}%
\end{pgfscope}%
\begin{pgfscope}%
\pgfsetbuttcap%
\pgfsetroundjoin%
\definecolor{currentfill}{rgb}{0.150000,0.150000,0.150000}%
\pgfsetfillcolor{currentfill}%
\pgfsetlinewidth{1.003750pt}%
\definecolor{currentstroke}{rgb}{0.150000,0.150000,0.150000}%
\pgfsetstrokecolor{currentstroke}%
\pgfsetdash{}{0pt}%
\pgfsys@defobject{currentmarker}{\pgfqpoint{-0.083333in}{0.000000in}}{\pgfqpoint{0.000000in}{0.000000in}}{%
\pgfpathmoveto{\pgfqpoint{0.000000in}{0.000000in}}%
\pgfpathlineto{\pgfqpoint{-0.083333in}{0.000000in}}%
\pgfusepath{stroke,fill}%
}%
\begin{pgfscope}%
\pgfsys@transformshift{0.548317in}{2.488647in}%
\pgfsys@useobject{currentmarker}{}%
\end{pgfscope}%
\end{pgfscope}%
\begin{pgfscope}%
\definecolor{textcolor}{rgb}{0.150000,0.150000,0.150000}%
\pgfsetstrokecolor{textcolor}%
\pgfsetfillcolor{textcolor}%
\pgftext[x=0.190292in,y=2.438505in,left,base]{\color{textcolor}\sffamily\fontsize{10.000000}{12.000000}\selectfont \(\displaystyle 1.0\)}%
\end{pgfscope}%
\begin{pgfscope}%
\pgfpathrectangle{\pgfqpoint{0.548317in}{0.386884in}}{\pgfqpoint{3.576335in}{2.101763in}} %
\pgfusepath{clip}%
\pgfsetbuttcap%
\pgfsetroundjoin%
\definecolor{currentfill}{rgb}{0.400000,0.760784,0.647059}%
\pgfsetfillcolor{currentfill}%
\pgfsetlinewidth{0.000000pt}%
\definecolor{currentstroke}{rgb}{0.400000,0.760784,0.647059}%
\pgfsetstrokecolor{currentstroke}%
\pgfsetdash{}{0pt}%
\pgfsys@defobject{currentmarker}{\pgfqpoint{-0.048611in}{-0.048611in}}{\pgfqpoint{0.048611in}{0.048611in}}{%
\pgfpathmoveto{\pgfqpoint{0.000000in}{-0.048611in}}%
\pgfpathcurveto{\pgfqpoint{0.012892in}{-0.048611in}}{\pgfqpoint{0.025257in}{-0.043489in}}{\pgfqpoint{0.034373in}{-0.034373in}}%
\pgfpathcurveto{\pgfqpoint{0.043489in}{-0.025257in}}{\pgfqpoint{0.048611in}{-0.012892in}}{\pgfqpoint{0.048611in}{0.000000in}}%
\pgfpathcurveto{\pgfqpoint{0.048611in}{0.012892in}}{\pgfqpoint{0.043489in}{0.025257in}}{\pgfqpoint{0.034373in}{0.034373in}}%
\pgfpathcurveto{\pgfqpoint{0.025257in}{0.043489in}}{\pgfqpoint{0.012892in}{0.048611in}}{\pgfqpoint{0.000000in}{0.048611in}}%
\pgfpathcurveto{\pgfqpoint{-0.012892in}{0.048611in}}{\pgfqpoint{-0.025257in}{0.043489in}}{\pgfqpoint{-0.034373in}{0.034373in}}%
\pgfpathcurveto{\pgfqpoint{-0.043489in}{0.025257in}}{\pgfqpoint{-0.048611in}{0.012892in}}{\pgfqpoint{-0.048611in}{0.000000in}}%
\pgfpathcurveto{\pgfqpoint{-0.048611in}{-0.012892in}}{\pgfqpoint{-0.043489in}{-0.025257in}}{\pgfqpoint{-0.034373in}{-0.034373in}}%
\pgfpathcurveto{\pgfqpoint{-0.025257in}{-0.043489in}}{\pgfqpoint{-0.012892in}{-0.048611in}}{\pgfqpoint{0.000000in}{-0.048611in}}%
\pgfpathclose%
\pgfusepath{fill}%
}%
\begin{pgfscope}%
\pgfsys@transformshift{2.260811in}{0.754316in}%
\pgfsys@useobject{currentmarker}{}%
\end{pgfscope}%
\begin{pgfscope}%
\pgfsys@transformshift{2.611975in}{0.899184in}%
\pgfsys@useobject{currentmarker}{}%
\end{pgfscope}%
\begin{pgfscope}%
\pgfsys@transformshift{3.012052in}{1.144101in}%
\pgfsys@useobject{currentmarker}{}%
\end{pgfscope}%
\end{pgfscope}%
\begin{pgfscope}%
\pgfpathrectangle{\pgfqpoint{0.548317in}{0.386884in}}{\pgfqpoint{3.576335in}{2.101763in}} %
\pgfusepath{clip}%
\pgfsetbuttcap%
\pgfsetroundjoin%
\pgfsetlinewidth{1.756562pt}%
\definecolor{currentstroke}{rgb}{0.988235,0.552941,0.384314}%
\pgfsetstrokecolor{currentstroke}%
\pgfsetdash{{5.600000pt}{2.400000pt}}{0.000000pt}%
\pgfpathmoveto{\pgfqpoint{0.534428in}{0.382588in}}%
\pgfpathlineto{\pgfqpoint{0.584442in}{0.397288in}}%
\pgfpathlineto{\pgfqpoint{0.656691in}{0.416887in}}%
\pgfpathlineto{\pgfqpoint{0.728940in}{0.435004in}}%
\pgfpathlineto{\pgfqpoint{0.801189in}{0.451796in}}%
\pgfpathlineto{\pgfqpoint{0.873438in}{0.467418in}}%
\pgfpathlineto{\pgfqpoint{0.945688in}{0.482026in}}%
\pgfpathlineto{\pgfqpoint{1.017937in}{0.495776in}}%
\pgfpathlineto{\pgfqpoint{1.090186in}{0.508825in}}%
\pgfpathlineto{\pgfqpoint{1.162435in}{0.521328in}}%
\pgfpathlineto{\pgfqpoint{1.234684in}{0.533440in}}%
\pgfpathlineto{\pgfqpoint{1.306933in}{0.545319in}}%
\pgfpathlineto{\pgfqpoint{1.379183in}{0.557120in}}%
\pgfpathlineto{\pgfqpoint{1.451432in}{0.568999in}}%
\pgfpathlineto{\pgfqpoint{1.523681in}{0.581112in}}%
\pgfpathlineto{\pgfqpoint{1.595930in}{0.593615in}}%
\pgfpathlineto{\pgfqpoint{1.668179in}{0.606663in}}%
\pgfpathlineto{\pgfqpoint{1.740429in}{0.620414in}}%
\pgfpathlineto{\pgfqpoint{1.812678in}{0.635022in}}%
\pgfpathlineto{\pgfqpoint{1.884927in}{0.650644in}}%
\pgfpathlineto{\pgfqpoint{1.957176in}{0.667435in}}%
\pgfpathlineto{\pgfqpoint{2.029425in}{0.685553in}}%
\pgfpathlineto{\pgfqpoint{2.101675in}{0.705151in}}%
\pgfpathlineto{\pgfqpoint{2.173924in}{0.726388in}}%
\pgfpathlineto{\pgfqpoint{2.246173in}{0.749418in}}%
\pgfpathlineto{\pgfqpoint{2.318422in}{0.774397in}}%
\pgfpathlineto{\pgfqpoint{2.390671in}{0.801482in}}%
\pgfpathlineto{\pgfqpoint{2.462921in}{0.830828in}}%
\pgfpathlineto{\pgfqpoint{2.535170in}{0.862592in}}%
\pgfpathlineto{\pgfqpoint{2.607419in}{0.896929in}}%
\pgfpathlineto{\pgfqpoint{2.679668in}{0.933995in}}%
\pgfpathlineto{\pgfqpoint{2.751917in}{0.973947in}}%
\pgfpathlineto{\pgfqpoint{2.824166in}{1.016940in}}%
\pgfpathlineto{\pgfqpoint{2.896416in}{1.063129in}}%
\pgfpathlineto{\pgfqpoint{2.968665in}{1.112672in}}%
\pgfpathlineto{\pgfqpoint{3.040914in}{1.165725in}}%
\pgfpathlineto{\pgfqpoint{3.113163in}{1.222442in}}%
\pgfpathlineto{\pgfqpoint{3.185412in}{1.282980in}}%
\pgfpathlineto{\pgfqpoint{3.257662in}{1.347495in}}%
\pgfpathlineto{\pgfqpoint{3.329911in}{1.416143in}}%
\pgfpathlineto{\pgfqpoint{3.402160in}{1.489080in}}%
\pgfpathlineto{\pgfqpoint{3.474409in}{1.566461in}}%
\pgfpathlineto{\pgfqpoint{3.546658in}{1.648444in}}%
\pgfpathlineto{\pgfqpoint{3.618908in}{1.735183in}}%
\pgfpathlineto{\pgfqpoint{3.691157in}{1.826835in}}%
\pgfpathlineto{\pgfqpoint{3.763406in}{1.923556in}}%
\pgfpathlineto{\pgfqpoint{3.835655in}{2.025501in}}%
\pgfpathlineto{\pgfqpoint{3.907904in}{2.132827in}}%
\pgfpathlineto{\pgfqpoint{3.980154in}{2.245689in}}%
\pgfpathlineto{\pgfqpoint{4.052403in}{2.364244in}}%
\pgfpathlineto{\pgfqpoint{4.124652in}{2.488647in}}%
\pgfusepath{stroke}%
\end{pgfscope}%
\begin{pgfscope}%
\pgfpathrectangle{\pgfqpoint{0.548317in}{0.386884in}}{\pgfqpoint{3.576335in}{2.101763in}} %
\pgfusepath{clip}%
\pgfsetroundcap%
\pgfsetroundjoin%
\pgfsetlinewidth{1.756562pt}%
\definecolor{currentstroke}{rgb}{0.552941,0.627451,0.796078}%
\pgfsetstrokecolor{currentstroke}%
\pgfsetdash{}{0pt}%
\pgfpathmoveto{\pgfqpoint{0.534428in}{0.664235in}}%
\pgfpathlineto{\pgfqpoint{0.584442in}{0.658600in}}%
\pgfpathlineto{\pgfqpoint{0.656691in}{0.650850in}}%
\pgfpathlineto{\pgfqpoint{0.728940in}{0.643574in}}%
\pgfpathlineto{\pgfqpoint{0.801189in}{0.636856in}}%
\pgfpathlineto{\pgfqpoint{0.873438in}{0.630780in}}%
\pgfpathlineto{\pgfqpoint{0.945688in}{0.625432in}}%
\pgfpathlineto{\pgfqpoint{1.017937in}{0.620894in}}%
\pgfpathlineto{\pgfqpoint{1.090186in}{0.617252in}}%
\pgfpathlineto{\pgfqpoint{1.162435in}{0.614590in}}%
\pgfpathlineto{\pgfqpoint{1.234684in}{0.612992in}}%
\pgfpathlineto{\pgfqpoint{1.306933in}{0.612542in}}%
\pgfpathlineto{\pgfqpoint{1.379183in}{0.613325in}}%
\pgfpathlineto{\pgfqpoint{1.451432in}{0.615424in}}%
\pgfpathlineto{\pgfqpoint{1.523681in}{0.618925in}}%
\pgfpathlineto{\pgfqpoint{1.595930in}{0.623912in}}%
\pgfpathlineto{\pgfqpoint{1.668179in}{0.630468in}}%
\pgfpathlineto{\pgfqpoint{1.740429in}{0.638678in}}%
\pgfpathlineto{\pgfqpoint{1.812678in}{0.648627in}}%
\pgfpathlineto{\pgfqpoint{1.884927in}{0.660399in}}%
\pgfpathlineto{\pgfqpoint{1.957176in}{0.674077in}}%
\pgfpathlineto{\pgfqpoint{2.029425in}{0.689747in}}%
\pgfpathlineto{\pgfqpoint{2.101675in}{0.707493in}}%
\pgfpathlineto{\pgfqpoint{2.173924in}{0.727398in}}%
\pgfpathlineto{\pgfqpoint{2.246173in}{0.749548in}}%
\pgfpathlineto{\pgfqpoint{2.318422in}{0.774026in}}%
\pgfpathlineto{\pgfqpoint{2.390671in}{0.800917in}}%
\pgfpathlineto{\pgfqpoint{2.462921in}{0.830305in}}%
\pgfpathlineto{\pgfqpoint{2.535170in}{0.862274in}}%
\pgfpathlineto{\pgfqpoint{2.607419in}{0.896909in}}%
\pgfpathlineto{\pgfqpoint{2.679668in}{0.934294in}}%
\pgfpathlineto{\pgfqpoint{2.751917in}{0.974513in}}%
\pgfpathlineto{\pgfqpoint{2.824166in}{1.017651in}}%
\pgfpathlineto{\pgfqpoint{2.896416in}{1.063792in}}%
\pgfpathlineto{\pgfqpoint{2.968665in}{1.113019in}}%
\pgfpathlineto{\pgfqpoint{3.040914in}{1.165419in}}%
\pgfpathlineto{\pgfqpoint{3.113163in}{1.221074in}}%
\pgfpathlineto{\pgfqpoint{3.185412in}{1.280069in}}%
\pgfpathlineto{\pgfqpoint{3.257662in}{1.342488in}}%
\pgfpathlineto{\pgfqpoint{3.329911in}{1.408416in}}%
\pgfpathlineto{\pgfqpoint{3.402160in}{1.477937in}}%
\pgfpathlineto{\pgfqpoint{3.474409in}{1.551136in}}%
\pgfpathlineto{\pgfqpoint{3.546658in}{1.628095in}}%
\pgfpathlineto{\pgfqpoint{3.618908in}{1.708901in}}%
\pgfpathlineto{\pgfqpoint{3.691157in}{1.793637in}}%
\pgfpathlineto{\pgfqpoint{3.763406in}{1.882387in}}%
\pgfpathlineto{\pgfqpoint{3.835655in}{1.975236in}}%
\pgfpathlineto{\pgfqpoint{3.907904in}{2.072268in}}%
\pgfpathlineto{\pgfqpoint{3.980154in}{2.173567in}}%
\pgfpathlineto{\pgfqpoint{4.052403in}{2.279218in}}%
\pgfpathlineto{\pgfqpoint{4.124652in}{2.389305in}}%
\pgfusepath{stroke}%
\end{pgfscope}%
\begin{pgfscope}%
\pgfsetrectcap%
\pgfsetmiterjoin%
\pgfsetlinewidth{1.254687pt}%
\definecolor{currentstroke}{rgb}{0.150000,0.150000,0.150000}%
\pgfsetstrokecolor{currentstroke}%
\pgfsetdash{}{0pt}%
\pgfpathmoveto{\pgfqpoint{0.548317in}{0.386884in}}%
\pgfpathlineto{\pgfqpoint{0.548317in}{2.488647in}}%
\pgfusepath{stroke}%
\end{pgfscope}%
\begin{pgfscope}%
\pgfsetrectcap%
\pgfsetmiterjoin%
\pgfsetlinewidth{1.254687pt}%
\definecolor{currentstroke}{rgb}{0.150000,0.150000,0.150000}%
\pgfsetstrokecolor{currentstroke}%
\pgfsetdash{}{0pt}%
\pgfpathmoveto{\pgfqpoint{0.548317in}{0.386884in}}%
\pgfpathlineto{\pgfqpoint{4.124652in}{0.386884in}}%
\pgfusepath{stroke}%
\end{pgfscope}%
\begin{pgfscope}%
\pgfsetbuttcap%
\pgfsetroundjoin%
\definecolor{currentfill}{rgb}{0.400000,0.760784,0.647059}%
\pgfsetfillcolor{currentfill}%
\pgfsetlinewidth{0.000000pt}%
\definecolor{currentstroke}{rgb}{0.400000,0.760784,0.647059}%
\pgfsetstrokecolor{currentstroke}%
\pgfsetdash{}{0pt}%
\pgfsys@defobject{currentmarker}{\pgfqpoint{-0.048611in}{-0.048611in}}{\pgfqpoint{0.048611in}{0.048611in}}{%
\pgfpathmoveto{\pgfqpoint{0.000000in}{-0.048611in}}%
\pgfpathcurveto{\pgfqpoint{0.012892in}{-0.048611in}}{\pgfqpoint{0.025257in}{-0.043489in}}{\pgfqpoint{0.034373in}{-0.034373in}}%
\pgfpathcurveto{\pgfqpoint{0.043489in}{-0.025257in}}{\pgfqpoint{0.048611in}{-0.012892in}}{\pgfqpoint{0.048611in}{0.000000in}}%
\pgfpathcurveto{\pgfqpoint{0.048611in}{0.012892in}}{\pgfqpoint{0.043489in}{0.025257in}}{\pgfqpoint{0.034373in}{0.034373in}}%
\pgfpathcurveto{\pgfqpoint{0.025257in}{0.043489in}}{\pgfqpoint{0.012892in}{0.048611in}}{\pgfqpoint{0.000000in}{0.048611in}}%
\pgfpathcurveto{\pgfqpoint{-0.012892in}{0.048611in}}{\pgfqpoint{-0.025257in}{0.043489in}}{\pgfqpoint{-0.034373in}{0.034373in}}%
\pgfpathcurveto{\pgfqpoint{-0.043489in}{0.025257in}}{\pgfqpoint{-0.048611in}{0.012892in}}{\pgfqpoint{-0.048611in}{0.000000in}}%
\pgfpathcurveto{\pgfqpoint{-0.048611in}{-0.012892in}}{\pgfqpoint{-0.043489in}{-0.025257in}}{\pgfqpoint{-0.034373in}{-0.034373in}}%
\pgfpathcurveto{\pgfqpoint{-0.025257in}{-0.043489in}}{\pgfqpoint{-0.012892in}{-0.048611in}}{\pgfqpoint{0.000000in}{-0.048611in}}%
\pgfpathclose%
\pgfusepath{fill}%
}%
\begin{pgfscope}%
\pgfsys@transformshift{0.812206in}{2.311974in}%
\pgfsys@useobject{currentmarker}{}%
\end{pgfscope}%
\end{pgfscope}%
\begin{pgfscope}%
\definecolor{textcolor}{rgb}{0.150000,0.150000,0.150000}%
\pgfsetstrokecolor{textcolor}%
\pgfsetfillcolor{textcolor}%
\pgftext[x=1.062206in,y=2.263363in,left,base]{\color{textcolor}\sffamily\fontsize{10.000000}{12.000000}\selectfont samples}%
\end{pgfscope}%
\begin{pgfscope}%
\pgfsetbuttcap%
\pgfsetroundjoin%
\pgfsetlinewidth{1.756562pt}%
\definecolor{currentstroke}{rgb}{0.988235,0.552941,0.384314}%
\pgfsetstrokecolor{currentstroke}%
\pgfsetdash{{5.600000pt}{2.400000pt}}{0.000000pt}%
\pgfpathmoveto{\pgfqpoint{0.673317in}{2.115246in}}%
\pgfpathlineto{\pgfqpoint{0.951095in}{2.115246in}}%
\pgfusepath{stroke}%
\end{pgfscope}%
\begin{pgfscope}%
\definecolor{textcolor}{rgb}{0.150000,0.150000,0.150000}%
\pgfsetstrokecolor{textcolor}%
\pgfsetfillcolor{textcolor}%
\pgftext[x=1.062206in,y=2.066635in,left,base]{\color{textcolor}\sffamily\fontsize{10.000000}{12.000000}\selectfont function}%
\end{pgfscope}%
\begin{pgfscope}%
\pgfsetroundcap%
\pgfsetroundjoin%
\pgfsetlinewidth{1.756562pt}%
\definecolor{currentstroke}{rgb}{0.552941,0.627451,0.796078}%
\pgfsetstrokecolor{currentstroke}%
\pgfsetdash{}{0pt}%
\pgfpathmoveto{\pgfqpoint{0.673317in}{1.918518in}}%
\pgfpathlineto{\pgfqpoint{0.951095in}{1.918518in}}%
\pgfusepath{stroke}%
\end{pgfscope}%
\begin{pgfscope}%
\definecolor{textcolor}{rgb}{0.150000,0.150000,0.150000}%
\pgfsetstrokecolor{textcolor}%
\pgfsetfillcolor{textcolor}%
\pgftext[x=1.062206in,y=1.869907in,left,base]{\color{textcolor}\sffamily\fontsize{10.000000}{12.000000}\selectfont 3rd order fit}%
\end{pgfscope}%
\end{pgfpicture}%
\makeatother%
\endgroup%
}
			\caption{\engordnumber{3}-order polynomial fitting 3 points}
			\label{fig:polyfit3rd}
		\end{subfigure}
		~
		\begin{subfigure}[t]{0.49\textwidth}
			\resizebox{\linewidth}{!}{%% Creator: Matplotlib, PGF backend
%%
%% To include the figure in your LaTeX document, write
%%   \input{<filename>.pgf}
%%
%% Make sure the required packages are loaded in your preamble
%%   \usepackage{pgf}
%%
%% Figures using additional raster images can only be included by \input if
%% they are in the same directory as the main LaTeX file. For loading figures
%% from other directories you can use the `import` package
%%   \usepackage{import}
%% and then include the figures with
%%   \import{<path to file>}{<filename>.pgf}
%%
%% Matplotlib used the following preamble
%%   \usepackage[utf8x]{inputenc}
%%   \usepackage[T1]{fontenc}
%%
\begingroup%
\makeatletter%
\begin{pgfpicture}%
\pgfpathrectangle{\pgfpointorigin}{\pgfqpoint{4.296389in}{2.655314in}}%
\pgfusepath{use as bounding box, clip}%
\begin{pgfscope}%
\pgfsetbuttcap%
\pgfsetmiterjoin%
\definecolor{currentfill}{rgb}{1.000000,1.000000,1.000000}%
\pgfsetfillcolor{currentfill}%
\pgfsetlinewidth{0.000000pt}%
\definecolor{currentstroke}{rgb}{1.000000,1.000000,1.000000}%
\pgfsetstrokecolor{currentstroke}%
\pgfsetdash{}{0pt}%
\pgfpathmoveto{\pgfqpoint{0.000000in}{0.000000in}}%
\pgfpathlineto{\pgfqpoint{4.296389in}{0.000000in}}%
\pgfpathlineto{\pgfqpoint{4.296389in}{2.655314in}}%
\pgfpathlineto{\pgfqpoint{0.000000in}{2.655314in}}%
\pgfpathclose%
\pgfusepath{fill}%
\end{pgfscope}%
\begin{pgfscope}%
\pgfsetbuttcap%
\pgfsetmiterjoin%
\definecolor{currentfill}{rgb}{1.000000,1.000000,1.000000}%
\pgfsetfillcolor{currentfill}%
\pgfsetlinewidth{0.000000pt}%
\definecolor{currentstroke}{rgb}{0.000000,0.000000,0.000000}%
\pgfsetstrokecolor{currentstroke}%
\pgfsetstrokeopacity{0.000000}%
\pgfsetdash{}{0pt}%
\pgfpathmoveto{\pgfqpoint{0.548317in}{0.386884in}}%
\pgfpathlineto{\pgfqpoint{4.124652in}{0.386884in}}%
\pgfpathlineto{\pgfqpoint{4.124652in}{2.488647in}}%
\pgfpathlineto{\pgfqpoint{0.548317in}{2.488647in}}%
\pgfpathclose%
\pgfusepath{fill}%
\end{pgfscope}%
\begin{pgfscope}%
\pgfsetbuttcap%
\pgfsetroundjoin%
\definecolor{currentfill}{rgb}{0.150000,0.150000,0.150000}%
\pgfsetfillcolor{currentfill}%
\pgfsetlinewidth{1.003750pt}%
\definecolor{currentstroke}{rgb}{0.150000,0.150000,0.150000}%
\pgfsetstrokecolor{currentstroke}%
\pgfsetdash{}{0pt}%
\pgfsys@defobject{currentmarker}{\pgfqpoint{0.000000in}{-0.083333in}}{\pgfqpoint{0.000000in}{0.000000in}}{%
\pgfpathmoveto{\pgfqpoint{0.000000in}{0.000000in}}%
\pgfpathlineto{\pgfqpoint{0.000000in}{-0.083333in}}%
\pgfusepath{stroke,fill}%
}%
\begin{pgfscope}%
\pgfsys@transformshift{0.548317in}{0.386884in}%
\pgfsys@useobject{currentmarker}{}%
\end{pgfscope}%
\end{pgfscope}%
\begin{pgfscope}%
\definecolor{textcolor}{rgb}{0.150000,0.150000,0.150000}%
\pgfsetstrokecolor{textcolor}%
\pgfsetfillcolor{textcolor}%
\pgftext[x=0.548317in,y=0.206329in,,top]{\color{textcolor}\sffamily\fontsize{10.000000}{12.000000}\selectfont \(\displaystyle 0.0\)}%
\end{pgfscope}%
\begin{pgfscope}%
\pgfsetbuttcap%
\pgfsetroundjoin%
\definecolor{currentfill}{rgb}{0.150000,0.150000,0.150000}%
\pgfsetfillcolor{currentfill}%
\pgfsetlinewidth{1.003750pt}%
\definecolor{currentstroke}{rgb}{0.150000,0.150000,0.150000}%
\pgfsetstrokecolor{currentstroke}%
\pgfsetdash{}{0pt}%
\pgfsys@defobject{currentmarker}{\pgfqpoint{0.000000in}{-0.083333in}}{\pgfqpoint{0.000000in}{0.000000in}}{%
\pgfpathmoveto{\pgfqpoint{0.000000in}{0.000000in}}%
\pgfpathlineto{\pgfqpoint{0.000000in}{-0.083333in}}%
\pgfusepath{stroke,fill}%
}%
\begin{pgfscope}%
\pgfsys@transformshift{1.263584in}{0.386884in}%
\pgfsys@useobject{currentmarker}{}%
\end{pgfscope}%
\end{pgfscope}%
\begin{pgfscope}%
\definecolor{textcolor}{rgb}{0.150000,0.150000,0.150000}%
\pgfsetstrokecolor{textcolor}%
\pgfsetfillcolor{textcolor}%
\pgftext[x=1.263584in,y=0.206329in,,top]{\color{textcolor}\sffamily\fontsize{10.000000}{12.000000}\selectfont \(\displaystyle 0.2\)}%
\end{pgfscope}%
\begin{pgfscope}%
\pgfsetbuttcap%
\pgfsetroundjoin%
\definecolor{currentfill}{rgb}{0.150000,0.150000,0.150000}%
\pgfsetfillcolor{currentfill}%
\pgfsetlinewidth{1.003750pt}%
\definecolor{currentstroke}{rgb}{0.150000,0.150000,0.150000}%
\pgfsetstrokecolor{currentstroke}%
\pgfsetdash{}{0pt}%
\pgfsys@defobject{currentmarker}{\pgfqpoint{0.000000in}{-0.083333in}}{\pgfqpoint{0.000000in}{0.000000in}}{%
\pgfpathmoveto{\pgfqpoint{0.000000in}{0.000000in}}%
\pgfpathlineto{\pgfqpoint{0.000000in}{-0.083333in}}%
\pgfusepath{stroke,fill}%
}%
\begin{pgfscope}%
\pgfsys@transformshift{1.978851in}{0.386884in}%
\pgfsys@useobject{currentmarker}{}%
\end{pgfscope}%
\end{pgfscope}%
\begin{pgfscope}%
\definecolor{textcolor}{rgb}{0.150000,0.150000,0.150000}%
\pgfsetstrokecolor{textcolor}%
\pgfsetfillcolor{textcolor}%
\pgftext[x=1.978851in,y=0.206329in,,top]{\color{textcolor}\sffamily\fontsize{10.000000}{12.000000}\selectfont \(\displaystyle 0.4\)}%
\end{pgfscope}%
\begin{pgfscope}%
\pgfsetbuttcap%
\pgfsetroundjoin%
\definecolor{currentfill}{rgb}{0.150000,0.150000,0.150000}%
\pgfsetfillcolor{currentfill}%
\pgfsetlinewidth{1.003750pt}%
\definecolor{currentstroke}{rgb}{0.150000,0.150000,0.150000}%
\pgfsetstrokecolor{currentstroke}%
\pgfsetdash{}{0pt}%
\pgfsys@defobject{currentmarker}{\pgfqpoint{0.000000in}{-0.083333in}}{\pgfqpoint{0.000000in}{0.000000in}}{%
\pgfpathmoveto{\pgfqpoint{0.000000in}{0.000000in}}%
\pgfpathlineto{\pgfqpoint{0.000000in}{-0.083333in}}%
\pgfusepath{stroke,fill}%
}%
\begin{pgfscope}%
\pgfsys@transformshift{2.694118in}{0.386884in}%
\pgfsys@useobject{currentmarker}{}%
\end{pgfscope}%
\end{pgfscope}%
\begin{pgfscope}%
\definecolor{textcolor}{rgb}{0.150000,0.150000,0.150000}%
\pgfsetstrokecolor{textcolor}%
\pgfsetfillcolor{textcolor}%
\pgftext[x=2.694118in,y=0.206329in,,top]{\color{textcolor}\sffamily\fontsize{10.000000}{12.000000}\selectfont \(\displaystyle 0.6\)}%
\end{pgfscope}%
\begin{pgfscope}%
\pgfsetbuttcap%
\pgfsetroundjoin%
\definecolor{currentfill}{rgb}{0.150000,0.150000,0.150000}%
\pgfsetfillcolor{currentfill}%
\pgfsetlinewidth{1.003750pt}%
\definecolor{currentstroke}{rgb}{0.150000,0.150000,0.150000}%
\pgfsetstrokecolor{currentstroke}%
\pgfsetdash{}{0pt}%
\pgfsys@defobject{currentmarker}{\pgfqpoint{0.000000in}{-0.083333in}}{\pgfqpoint{0.000000in}{0.000000in}}{%
\pgfpathmoveto{\pgfqpoint{0.000000in}{0.000000in}}%
\pgfpathlineto{\pgfqpoint{0.000000in}{-0.083333in}}%
\pgfusepath{stroke,fill}%
}%
\begin{pgfscope}%
\pgfsys@transformshift{3.409385in}{0.386884in}%
\pgfsys@useobject{currentmarker}{}%
\end{pgfscope}%
\end{pgfscope}%
\begin{pgfscope}%
\definecolor{textcolor}{rgb}{0.150000,0.150000,0.150000}%
\pgfsetstrokecolor{textcolor}%
\pgfsetfillcolor{textcolor}%
\pgftext[x=3.409385in,y=0.206329in,,top]{\color{textcolor}\sffamily\fontsize{10.000000}{12.000000}\selectfont \(\displaystyle 0.8\)}%
\end{pgfscope}%
\begin{pgfscope}%
\pgfsetbuttcap%
\pgfsetroundjoin%
\definecolor{currentfill}{rgb}{0.150000,0.150000,0.150000}%
\pgfsetfillcolor{currentfill}%
\pgfsetlinewidth{1.003750pt}%
\definecolor{currentstroke}{rgb}{0.150000,0.150000,0.150000}%
\pgfsetstrokecolor{currentstroke}%
\pgfsetdash{}{0pt}%
\pgfsys@defobject{currentmarker}{\pgfqpoint{0.000000in}{-0.083333in}}{\pgfqpoint{0.000000in}{0.000000in}}{%
\pgfpathmoveto{\pgfqpoint{0.000000in}{0.000000in}}%
\pgfpathlineto{\pgfqpoint{0.000000in}{-0.083333in}}%
\pgfusepath{stroke,fill}%
}%
\begin{pgfscope}%
\pgfsys@transformshift{4.124652in}{0.386884in}%
\pgfsys@useobject{currentmarker}{}%
\end{pgfscope}%
\end{pgfscope}%
\begin{pgfscope}%
\definecolor{textcolor}{rgb}{0.150000,0.150000,0.150000}%
\pgfsetstrokecolor{textcolor}%
\pgfsetfillcolor{textcolor}%
\pgftext[x=4.124652in,y=0.206329in,,top]{\color{textcolor}\sffamily\fontsize{10.000000}{12.000000}\selectfont \(\displaystyle 1.0\)}%
\end{pgfscope}%
\begin{pgfscope}%
\pgfsetbuttcap%
\pgfsetroundjoin%
\definecolor{currentfill}{rgb}{0.150000,0.150000,0.150000}%
\pgfsetfillcolor{currentfill}%
\pgfsetlinewidth{1.003750pt}%
\definecolor{currentstroke}{rgb}{0.150000,0.150000,0.150000}%
\pgfsetstrokecolor{currentstroke}%
\pgfsetdash{}{0pt}%
\pgfsys@defobject{currentmarker}{\pgfqpoint{-0.083333in}{0.000000in}}{\pgfqpoint{0.000000in}{0.000000in}}{%
\pgfpathmoveto{\pgfqpoint{0.000000in}{0.000000in}}%
\pgfpathlineto{\pgfqpoint{-0.083333in}{0.000000in}}%
\pgfusepath{stroke,fill}%
}%
\begin{pgfscope}%
\pgfsys@transformshift{0.548317in}{0.386884in}%
\pgfsys@useobject{currentmarker}{}%
\end{pgfscope}%
\end{pgfscope}%
\begin{pgfscope}%
\definecolor{textcolor}{rgb}{0.150000,0.150000,0.150000}%
\pgfsetstrokecolor{textcolor}%
\pgfsetfillcolor{textcolor}%
\pgftext[x=0.082267in,y=0.336742in,left,base]{\color{textcolor}\sffamily\fontsize{10.000000}{12.000000}\selectfont \(\displaystyle -1.0\)}%
\end{pgfscope}%
\begin{pgfscope}%
\pgfsetbuttcap%
\pgfsetroundjoin%
\definecolor{currentfill}{rgb}{0.150000,0.150000,0.150000}%
\pgfsetfillcolor{currentfill}%
\pgfsetlinewidth{1.003750pt}%
\definecolor{currentstroke}{rgb}{0.150000,0.150000,0.150000}%
\pgfsetstrokecolor{currentstroke}%
\pgfsetdash{}{0pt}%
\pgfsys@defobject{currentmarker}{\pgfqpoint{-0.083333in}{0.000000in}}{\pgfqpoint{0.000000in}{0.000000in}}{%
\pgfpathmoveto{\pgfqpoint{0.000000in}{0.000000in}}%
\pgfpathlineto{\pgfqpoint{-0.083333in}{0.000000in}}%
\pgfusepath{stroke,fill}%
}%
\begin{pgfscope}%
\pgfsys@transformshift{0.548317in}{0.912325in}%
\pgfsys@useobject{currentmarker}{}%
\end{pgfscope}%
\end{pgfscope}%
\begin{pgfscope}%
\definecolor{textcolor}{rgb}{0.150000,0.150000,0.150000}%
\pgfsetstrokecolor{textcolor}%
\pgfsetfillcolor{textcolor}%
\pgftext[x=0.082267in,y=0.862183in,left,base]{\color{textcolor}\sffamily\fontsize{10.000000}{12.000000}\selectfont \(\displaystyle -0.5\)}%
\end{pgfscope}%
\begin{pgfscope}%
\pgfsetbuttcap%
\pgfsetroundjoin%
\definecolor{currentfill}{rgb}{0.150000,0.150000,0.150000}%
\pgfsetfillcolor{currentfill}%
\pgfsetlinewidth{1.003750pt}%
\definecolor{currentstroke}{rgb}{0.150000,0.150000,0.150000}%
\pgfsetstrokecolor{currentstroke}%
\pgfsetdash{}{0pt}%
\pgfsys@defobject{currentmarker}{\pgfqpoint{-0.083333in}{0.000000in}}{\pgfqpoint{0.000000in}{0.000000in}}{%
\pgfpathmoveto{\pgfqpoint{0.000000in}{0.000000in}}%
\pgfpathlineto{\pgfqpoint{-0.083333in}{0.000000in}}%
\pgfusepath{stroke,fill}%
}%
\begin{pgfscope}%
\pgfsys@transformshift{0.548317in}{1.437766in}%
\pgfsys@useobject{currentmarker}{}%
\end{pgfscope}%
\end{pgfscope}%
\begin{pgfscope}%
\definecolor{textcolor}{rgb}{0.150000,0.150000,0.150000}%
\pgfsetstrokecolor{textcolor}%
\pgfsetfillcolor{textcolor}%
\pgftext[x=0.190292in,y=1.387624in,left,base]{\color{textcolor}\sffamily\fontsize{10.000000}{12.000000}\selectfont \(\displaystyle 0.0\)}%
\end{pgfscope}%
\begin{pgfscope}%
\pgfsetbuttcap%
\pgfsetroundjoin%
\definecolor{currentfill}{rgb}{0.150000,0.150000,0.150000}%
\pgfsetfillcolor{currentfill}%
\pgfsetlinewidth{1.003750pt}%
\definecolor{currentstroke}{rgb}{0.150000,0.150000,0.150000}%
\pgfsetstrokecolor{currentstroke}%
\pgfsetdash{}{0pt}%
\pgfsys@defobject{currentmarker}{\pgfqpoint{-0.083333in}{0.000000in}}{\pgfqpoint{0.000000in}{0.000000in}}{%
\pgfpathmoveto{\pgfqpoint{0.000000in}{0.000000in}}%
\pgfpathlineto{\pgfqpoint{-0.083333in}{0.000000in}}%
\pgfusepath{stroke,fill}%
}%
\begin{pgfscope}%
\pgfsys@transformshift{0.548317in}{1.963207in}%
\pgfsys@useobject{currentmarker}{}%
\end{pgfscope}%
\end{pgfscope}%
\begin{pgfscope}%
\definecolor{textcolor}{rgb}{0.150000,0.150000,0.150000}%
\pgfsetstrokecolor{textcolor}%
\pgfsetfillcolor{textcolor}%
\pgftext[x=0.190292in,y=1.913065in,left,base]{\color{textcolor}\sffamily\fontsize{10.000000}{12.000000}\selectfont \(\displaystyle 0.5\)}%
\end{pgfscope}%
\begin{pgfscope}%
\pgfsetbuttcap%
\pgfsetroundjoin%
\definecolor{currentfill}{rgb}{0.150000,0.150000,0.150000}%
\pgfsetfillcolor{currentfill}%
\pgfsetlinewidth{1.003750pt}%
\definecolor{currentstroke}{rgb}{0.150000,0.150000,0.150000}%
\pgfsetstrokecolor{currentstroke}%
\pgfsetdash{}{0pt}%
\pgfsys@defobject{currentmarker}{\pgfqpoint{-0.083333in}{0.000000in}}{\pgfqpoint{0.000000in}{0.000000in}}{%
\pgfpathmoveto{\pgfqpoint{0.000000in}{0.000000in}}%
\pgfpathlineto{\pgfqpoint{-0.083333in}{0.000000in}}%
\pgfusepath{stroke,fill}%
}%
\begin{pgfscope}%
\pgfsys@transformshift{0.548317in}{2.488647in}%
\pgfsys@useobject{currentmarker}{}%
\end{pgfscope}%
\end{pgfscope}%
\begin{pgfscope}%
\definecolor{textcolor}{rgb}{0.150000,0.150000,0.150000}%
\pgfsetstrokecolor{textcolor}%
\pgfsetfillcolor{textcolor}%
\pgftext[x=0.190292in,y=2.438505in,left,base]{\color{textcolor}\sffamily\fontsize{10.000000}{12.000000}\selectfont \(\displaystyle 1.0\)}%
\end{pgfscope}%
\begin{pgfscope}%
\pgfpathrectangle{\pgfqpoint{0.548317in}{0.386884in}}{\pgfqpoint{3.576335in}{2.101763in}} %
\pgfusepath{clip}%
\pgfsetbuttcap%
\pgfsetroundjoin%
\definecolor{currentfill}{rgb}{0.400000,0.760784,0.647059}%
\pgfsetfillcolor{currentfill}%
\pgfsetlinewidth{0.000000pt}%
\definecolor{currentstroke}{rgb}{0.400000,0.760784,0.647059}%
\pgfsetstrokecolor{currentstroke}%
\pgfsetdash{}{0pt}%
\pgfsys@defobject{currentmarker}{\pgfqpoint{-0.048611in}{-0.048611in}}{\pgfqpoint{0.048611in}{0.048611in}}{%
\pgfpathmoveto{\pgfqpoint{0.000000in}{-0.048611in}}%
\pgfpathcurveto{\pgfqpoint{0.012892in}{-0.048611in}}{\pgfqpoint{0.025257in}{-0.043489in}}{\pgfqpoint{0.034373in}{-0.034373in}}%
\pgfpathcurveto{\pgfqpoint{0.043489in}{-0.025257in}}{\pgfqpoint{0.048611in}{-0.012892in}}{\pgfqpoint{0.048611in}{0.000000in}}%
\pgfpathcurveto{\pgfqpoint{0.048611in}{0.012892in}}{\pgfqpoint{0.043489in}{0.025257in}}{\pgfqpoint{0.034373in}{0.034373in}}%
\pgfpathcurveto{\pgfqpoint{0.025257in}{0.043489in}}{\pgfqpoint{0.012892in}{0.048611in}}{\pgfqpoint{0.000000in}{0.048611in}}%
\pgfpathcurveto{\pgfqpoint{-0.012892in}{0.048611in}}{\pgfqpoint{-0.025257in}{0.043489in}}{\pgfqpoint{-0.034373in}{0.034373in}}%
\pgfpathcurveto{\pgfqpoint{-0.043489in}{0.025257in}}{\pgfqpoint{-0.048611in}{0.012892in}}{\pgfqpoint{-0.048611in}{0.000000in}}%
\pgfpathcurveto{\pgfqpoint{-0.048611in}{-0.012892in}}{\pgfqpoint{-0.043489in}{-0.025257in}}{\pgfqpoint{-0.034373in}{-0.034373in}}%
\pgfpathcurveto{\pgfqpoint{-0.025257in}{-0.043489in}}{\pgfqpoint{-0.012892in}{-0.048611in}}{\pgfqpoint{0.000000in}{-0.048611in}}%
\pgfpathclose%
\pgfusepath{fill}%
}%
\begin{pgfscope}%
\pgfsys@transformshift{2.260811in}{0.754316in}%
\pgfsys@useobject{currentmarker}{}%
\end{pgfscope}%
\begin{pgfscope}%
\pgfsys@transformshift{2.611975in}{0.899184in}%
\pgfsys@useobject{currentmarker}{}%
\end{pgfscope}%
\begin{pgfscope}%
\pgfsys@transformshift{3.012052in}{1.144101in}%
\pgfsys@useobject{currentmarker}{}%
\end{pgfscope}%
\end{pgfscope}%
\begin{pgfscope}%
\pgfpathrectangle{\pgfqpoint{0.548317in}{0.386884in}}{\pgfqpoint{3.576335in}{2.101763in}} %
\pgfusepath{clip}%
\pgfsetbuttcap%
\pgfsetroundjoin%
\pgfsetlinewidth{1.756562pt}%
\definecolor{currentstroke}{rgb}{0.988235,0.552941,0.384314}%
\pgfsetstrokecolor{currentstroke}%
\pgfsetdash{{5.600000pt}{2.400000pt}}{0.000000pt}%
\pgfpathmoveto{\pgfqpoint{0.548317in}{0.386884in}}%
\pgfpathlineto{\pgfqpoint{0.584442in}{0.397288in}}%
\pgfpathlineto{\pgfqpoint{0.620566in}{0.407283in}}%
\pgfpathlineto{\pgfqpoint{0.656691in}{0.416887in}}%
\pgfpathlineto{\pgfqpoint{0.692815in}{0.426121in}}%
\pgfpathlineto{\pgfqpoint{0.728940in}{0.435004in}}%
\pgfpathlineto{\pgfqpoint{0.765065in}{0.443556in}}%
\pgfpathlineto{\pgfqpoint{0.801189in}{0.451796in}}%
\pgfpathlineto{\pgfqpoint{0.837314in}{0.459743in}}%
\pgfpathlineto{\pgfqpoint{0.873438in}{0.467418in}}%
\pgfpathlineto{\pgfqpoint{0.909563in}{0.474839in}}%
\pgfpathlineto{\pgfqpoint{0.945688in}{0.482026in}}%
\pgfpathlineto{\pgfqpoint{0.981812in}{0.488999in}}%
\pgfpathlineto{\pgfqpoint{1.017937in}{0.495776in}}%
\pgfpathlineto{\pgfqpoint{1.054061in}{0.502379in}}%
\pgfpathlineto{\pgfqpoint{1.090186in}{0.508825in}}%
\pgfpathlineto{\pgfqpoint{1.126311in}{0.515135in}}%
\pgfpathlineto{\pgfqpoint{1.162435in}{0.521328in}}%
\pgfpathlineto{\pgfqpoint{1.198560in}{0.527423in}}%
\pgfpathlineto{\pgfqpoint{1.234684in}{0.533440in}}%
\pgfpathlineto{\pgfqpoint{1.270809in}{0.539399in}}%
\pgfpathlineto{\pgfqpoint{1.306933in}{0.545319in}}%
\pgfpathlineto{\pgfqpoint{1.343058in}{0.551220in}}%
\pgfpathlineto{\pgfqpoint{1.379183in}{0.557120in}}%
\pgfpathlineto{\pgfqpoint{1.415307in}{0.563040in}}%
\pgfpathlineto{\pgfqpoint{1.451432in}{0.568999in}}%
\pgfpathlineto{\pgfqpoint{1.487556in}{0.575017in}}%
\pgfpathlineto{\pgfqpoint{1.523681in}{0.581112in}}%
\pgfpathlineto{\pgfqpoint{1.559806in}{0.587305in}}%
\pgfpathlineto{\pgfqpoint{1.595930in}{0.593615in}}%
\pgfpathlineto{\pgfqpoint{1.632055in}{0.600061in}}%
\pgfpathlineto{\pgfqpoint{1.668179in}{0.606663in}}%
\pgfpathlineto{\pgfqpoint{1.704304in}{0.613441in}}%
\pgfpathlineto{\pgfqpoint{1.740429in}{0.620414in}}%
\pgfpathlineto{\pgfqpoint{1.776553in}{0.627601in}}%
\pgfpathlineto{\pgfqpoint{1.812678in}{0.635022in}}%
\pgfpathlineto{\pgfqpoint{1.848802in}{0.642696in}}%
\pgfpathlineto{\pgfqpoint{1.884927in}{0.650644in}}%
\pgfpathlineto{\pgfqpoint{1.921052in}{0.658884in}}%
\pgfpathlineto{\pgfqpoint{1.957176in}{0.667435in}}%
\pgfpathlineto{\pgfqpoint{1.993301in}{0.676318in}}%
\pgfpathlineto{\pgfqpoint{2.029425in}{0.685553in}}%
\pgfpathlineto{\pgfqpoint{2.065550in}{0.695157in}}%
\pgfpathlineto{\pgfqpoint{2.101675in}{0.705151in}}%
\pgfpathlineto{\pgfqpoint{2.137799in}{0.715555in}}%
\pgfpathlineto{\pgfqpoint{2.173924in}{0.726388in}}%
\pgfpathlineto{\pgfqpoint{2.210048in}{0.737669in}}%
\pgfpathlineto{\pgfqpoint{2.246173in}{0.749418in}}%
\pgfpathlineto{\pgfqpoint{2.282298in}{0.761654in}}%
\pgfpathlineto{\pgfqpoint{2.318422in}{0.774397in}}%
\pgfpathlineto{\pgfqpoint{2.354547in}{0.787667in}}%
\pgfpathlineto{\pgfqpoint{2.390671in}{0.801482in}}%
\pgfpathlineto{\pgfqpoint{2.426796in}{0.815863in}}%
\pgfpathlineto{\pgfqpoint{2.462921in}{0.830828in}}%
\pgfpathlineto{\pgfqpoint{2.499045in}{0.846398in}}%
\pgfpathlineto{\pgfqpoint{2.535170in}{0.862592in}}%
\pgfpathlineto{\pgfqpoint{2.571294in}{0.879429in}}%
\pgfpathlineto{\pgfqpoint{2.607419in}{0.896929in}}%
\pgfpathlineto{\pgfqpoint{2.643543in}{0.915111in}}%
\pgfpathlineto{\pgfqpoint{2.679668in}{0.933995in}}%
\pgfpathlineto{\pgfqpoint{2.715793in}{0.953601in}}%
\pgfpathlineto{\pgfqpoint{2.751917in}{0.973947in}}%
\pgfpathlineto{\pgfqpoint{2.788042in}{0.995053in}}%
\pgfpathlineto{\pgfqpoint{2.824166in}{1.016940in}}%
\pgfpathlineto{\pgfqpoint{2.860291in}{1.039625in}}%
\pgfpathlineto{\pgfqpoint{2.896416in}{1.063129in}}%
\pgfpathlineto{\pgfqpoint{2.932540in}{1.087472in}}%
\pgfpathlineto{\pgfqpoint{2.968665in}{1.112672in}}%
\pgfpathlineto{\pgfqpoint{3.004789in}{1.138750in}}%
\pgfpathlineto{\pgfqpoint{3.040914in}{1.165725in}}%
\pgfpathlineto{\pgfqpoint{3.077039in}{1.193615in}}%
\pgfpathlineto{\pgfqpoint{3.113163in}{1.222442in}}%
\pgfpathlineto{\pgfqpoint{3.149288in}{1.252223in}}%
\pgfpathlineto{\pgfqpoint{3.185412in}{1.282980in}}%
\pgfpathlineto{\pgfqpoint{3.221537in}{1.314730in}}%
\pgfpathlineto{\pgfqpoint{3.257662in}{1.347495in}}%
\pgfpathlineto{\pgfqpoint{3.293786in}{1.381293in}}%
\pgfpathlineto{\pgfqpoint{3.329911in}{1.416143in}}%
\pgfpathlineto{\pgfqpoint{3.366035in}{1.452065in}}%
\pgfpathlineto{\pgfqpoint{3.402160in}{1.489080in}}%
\pgfpathlineto{\pgfqpoint{3.438285in}{1.527205in}}%
\pgfpathlineto{\pgfqpoint{3.474409in}{1.566461in}}%
\pgfpathlineto{\pgfqpoint{3.510534in}{1.606868in}}%
\pgfpathlineto{\pgfqpoint{3.546658in}{1.648444in}}%
\pgfpathlineto{\pgfqpoint{3.582783in}{1.691209in}}%
\pgfpathlineto{\pgfqpoint{3.618908in}{1.735183in}}%
\pgfpathlineto{\pgfqpoint{3.655032in}{1.780385in}}%
\pgfpathlineto{\pgfqpoint{3.691157in}{1.826835in}}%
\pgfpathlineto{\pgfqpoint{3.727281in}{1.874552in}}%
\pgfpathlineto{\pgfqpoint{3.763406in}{1.923556in}}%
\pgfpathlineto{\pgfqpoint{3.799531in}{1.973866in}}%
\pgfpathlineto{\pgfqpoint{3.835655in}{2.025501in}}%
\pgfpathlineto{\pgfqpoint{3.871780in}{2.078482in}}%
\pgfpathlineto{\pgfqpoint{3.907904in}{2.132827in}}%
\pgfpathlineto{\pgfqpoint{3.944029in}{2.188556in}}%
\pgfpathlineto{\pgfqpoint{3.980154in}{2.245689in}}%
\pgfpathlineto{\pgfqpoint{4.016278in}{2.304245in}}%
\pgfpathlineto{\pgfqpoint{4.052403in}{2.364244in}}%
\pgfpathlineto{\pgfqpoint{4.088527in}{2.425705in}}%
\pgfpathlineto{\pgfqpoint{4.124652in}{2.488647in}}%
\pgfusepath{stroke}%
\end{pgfscope}%
\begin{pgfscope}%
\pgfpathrectangle{\pgfqpoint{0.548317in}{0.386884in}}{\pgfqpoint{3.576335in}{2.101763in}} %
\pgfusepath{clip}%
\pgfsetroundcap%
\pgfsetroundjoin%
\pgfsetlinewidth{1.756562pt}%
\definecolor{currentstroke}{rgb}{0.552941,0.627451,0.796078}%
\pgfsetstrokecolor{currentstroke}%
\pgfsetdash{}{0pt}%
\pgfpathmoveto{\pgfqpoint{0.548317in}{0.950836in}}%
\pgfpathlineto{\pgfqpoint{0.584442in}{0.945928in}}%
\pgfpathlineto{\pgfqpoint{0.620566in}{0.940953in}}%
\pgfpathlineto{\pgfqpoint{0.656691in}{0.935910in}}%
\pgfpathlineto{\pgfqpoint{0.692815in}{0.930800in}}%
\pgfpathlineto{\pgfqpoint{0.728940in}{0.925622in}}%
\pgfpathlineto{\pgfqpoint{0.765065in}{0.920379in}}%
\pgfpathlineto{\pgfqpoint{0.801189in}{0.915069in}}%
\pgfpathlineto{\pgfqpoint{0.837314in}{0.909696in}}%
\pgfpathlineto{\pgfqpoint{0.873438in}{0.904259in}}%
\pgfpathlineto{\pgfqpoint{0.909563in}{0.898761in}}%
\pgfpathlineto{\pgfqpoint{0.945688in}{0.893204in}}%
\pgfpathlineto{\pgfqpoint{0.981812in}{0.887590in}}%
\pgfpathlineto{\pgfqpoint{1.017937in}{0.881921in}}%
\pgfpathlineto{\pgfqpoint{1.054061in}{0.876202in}}%
\pgfpathlineto{\pgfqpoint{1.090186in}{0.870436in}}%
\pgfpathlineto{\pgfqpoint{1.126311in}{0.864626in}}%
\pgfpathlineto{\pgfqpoint{1.162435in}{0.858779in}}%
\pgfpathlineto{\pgfqpoint{1.198560in}{0.852899in}}%
\pgfpathlineto{\pgfqpoint{1.234684in}{0.846994in}}%
\pgfpathlineto{\pgfqpoint{1.270809in}{0.841069in}}%
\pgfpathlineto{\pgfqpoint{1.306933in}{0.835135in}}%
\pgfpathlineto{\pgfqpoint{1.343058in}{0.829199in}}%
\pgfpathlineto{\pgfqpoint{1.379183in}{0.823273in}}%
\pgfpathlineto{\pgfqpoint{1.415307in}{0.817368in}}%
\pgfpathlineto{\pgfqpoint{1.451432in}{0.811497in}}%
\pgfpathlineto{\pgfqpoint{1.487556in}{0.805676in}}%
\pgfpathlineto{\pgfqpoint{1.523681in}{0.799921in}}%
\pgfpathlineto{\pgfqpoint{1.559806in}{0.794250in}}%
\pgfpathlineto{\pgfqpoint{1.595930in}{0.788685in}}%
\pgfpathlineto{\pgfqpoint{1.632055in}{0.783249in}}%
\pgfpathlineto{\pgfqpoint{1.668179in}{0.777968in}}%
\pgfpathlineto{\pgfqpoint{1.704304in}{0.772872in}}%
\pgfpathlineto{\pgfqpoint{1.740429in}{0.767992in}}%
\pgfpathlineto{\pgfqpoint{1.776553in}{0.763365in}}%
\pgfpathlineto{\pgfqpoint{1.812678in}{0.759031in}}%
\pgfpathlineto{\pgfqpoint{1.848802in}{0.755035in}}%
\pgfpathlineto{\pgfqpoint{1.884927in}{0.751427in}}%
\pgfpathlineto{\pgfqpoint{1.921052in}{0.748261in}}%
\pgfpathlineto{\pgfqpoint{1.957176in}{0.745600in}}%
\pgfpathlineto{\pgfqpoint{1.993301in}{0.743510in}}%
\pgfpathlineto{\pgfqpoint{2.029425in}{0.742066in}}%
\pgfpathlineto{\pgfqpoint{2.065550in}{0.741350in}}%
\pgfpathlineto{\pgfqpoint{2.101675in}{0.741452in}}%
\pgfpathlineto{\pgfqpoint{2.137799in}{0.742469in}}%
\pgfpathlineto{\pgfqpoint{2.173924in}{0.744509in}}%
\pgfpathlineto{\pgfqpoint{2.210048in}{0.747688in}}%
\pgfpathlineto{\pgfqpoint{2.246173in}{0.752128in}}%
\pgfpathlineto{\pgfqpoint{2.282298in}{0.757962in}}%
\pgfpathlineto{\pgfqpoint{2.318422in}{0.765329in}}%
\pgfpathlineto{\pgfqpoint{2.354547in}{0.774374in}}%
\pgfpathlineto{\pgfqpoint{2.390671in}{0.785242in}}%
\pgfpathlineto{\pgfqpoint{2.426796in}{0.798080in}}%
\pgfpathlineto{\pgfqpoint{2.462921in}{0.813027in}}%
\pgfpathlineto{\pgfqpoint{2.499045in}{0.830208in}}%
\pgfpathlineto{\pgfqpoint{2.535170in}{0.849723in}}%
\pgfpathlineto{\pgfqpoint{2.571294in}{0.871634in}}%
\pgfpathlineto{\pgfqpoint{2.607419in}{0.895949in}}%
\pgfpathlineto{\pgfqpoint{2.643543in}{0.922596in}}%
\pgfpathlineto{\pgfqpoint{2.679668in}{0.951392in}}%
\pgfpathlineto{\pgfqpoint{2.715793in}{0.982006in}}%
\pgfpathlineto{\pgfqpoint{2.751917in}{1.013910in}}%
\pgfpathlineto{\pgfqpoint{2.788042in}{1.046310in}}%
\pgfpathlineto{\pgfqpoint{2.824166in}{1.078072in}}%
\pgfpathlineto{\pgfqpoint{2.860291in}{1.107616in}}%
\pgfpathlineto{\pgfqpoint{2.896416in}{1.132793in}}%
\pgfpathlineto{\pgfqpoint{2.932540in}{1.150720in}}%
\pgfpathlineto{\pgfqpoint{2.968665in}{1.157590in}}%
\pgfpathlineto{\pgfqpoint{3.004789in}{1.148422in}}%
\pgfpathlineto{\pgfqpoint{3.040914in}{1.116762in}}%
\pgfpathlineto{\pgfqpoint{3.077039in}{1.054310in}}%
\pgfpathlineto{\pgfqpoint{3.113163in}{0.950463in}}%
\pgfpathlineto{\pgfqpoint{3.149288in}{0.791758in}}%
\pgfpathlineto{\pgfqpoint{3.185412in}{0.561190in}}%
\pgfpathlineto{\pgfqpoint{3.206408in}{0.372996in}}%
\pgfusepath{stroke}%
\end{pgfscope}%
\begin{pgfscope}%
\pgfsetrectcap%
\pgfsetmiterjoin%
\pgfsetlinewidth{1.254687pt}%
\definecolor{currentstroke}{rgb}{0.150000,0.150000,0.150000}%
\pgfsetstrokecolor{currentstroke}%
\pgfsetdash{}{0pt}%
\pgfpathmoveto{\pgfqpoint{0.548317in}{0.386884in}}%
\pgfpathlineto{\pgfqpoint{0.548317in}{2.488647in}}%
\pgfusepath{stroke}%
\end{pgfscope}%
\begin{pgfscope}%
\pgfsetrectcap%
\pgfsetmiterjoin%
\pgfsetlinewidth{1.254687pt}%
\definecolor{currentstroke}{rgb}{0.150000,0.150000,0.150000}%
\pgfsetstrokecolor{currentstroke}%
\pgfsetdash{}{0pt}%
\pgfpathmoveto{\pgfqpoint{0.548317in}{0.386884in}}%
\pgfpathlineto{\pgfqpoint{4.124652in}{0.386884in}}%
\pgfusepath{stroke}%
\end{pgfscope}%
\begin{pgfscope}%
\pgfsetbuttcap%
\pgfsetroundjoin%
\definecolor{currentfill}{rgb}{0.400000,0.760784,0.647059}%
\pgfsetfillcolor{currentfill}%
\pgfsetlinewidth{0.000000pt}%
\definecolor{currentstroke}{rgb}{0.400000,0.760784,0.647059}%
\pgfsetstrokecolor{currentstroke}%
\pgfsetdash{}{0pt}%
\pgfsys@defobject{currentmarker}{\pgfqpoint{-0.048611in}{-0.048611in}}{\pgfqpoint{0.048611in}{0.048611in}}{%
\pgfpathmoveto{\pgfqpoint{0.000000in}{-0.048611in}}%
\pgfpathcurveto{\pgfqpoint{0.012892in}{-0.048611in}}{\pgfqpoint{0.025257in}{-0.043489in}}{\pgfqpoint{0.034373in}{-0.034373in}}%
\pgfpathcurveto{\pgfqpoint{0.043489in}{-0.025257in}}{\pgfqpoint{0.048611in}{-0.012892in}}{\pgfqpoint{0.048611in}{0.000000in}}%
\pgfpathcurveto{\pgfqpoint{0.048611in}{0.012892in}}{\pgfqpoint{0.043489in}{0.025257in}}{\pgfqpoint{0.034373in}{0.034373in}}%
\pgfpathcurveto{\pgfqpoint{0.025257in}{0.043489in}}{\pgfqpoint{0.012892in}{0.048611in}}{\pgfqpoint{0.000000in}{0.048611in}}%
\pgfpathcurveto{\pgfqpoint{-0.012892in}{0.048611in}}{\pgfqpoint{-0.025257in}{0.043489in}}{\pgfqpoint{-0.034373in}{0.034373in}}%
\pgfpathcurveto{\pgfqpoint{-0.043489in}{0.025257in}}{\pgfqpoint{-0.048611in}{0.012892in}}{\pgfqpoint{-0.048611in}{0.000000in}}%
\pgfpathcurveto{\pgfqpoint{-0.048611in}{-0.012892in}}{\pgfqpoint{-0.043489in}{-0.025257in}}{\pgfqpoint{-0.034373in}{-0.034373in}}%
\pgfpathcurveto{\pgfqpoint{-0.025257in}{-0.043489in}}{\pgfqpoint{-0.012892in}{-0.048611in}}{\pgfqpoint{0.000000in}{-0.048611in}}%
\pgfpathclose%
\pgfusepath{fill}%
}%
\begin{pgfscope}%
\pgfsys@transformshift{0.812206in}{2.311974in}%
\pgfsys@useobject{currentmarker}{}%
\end{pgfscope}%
\end{pgfscope}%
\begin{pgfscope}%
\definecolor{textcolor}{rgb}{0.150000,0.150000,0.150000}%
\pgfsetstrokecolor{textcolor}%
\pgfsetfillcolor{textcolor}%
\pgftext[x=1.062206in,y=2.263363in,left,base]{\color{textcolor}\sffamily\fontsize{10.000000}{12.000000}\selectfont samples}%
\end{pgfscope}%
\begin{pgfscope}%
\pgfsetbuttcap%
\pgfsetroundjoin%
\pgfsetlinewidth{1.756562pt}%
\definecolor{currentstroke}{rgb}{0.988235,0.552941,0.384314}%
\pgfsetstrokecolor{currentstroke}%
\pgfsetdash{{5.600000pt}{2.400000pt}}{0.000000pt}%
\pgfpathmoveto{\pgfqpoint{0.673317in}{2.115246in}}%
\pgfpathlineto{\pgfqpoint{0.951095in}{2.115246in}}%
\pgfusepath{stroke}%
\end{pgfscope}%
\begin{pgfscope}%
\definecolor{textcolor}{rgb}{0.150000,0.150000,0.150000}%
\pgfsetstrokecolor{textcolor}%
\pgfsetfillcolor{textcolor}%
\pgftext[x=1.062206in,y=2.066635in,left,base]{\color{textcolor}\sffamily\fontsize{10.000000}{12.000000}\selectfont function}%
\end{pgfscope}%
\begin{pgfscope}%
\pgfsetroundcap%
\pgfsetroundjoin%
\pgfsetlinewidth{1.756562pt}%
\definecolor{currentstroke}{rgb}{0.552941,0.627451,0.796078}%
\pgfsetstrokecolor{currentstroke}%
\pgfsetdash{}{0pt}%
\pgfpathmoveto{\pgfqpoint{0.673317in}{1.918518in}}%
\pgfpathlineto{\pgfqpoint{0.951095in}{1.918518in}}%
\pgfusepath{stroke}%
\end{pgfscope}%
\begin{pgfscope}%
\definecolor{textcolor}{rgb}{0.150000,0.150000,0.150000}%
\pgfsetstrokecolor{textcolor}%
\pgfsetfillcolor{textcolor}%
\pgftext[x=1.062206in,y=1.869907in,left,base]{\color{textcolor}\sffamily\fontsize{10.000000}{12.000000}\selectfont 20th order fit}%
\end{pgfscope}%
\end{pgfpicture}%
\makeatother%
\endgroup%
}
			\caption{\engordnumber{20}-order polynomial fitting 3 points}
			\label{fig:polyfit20th}
		\end{subfigure}\\	
		\begin{subfigure}[t]{0.49\textwidth}
			\resizebox{\linewidth}{!}{%% Creator: Matplotlib, PGF backend
%%
%% To include the figure in your LaTeX document, write
%%   \input{<filename>.pgf}
%%
%% Make sure the required packages are loaded in your preamble
%%   \usepackage{pgf}
%%
%% Figures using additional raster images can only be included by \input if
%% they are in the same directory as the main LaTeX file. For loading figures
%% from other directories you can use the `import` package
%%   \usepackage{import}
%% and then include the figures with
%%   \import{<path to file>}{<filename>.pgf}
%%
%% Matplotlib used the following preamble
%%   \usepackage[utf8x]{inputenc}
%%   \usepackage[T1]{fontenc}
%%
\begingroup%
\makeatletter%
\begin{pgfpicture}%
\pgfpathrectangle{\pgfpointorigin}{\pgfqpoint{4.296389in}{2.655314in}}%
\pgfusepath{use as bounding box, clip}%
\begin{pgfscope}%
\pgfsetbuttcap%
\pgfsetmiterjoin%
\definecolor{currentfill}{rgb}{1.000000,1.000000,1.000000}%
\pgfsetfillcolor{currentfill}%
\pgfsetlinewidth{0.000000pt}%
\definecolor{currentstroke}{rgb}{1.000000,1.000000,1.000000}%
\pgfsetstrokecolor{currentstroke}%
\pgfsetdash{}{0pt}%
\pgfpathmoveto{\pgfqpoint{0.000000in}{0.000000in}}%
\pgfpathlineto{\pgfqpoint{4.296389in}{0.000000in}}%
\pgfpathlineto{\pgfqpoint{4.296389in}{2.655314in}}%
\pgfpathlineto{\pgfqpoint{0.000000in}{2.655314in}}%
\pgfpathclose%
\pgfusepath{fill}%
\end{pgfscope}%
\begin{pgfscope}%
\pgfsetbuttcap%
\pgfsetmiterjoin%
\definecolor{currentfill}{rgb}{1.000000,1.000000,1.000000}%
\pgfsetfillcolor{currentfill}%
\pgfsetlinewidth{0.000000pt}%
\definecolor{currentstroke}{rgb}{0.000000,0.000000,0.000000}%
\pgfsetstrokecolor{currentstroke}%
\pgfsetstrokeopacity{0.000000}%
\pgfsetdash{}{0pt}%
\pgfpathmoveto{\pgfqpoint{0.548317in}{0.386884in}}%
\pgfpathlineto{\pgfqpoint{4.124652in}{0.386884in}}%
\pgfpathlineto{\pgfqpoint{4.124652in}{2.488647in}}%
\pgfpathlineto{\pgfqpoint{0.548317in}{2.488647in}}%
\pgfpathclose%
\pgfusepath{fill}%
\end{pgfscope}%
\begin{pgfscope}%
\pgfsetbuttcap%
\pgfsetroundjoin%
\definecolor{currentfill}{rgb}{0.150000,0.150000,0.150000}%
\pgfsetfillcolor{currentfill}%
\pgfsetlinewidth{1.003750pt}%
\definecolor{currentstroke}{rgb}{0.150000,0.150000,0.150000}%
\pgfsetstrokecolor{currentstroke}%
\pgfsetdash{}{0pt}%
\pgfsys@defobject{currentmarker}{\pgfqpoint{0.000000in}{-0.083333in}}{\pgfqpoint{0.000000in}{0.000000in}}{%
\pgfpathmoveto{\pgfqpoint{0.000000in}{0.000000in}}%
\pgfpathlineto{\pgfqpoint{0.000000in}{-0.083333in}}%
\pgfusepath{stroke,fill}%
}%
\begin{pgfscope}%
\pgfsys@transformshift{0.548317in}{0.386884in}%
\pgfsys@useobject{currentmarker}{}%
\end{pgfscope}%
\end{pgfscope}%
\begin{pgfscope}%
\definecolor{textcolor}{rgb}{0.150000,0.150000,0.150000}%
\pgfsetstrokecolor{textcolor}%
\pgfsetfillcolor{textcolor}%
\pgftext[x=0.548317in,y=0.206329in,,top]{\color{textcolor}\sffamily\fontsize{10.000000}{12.000000}\selectfont \(\displaystyle 0.0\)}%
\end{pgfscope}%
\begin{pgfscope}%
\pgfsetbuttcap%
\pgfsetroundjoin%
\definecolor{currentfill}{rgb}{0.150000,0.150000,0.150000}%
\pgfsetfillcolor{currentfill}%
\pgfsetlinewidth{1.003750pt}%
\definecolor{currentstroke}{rgb}{0.150000,0.150000,0.150000}%
\pgfsetstrokecolor{currentstroke}%
\pgfsetdash{}{0pt}%
\pgfsys@defobject{currentmarker}{\pgfqpoint{0.000000in}{-0.083333in}}{\pgfqpoint{0.000000in}{0.000000in}}{%
\pgfpathmoveto{\pgfqpoint{0.000000in}{0.000000in}}%
\pgfpathlineto{\pgfqpoint{0.000000in}{-0.083333in}}%
\pgfusepath{stroke,fill}%
}%
\begin{pgfscope}%
\pgfsys@transformshift{1.263584in}{0.386884in}%
\pgfsys@useobject{currentmarker}{}%
\end{pgfscope}%
\end{pgfscope}%
\begin{pgfscope}%
\definecolor{textcolor}{rgb}{0.150000,0.150000,0.150000}%
\pgfsetstrokecolor{textcolor}%
\pgfsetfillcolor{textcolor}%
\pgftext[x=1.263584in,y=0.206329in,,top]{\color{textcolor}\sffamily\fontsize{10.000000}{12.000000}\selectfont \(\displaystyle 0.2\)}%
\end{pgfscope}%
\begin{pgfscope}%
\pgfsetbuttcap%
\pgfsetroundjoin%
\definecolor{currentfill}{rgb}{0.150000,0.150000,0.150000}%
\pgfsetfillcolor{currentfill}%
\pgfsetlinewidth{1.003750pt}%
\definecolor{currentstroke}{rgb}{0.150000,0.150000,0.150000}%
\pgfsetstrokecolor{currentstroke}%
\pgfsetdash{}{0pt}%
\pgfsys@defobject{currentmarker}{\pgfqpoint{0.000000in}{-0.083333in}}{\pgfqpoint{0.000000in}{0.000000in}}{%
\pgfpathmoveto{\pgfqpoint{0.000000in}{0.000000in}}%
\pgfpathlineto{\pgfqpoint{0.000000in}{-0.083333in}}%
\pgfusepath{stroke,fill}%
}%
\begin{pgfscope}%
\pgfsys@transformshift{1.978851in}{0.386884in}%
\pgfsys@useobject{currentmarker}{}%
\end{pgfscope}%
\end{pgfscope}%
\begin{pgfscope}%
\definecolor{textcolor}{rgb}{0.150000,0.150000,0.150000}%
\pgfsetstrokecolor{textcolor}%
\pgfsetfillcolor{textcolor}%
\pgftext[x=1.978851in,y=0.206329in,,top]{\color{textcolor}\sffamily\fontsize{10.000000}{12.000000}\selectfont \(\displaystyle 0.4\)}%
\end{pgfscope}%
\begin{pgfscope}%
\pgfsetbuttcap%
\pgfsetroundjoin%
\definecolor{currentfill}{rgb}{0.150000,0.150000,0.150000}%
\pgfsetfillcolor{currentfill}%
\pgfsetlinewidth{1.003750pt}%
\definecolor{currentstroke}{rgb}{0.150000,0.150000,0.150000}%
\pgfsetstrokecolor{currentstroke}%
\pgfsetdash{}{0pt}%
\pgfsys@defobject{currentmarker}{\pgfqpoint{0.000000in}{-0.083333in}}{\pgfqpoint{0.000000in}{0.000000in}}{%
\pgfpathmoveto{\pgfqpoint{0.000000in}{0.000000in}}%
\pgfpathlineto{\pgfqpoint{0.000000in}{-0.083333in}}%
\pgfusepath{stroke,fill}%
}%
\begin{pgfscope}%
\pgfsys@transformshift{2.694118in}{0.386884in}%
\pgfsys@useobject{currentmarker}{}%
\end{pgfscope}%
\end{pgfscope}%
\begin{pgfscope}%
\definecolor{textcolor}{rgb}{0.150000,0.150000,0.150000}%
\pgfsetstrokecolor{textcolor}%
\pgfsetfillcolor{textcolor}%
\pgftext[x=2.694118in,y=0.206329in,,top]{\color{textcolor}\sffamily\fontsize{10.000000}{12.000000}\selectfont \(\displaystyle 0.6\)}%
\end{pgfscope}%
\begin{pgfscope}%
\pgfsetbuttcap%
\pgfsetroundjoin%
\definecolor{currentfill}{rgb}{0.150000,0.150000,0.150000}%
\pgfsetfillcolor{currentfill}%
\pgfsetlinewidth{1.003750pt}%
\definecolor{currentstroke}{rgb}{0.150000,0.150000,0.150000}%
\pgfsetstrokecolor{currentstroke}%
\pgfsetdash{}{0pt}%
\pgfsys@defobject{currentmarker}{\pgfqpoint{0.000000in}{-0.083333in}}{\pgfqpoint{0.000000in}{0.000000in}}{%
\pgfpathmoveto{\pgfqpoint{0.000000in}{0.000000in}}%
\pgfpathlineto{\pgfqpoint{0.000000in}{-0.083333in}}%
\pgfusepath{stroke,fill}%
}%
\begin{pgfscope}%
\pgfsys@transformshift{3.409385in}{0.386884in}%
\pgfsys@useobject{currentmarker}{}%
\end{pgfscope}%
\end{pgfscope}%
\begin{pgfscope}%
\definecolor{textcolor}{rgb}{0.150000,0.150000,0.150000}%
\pgfsetstrokecolor{textcolor}%
\pgfsetfillcolor{textcolor}%
\pgftext[x=3.409385in,y=0.206329in,,top]{\color{textcolor}\sffamily\fontsize{10.000000}{12.000000}\selectfont \(\displaystyle 0.8\)}%
\end{pgfscope}%
\begin{pgfscope}%
\pgfsetbuttcap%
\pgfsetroundjoin%
\definecolor{currentfill}{rgb}{0.150000,0.150000,0.150000}%
\pgfsetfillcolor{currentfill}%
\pgfsetlinewidth{1.003750pt}%
\definecolor{currentstroke}{rgb}{0.150000,0.150000,0.150000}%
\pgfsetstrokecolor{currentstroke}%
\pgfsetdash{}{0pt}%
\pgfsys@defobject{currentmarker}{\pgfqpoint{0.000000in}{-0.083333in}}{\pgfqpoint{0.000000in}{0.000000in}}{%
\pgfpathmoveto{\pgfqpoint{0.000000in}{0.000000in}}%
\pgfpathlineto{\pgfqpoint{0.000000in}{-0.083333in}}%
\pgfusepath{stroke,fill}%
}%
\begin{pgfscope}%
\pgfsys@transformshift{4.124652in}{0.386884in}%
\pgfsys@useobject{currentmarker}{}%
\end{pgfscope}%
\end{pgfscope}%
\begin{pgfscope}%
\definecolor{textcolor}{rgb}{0.150000,0.150000,0.150000}%
\pgfsetstrokecolor{textcolor}%
\pgfsetfillcolor{textcolor}%
\pgftext[x=4.124652in,y=0.206329in,,top]{\color{textcolor}\sffamily\fontsize{10.000000}{12.000000}\selectfont \(\displaystyle 1.0\)}%
\end{pgfscope}%
\begin{pgfscope}%
\pgfsetbuttcap%
\pgfsetroundjoin%
\definecolor{currentfill}{rgb}{0.150000,0.150000,0.150000}%
\pgfsetfillcolor{currentfill}%
\pgfsetlinewidth{1.003750pt}%
\definecolor{currentstroke}{rgb}{0.150000,0.150000,0.150000}%
\pgfsetstrokecolor{currentstroke}%
\pgfsetdash{}{0pt}%
\pgfsys@defobject{currentmarker}{\pgfqpoint{-0.083333in}{0.000000in}}{\pgfqpoint{0.000000in}{0.000000in}}{%
\pgfpathmoveto{\pgfqpoint{0.000000in}{0.000000in}}%
\pgfpathlineto{\pgfqpoint{-0.083333in}{0.000000in}}%
\pgfusepath{stroke,fill}%
}%
\begin{pgfscope}%
\pgfsys@transformshift{0.548317in}{0.386884in}%
\pgfsys@useobject{currentmarker}{}%
\end{pgfscope}%
\end{pgfscope}%
\begin{pgfscope}%
\definecolor{textcolor}{rgb}{0.150000,0.150000,0.150000}%
\pgfsetstrokecolor{textcolor}%
\pgfsetfillcolor{textcolor}%
\pgftext[x=0.082267in,y=0.336742in,left,base]{\color{textcolor}\sffamily\fontsize{10.000000}{12.000000}\selectfont \(\displaystyle -1.0\)}%
\end{pgfscope}%
\begin{pgfscope}%
\pgfsetbuttcap%
\pgfsetroundjoin%
\definecolor{currentfill}{rgb}{0.150000,0.150000,0.150000}%
\pgfsetfillcolor{currentfill}%
\pgfsetlinewidth{1.003750pt}%
\definecolor{currentstroke}{rgb}{0.150000,0.150000,0.150000}%
\pgfsetstrokecolor{currentstroke}%
\pgfsetdash{}{0pt}%
\pgfsys@defobject{currentmarker}{\pgfqpoint{-0.083333in}{0.000000in}}{\pgfqpoint{0.000000in}{0.000000in}}{%
\pgfpathmoveto{\pgfqpoint{0.000000in}{0.000000in}}%
\pgfpathlineto{\pgfqpoint{-0.083333in}{0.000000in}}%
\pgfusepath{stroke,fill}%
}%
\begin{pgfscope}%
\pgfsys@transformshift{0.548317in}{0.912325in}%
\pgfsys@useobject{currentmarker}{}%
\end{pgfscope}%
\end{pgfscope}%
\begin{pgfscope}%
\definecolor{textcolor}{rgb}{0.150000,0.150000,0.150000}%
\pgfsetstrokecolor{textcolor}%
\pgfsetfillcolor{textcolor}%
\pgftext[x=0.082267in,y=0.862183in,left,base]{\color{textcolor}\sffamily\fontsize{10.000000}{12.000000}\selectfont \(\displaystyle -0.5\)}%
\end{pgfscope}%
\begin{pgfscope}%
\pgfsetbuttcap%
\pgfsetroundjoin%
\definecolor{currentfill}{rgb}{0.150000,0.150000,0.150000}%
\pgfsetfillcolor{currentfill}%
\pgfsetlinewidth{1.003750pt}%
\definecolor{currentstroke}{rgb}{0.150000,0.150000,0.150000}%
\pgfsetstrokecolor{currentstroke}%
\pgfsetdash{}{0pt}%
\pgfsys@defobject{currentmarker}{\pgfqpoint{-0.083333in}{0.000000in}}{\pgfqpoint{0.000000in}{0.000000in}}{%
\pgfpathmoveto{\pgfqpoint{0.000000in}{0.000000in}}%
\pgfpathlineto{\pgfqpoint{-0.083333in}{0.000000in}}%
\pgfusepath{stroke,fill}%
}%
\begin{pgfscope}%
\pgfsys@transformshift{0.548317in}{1.437766in}%
\pgfsys@useobject{currentmarker}{}%
\end{pgfscope}%
\end{pgfscope}%
\begin{pgfscope}%
\definecolor{textcolor}{rgb}{0.150000,0.150000,0.150000}%
\pgfsetstrokecolor{textcolor}%
\pgfsetfillcolor{textcolor}%
\pgftext[x=0.190292in,y=1.387624in,left,base]{\color{textcolor}\sffamily\fontsize{10.000000}{12.000000}\selectfont \(\displaystyle 0.0\)}%
\end{pgfscope}%
\begin{pgfscope}%
\pgfsetbuttcap%
\pgfsetroundjoin%
\definecolor{currentfill}{rgb}{0.150000,0.150000,0.150000}%
\pgfsetfillcolor{currentfill}%
\pgfsetlinewidth{1.003750pt}%
\definecolor{currentstroke}{rgb}{0.150000,0.150000,0.150000}%
\pgfsetstrokecolor{currentstroke}%
\pgfsetdash{}{0pt}%
\pgfsys@defobject{currentmarker}{\pgfqpoint{-0.083333in}{0.000000in}}{\pgfqpoint{0.000000in}{0.000000in}}{%
\pgfpathmoveto{\pgfqpoint{0.000000in}{0.000000in}}%
\pgfpathlineto{\pgfqpoint{-0.083333in}{0.000000in}}%
\pgfusepath{stroke,fill}%
}%
\begin{pgfscope}%
\pgfsys@transformshift{0.548317in}{1.963207in}%
\pgfsys@useobject{currentmarker}{}%
\end{pgfscope}%
\end{pgfscope}%
\begin{pgfscope}%
\definecolor{textcolor}{rgb}{0.150000,0.150000,0.150000}%
\pgfsetstrokecolor{textcolor}%
\pgfsetfillcolor{textcolor}%
\pgftext[x=0.190292in,y=1.913065in,left,base]{\color{textcolor}\sffamily\fontsize{10.000000}{12.000000}\selectfont \(\displaystyle 0.5\)}%
\end{pgfscope}%
\begin{pgfscope}%
\pgfsetbuttcap%
\pgfsetroundjoin%
\definecolor{currentfill}{rgb}{0.150000,0.150000,0.150000}%
\pgfsetfillcolor{currentfill}%
\pgfsetlinewidth{1.003750pt}%
\definecolor{currentstroke}{rgb}{0.150000,0.150000,0.150000}%
\pgfsetstrokecolor{currentstroke}%
\pgfsetdash{}{0pt}%
\pgfsys@defobject{currentmarker}{\pgfqpoint{-0.083333in}{0.000000in}}{\pgfqpoint{0.000000in}{0.000000in}}{%
\pgfpathmoveto{\pgfqpoint{0.000000in}{0.000000in}}%
\pgfpathlineto{\pgfqpoint{-0.083333in}{0.000000in}}%
\pgfusepath{stroke,fill}%
}%
\begin{pgfscope}%
\pgfsys@transformshift{0.548317in}{2.488647in}%
\pgfsys@useobject{currentmarker}{}%
\end{pgfscope}%
\end{pgfscope}%
\begin{pgfscope}%
\definecolor{textcolor}{rgb}{0.150000,0.150000,0.150000}%
\pgfsetstrokecolor{textcolor}%
\pgfsetfillcolor{textcolor}%
\pgftext[x=0.190292in,y=2.438505in,left,base]{\color{textcolor}\sffamily\fontsize{10.000000}{12.000000}\selectfont \(\displaystyle 1.0\)}%
\end{pgfscope}%
\begin{pgfscope}%
\pgfpathrectangle{\pgfqpoint{0.548317in}{0.386884in}}{\pgfqpoint{3.576335in}{2.101763in}} %
\pgfusepath{clip}%
\pgfsetbuttcap%
\pgfsetroundjoin%
\definecolor{currentfill}{rgb}{0.400000,0.760784,0.647059}%
\pgfsetfillcolor{currentfill}%
\pgfsetlinewidth{0.000000pt}%
\definecolor{currentstroke}{rgb}{0.400000,0.760784,0.647059}%
\pgfsetstrokecolor{currentstroke}%
\pgfsetdash{}{0pt}%
\pgfsys@defobject{currentmarker}{\pgfqpoint{-0.048611in}{-0.048611in}}{\pgfqpoint{0.048611in}{0.048611in}}{%
\pgfpathmoveto{\pgfqpoint{0.000000in}{-0.048611in}}%
\pgfpathcurveto{\pgfqpoint{0.012892in}{-0.048611in}}{\pgfqpoint{0.025257in}{-0.043489in}}{\pgfqpoint{0.034373in}{-0.034373in}}%
\pgfpathcurveto{\pgfqpoint{0.043489in}{-0.025257in}}{\pgfqpoint{0.048611in}{-0.012892in}}{\pgfqpoint{0.048611in}{0.000000in}}%
\pgfpathcurveto{\pgfqpoint{0.048611in}{0.012892in}}{\pgfqpoint{0.043489in}{0.025257in}}{\pgfqpoint{0.034373in}{0.034373in}}%
\pgfpathcurveto{\pgfqpoint{0.025257in}{0.043489in}}{\pgfqpoint{0.012892in}{0.048611in}}{\pgfqpoint{0.000000in}{0.048611in}}%
\pgfpathcurveto{\pgfqpoint{-0.012892in}{0.048611in}}{\pgfqpoint{-0.025257in}{0.043489in}}{\pgfqpoint{-0.034373in}{0.034373in}}%
\pgfpathcurveto{\pgfqpoint{-0.043489in}{0.025257in}}{\pgfqpoint{-0.048611in}{0.012892in}}{\pgfqpoint{-0.048611in}{0.000000in}}%
\pgfpathcurveto{\pgfqpoint{-0.048611in}{-0.012892in}}{\pgfqpoint{-0.043489in}{-0.025257in}}{\pgfqpoint{-0.034373in}{-0.034373in}}%
\pgfpathcurveto{\pgfqpoint{-0.025257in}{-0.043489in}}{\pgfqpoint{-0.012892in}{-0.048611in}}{\pgfqpoint{0.000000in}{-0.048611in}}%
\pgfpathclose%
\pgfusepath{fill}%
}%
\begin{pgfscope}%
\pgfsys@transformshift{2.070523in}{0.696509in}%
\pgfsys@useobject{currentmarker}{}%
\end{pgfscope}%
\begin{pgfscope}%
\pgfsys@transformshift{3.072668in}{1.190191in}%
\pgfsys@useobject{currentmarker}{}%
\end{pgfscope}%
\begin{pgfscope}%
\pgfsys@transformshift{1.924913in}{0.659782in}%
\pgfsys@useobject{currentmarker}{}%
\end{pgfscope}%
\begin{pgfscope}%
\pgfsys@transformshift{2.346466in}{0.784652in}%
\pgfsys@useobject{currentmarker}{}%
\end{pgfscope}%
\begin{pgfscope}%
\pgfsys@transformshift{1.785447in}{0.629405in}%
\pgfsys@useobject{currentmarker}{}%
\end{pgfscope}%
\begin{pgfscope}%
\pgfsys@transformshift{2.976879in}{1.118524in}%
\pgfsys@useobject{currentmarker}{}%
\end{pgfscope}%
\begin{pgfscope}%
\pgfsys@transformshift{1.535300in}{0.583092in}%
\pgfsys@useobject{currentmarker}{}%
\end{pgfscope}%
\begin{pgfscope}%
\pgfsys@transformshift{2.445080in}{0.823363in}%
\pgfsys@useobject{currentmarker}{}%
\end{pgfscope}%
\begin{pgfscope}%
\pgfsys@transformshift{1.459375in}{0.570316in}%
\pgfsys@useobject{currentmarker}{}%
\end{pgfscope}%
\begin{pgfscope}%
\pgfsys@transformshift{2.736514in}{0.965179in}%
\pgfsys@useobject{currentmarker}{}%
\end{pgfscope}%
\end{pgfscope}%
\begin{pgfscope}%
\pgfpathrectangle{\pgfqpoint{0.548317in}{0.386884in}}{\pgfqpoint{3.576335in}{2.101763in}} %
\pgfusepath{clip}%
\pgfsetbuttcap%
\pgfsetroundjoin%
\pgfsetlinewidth{1.756562pt}%
\definecolor{currentstroke}{rgb}{0.988235,0.552941,0.384314}%
\pgfsetstrokecolor{currentstroke}%
\pgfsetdash{{5.600000pt}{2.400000pt}}{0.000000pt}%
\pgfpathmoveto{\pgfqpoint{0.548317in}{0.386884in}}%
\pgfpathlineto{\pgfqpoint{0.584442in}{0.397288in}}%
\pgfpathlineto{\pgfqpoint{0.620566in}{0.407283in}}%
\pgfpathlineto{\pgfqpoint{0.656691in}{0.416887in}}%
\pgfpathlineto{\pgfqpoint{0.692815in}{0.426121in}}%
\pgfpathlineto{\pgfqpoint{0.728940in}{0.435004in}}%
\pgfpathlineto{\pgfqpoint{0.765065in}{0.443556in}}%
\pgfpathlineto{\pgfqpoint{0.801189in}{0.451796in}}%
\pgfpathlineto{\pgfqpoint{0.837314in}{0.459743in}}%
\pgfpathlineto{\pgfqpoint{0.873438in}{0.467418in}}%
\pgfpathlineto{\pgfqpoint{0.909563in}{0.474839in}}%
\pgfpathlineto{\pgfqpoint{0.945688in}{0.482026in}}%
\pgfpathlineto{\pgfqpoint{0.981812in}{0.488999in}}%
\pgfpathlineto{\pgfqpoint{1.017937in}{0.495776in}}%
\pgfpathlineto{\pgfqpoint{1.054061in}{0.502379in}}%
\pgfpathlineto{\pgfqpoint{1.090186in}{0.508825in}}%
\pgfpathlineto{\pgfqpoint{1.126311in}{0.515135in}}%
\pgfpathlineto{\pgfqpoint{1.162435in}{0.521328in}}%
\pgfpathlineto{\pgfqpoint{1.198560in}{0.527423in}}%
\pgfpathlineto{\pgfqpoint{1.234684in}{0.533440in}}%
\pgfpathlineto{\pgfqpoint{1.270809in}{0.539399in}}%
\pgfpathlineto{\pgfqpoint{1.306933in}{0.545319in}}%
\pgfpathlineto{\pgfqpoint{1.343058in}{0.551220in}}%
\pgfpathlineto{\pgfqpoint{1.379183in}{0.557120in}}%
\pgfpathlineto{\pgfqpoint{1.415307in}{0.563040in}}%
\pgfpathlineto{\pgfqpoint{1.451432in}{0.568999in}}%
\pgfpathlineto{\pgfqpoint{1.487556in}{0.575017in}}%
\pgfpathlineto{\pgfqpoint{1.523681in}{0.581112in}}%
\pgfpathlineto{\pgfqpoint{1.559806in}{0.587305in}}%
\pgfpathlineto{\pgfqpoint{1.595930in}{0.593615in}}%
\pgfpathlineto{\pgfqpoint{1.632055in}{0.600061in}}%
\pgfpathlineto{\pgfqpoint{1.668179in}{0.606663in}}%
\pgfpathlineto{\pgfqpoint{1.704304in}{0.613441in}}%
\pgfpathlineto{\pgfqpoint{1.740429in}{0.620414in}}%
\pgfpathlineto{\pgfqpoint{1.776553in}{0.627601in}}%
\pgfpathlineto{\pgfqpoint{1.812678in}{0.635022in}}%
\pgfpathlineto{\pgfqpoint{1.848802in}{0.642696in}}%
\pgfpathlineto{\pgfqpoint{1.884927in}{0.650644in}}%
\pgfpathlineto{\pgfqpoint{1.921052in}{0.658884in}}%
\pgfpathlineto{\pgfqpoint{1.957176in}{0.667435in}}%
\pgfpathlineto{\pgfqpoint{1.993301in}{0.676318in}}%
\pgfpathlineto{\pgfqpoint{2.029425in}{0.685553in}}%
\pgfpathlineto{\pgfqpoint{2.065550in}{0.695157in}}%
\pgfpathlineto{\pgfqpoint{2.101675in}{0.705151in}}%
\pgfpathlineto{\pgfqpoint{2.137799in}{0.715555in}}%
\pgfpathlineto{\pgfqpoint{2.173924in}{0.726388in}}%
\pgfpathlineto{\pgfqpoint{2.210048in}{0.737669in}}%
\pgfpathlineto{\pgfqpoint{2.246173in}{0.749418in}}%
\pgfpathlineto{\pgfqpoint{2.282298in}{0.761654in}}%
\pgfpathlineto{\pgfqpoint{2.318422in}{0.774397in}}%
\pgfpathlineto{\pgfqpoint{2.354547in}{0.787667in}}%
\pgfpathlineto{\pgfqpoint{2.390671in}{0.801482in}}%
\pgfpathlineto{\pgfqpoint{2.426796in}{0.815863in}}%
\pgfpathlineto{\pgfqpoint{2.462921in}{0.830828in}}%
\pgfpathlineto{\pgfqpoint{2.499045in}{0.846398in}}%
\pgfpathlineto{\pgfqpoint{2.535170in}{0.862592in}}%
\pgfpathlineto{\pgfqpoint{2.571294in}{0.879429in}}%
\pgfpathlineto{\pgfqpoint{2.607419in}{0.896929in}}%
\pgfpathlineto{\pgfqpoint{2.643543in}{0.915111in}}%
\pgfpathlineto{\pgfqpoint{2.679668in}{0.933995in}}%
\pgfpathlineto{\pgfqpoint{2.715793in}{0.953601in}}%
\pgfpathlineto{\pgfqpoint{2.751917in}{0.973947in}}%
\pgfpathlineto{\pgfqpoint{2.788042in}{0.995053in}}%
\pgfpathlineto{\pgfqpoint{2.824166in}{1.016940in}}%
\pgfpathlineto{\pgfqpoint{2.860291in}{1.039625in}}%
\pgfpathlineto{\pgfqpoint{2.896416in}{1.063129in}}%
\pgfpathlineto{\pgfqpoint{2.932540in}{1.087472in}}%
\pgfpathlineto{\pgfqpoint{2.968665in}{1.112672in}}%
\pgfpathlineto{\pgfqpoint{3.004789in}{1.138750in}}%
\pgfpathlineto{\pgfqpoint{3.040914in}{1.165725in}}%
\pgfpathlineto{\pgfqpoint{3.077039in}{1.193615in}}%
\pgfpathlineto{\pgfqpoint{3.113163in}{1.222442in}}%
\pgfpathlineto{\pgfqpoint{3.149288in}{1.252223in}}%
\pgfpathlineto{\pgfqpoint{3.185412in}{1.282980in}}%
\pgfpathlineto{\pgfqpoint{3.221537in}{1.314730in}}%
\pgfpathlineto{\pgfqpoint{3.257662in}{1.347495in}}%
\pgfpathlineto{\pgfqpoint{3.293786in}{1.381293in}}%
\pgfpathlineto{\pgfqpoint{3.329911in}{1.416143in}}%
\pgfpathlineto{\pgfqpoint{3.366035in}{1.452065in}}%
\pgfpathlineto{\pgfqpoint{3.402160in}{1.489080in}}%
\pgfpathlineto{\pgfqpoint{3.438285in}{1.527205in}}%
\pgfpathlineto{\pgfqpoint{3.474409in}{1.566461in}}%
\pgfpathlineto{\pgfqpoint{3.510534in}{1.606868in}}%
\pgfpathlineto{\pgfqpoint{3.546658in}{1.648444in}}%
\pgfpathlineto{\pgfqpoint{3.582783in}{1.691209in}}%
\pgfpathlineto{\pgfqpoint{3.618908in}{1.735183in}}%
\pgfpathlineto{\pgfqpoint{3.655032in}{1.780385in}}%
\pgfpathlineto{\pgfqpoint{3.691157in}{1.826835in}}%
\pgfpathlineto{\pgfqpoint{3.727281in}{1.874552in}}%
\pgfpathlineto{\pgfqpoint{3.763406in}{1.923556in}}%
\pgfpathlineto{\pgfqpoint{3.799531in}{1.973866in}}%
\pgfpathlineto{\pgfqpoint{3.835655in}{2.025501in}}%
\pgfpathlineto{\pgfqpoint{3.871780in}{2.078482in}}%
\pgfpathlineto{\pgfqpoint{3.907904in}{2.132827in}}%
\pgfpathlineto{\pgfqpoint{3.944029in}{2.188556in}}%
\pgfpathlineto{\pgfqpoint{3.980154in}{2.245689in}}%
\pgfpathlineto{\pgfqpoint{4.016278in}{2.304245in}}%
\pgfpathlineto{\pgfqpoint{4.052403in}{2.364244in}}%
\pgfpathlineto{\pgfqpoint{4.088527in}{2.425705in}}%
\pgfpathlineto{\pgfqpoint{4.124652in}{2.488647in}}%
\pgfusepath{stroke}%
\end{pgfscope}%
\begin{pgfscope}%
\pgfpathrectangle{\pgfqpoint{0.548317in}{0.386884in}}{\pgfqpoint{3.576335in}{2.101763in}} %
\pgfusepath{clip}%
\pgfsetroundcap%
\pgfsetroundjoin%
\pgfsetlinewidth{1.756562pt}%
\definecolor{currentstroke}{rgb}{0.552941,0.627451,0.796078}%
\pgfsetstrokecolor{currentstroke}%
\pgfsetdash{}{0pt}%
\pgfpathmoveto{\pgfqpoint{0.548317in}{0.386884in}}%
\pgfpathlineto{\pgfqpoint{0.584442in}{0.397288in}}%
\pgfpathlineto{\pgfqpoint{0.620566in}{0.407283in}}%
\pgfpathlineto{\pgfqpoint{0.656691in}{0.416887in}}%
\pgfpathlineto{\pgfqpoint{0.692815in}{0.426121in}}%
\pgfpathlineto{\pgfqpoint{0.728940in}{0.435004in}}%
\pgfpathlineto{\pgfqpoint{0.765065in}{0.443556in}}%
\pgfpathlineto{\pgfqpoint{0.801189in}{0.451796in}}%
\pgfpathlineto{\pgfqpoint{0.837314in}{0.459743in}}%
\pgfpathlineto{\pgfqpoint{0.873438in}{0.467418in}}%
\pgfpathlineto{\pgfqpoint{0.909563in}{0.474839in}}%
\pgfpathlineto{\pgfqpoint{0.945688in}{0.482026in}}%
\pgfpathlineto{\pgfqpoint{0.981812in}{0.488999in}}%
\pgfpathlineto{\pgfqpoint{1.017937in}{0.495776in}}%
\pgfpathlineto{\pgfqpoint{1.054061in}{0.502379in}}%
\pgfpathlineto{\pgfqpoint{1.090186in}{0.508825in}}%
\pgfpathlineto{\pgfqpoint{1.126311in}{0.515135in}}%
\pgfpathlineto{\pgfqpoint{1.162435in}{0.521328in}}%
\pgfpathlineto{\pgfqpoint{1.198560in}{0.527423in}}%
\pgfpathlineto{\pgfqpoint{1.234684in}{0.533440in}}%
\pgfpathlineto{\pgfqpoint{1.270809in}{0.539399in}}%
\pgfpathlineto{\pgfqpoint{1.306933in}{0.545319in}}%
\pgfpathlineto{\pgfqpoint{1.343058in}{0.551220in}}%
\pgfpathlineto{\pgfqpoint{1.379183in}{0.557120in}}%
\pgfpathlineto{\pgfqpoint{1.415307in}{0.563040in}}%
\pgfpathlineto{\pgfqpoint{1.451432in}{0.568999in}}%
\pgfpathlineto{\pgfqpoint{1.487556in}{0.575017in}}%
\pgfpathlineto{\pgfqpoint{1.523681in}{0.581112in}}%
\pgfpathlineto{\pgfqpoint{1.559806in}{0.587305in}}%
\pgfpathlineto{\pgfqpoint{1.595930in}{0.593615in}}%
\pgfpathlineto{\pgfqpoint{1.632055in}{0.600061in}}%
\pgfpathlineto{\pgfqpoint{1.668179in}{0.606663in}}%
\pgfpathlineto{\pgfqpoint{1.704304in}{0.613441in}}%
\pgfpathlineto{\pgfqpoint{1.740429in}{0.620414in}}%
\pgfpathlineto{\pgfqpoint{1.776553in}{0.627601in}}%
\pgfpathlineto{\pgfqpoint{1.812678in}{0.635022in}}%
\pgfpathlineto{\pgfqpoint{1.848802in}{0.642696in}}%
\pgfpathlineto{\pgfqpoint{1.884927in}{0.650644in}}%
\pgfpathlineto{\pgfqpoint{1.921052in}{0.658884in}}%
\pgfpathlineto{\pgfqpoint{1.957176in}{0.667435in}}%
\pgfpathlineto{\pgfqpoint{1.993301in}{0.676318in}}%
\pgfpathlineto{\pgfqpoint{2.029425in}{0.685553in}}%
\pgfpathlineto{\pgfqpoint{2.065550in}{0.695157in}}%
\pgfpathlineto{\pgfqpoint{2.101675in}{0.705151in}}%
\pgfpathlineto{\pgfqpoint{2.137799in}{0.715555in}}%
\pgfpathlineto{\pgfqpoint{2.173924in}{0.726388in}}%
\pgfpathlineto{\pgfqpoint{2.210048in}{0.737669in}}%
\pgfpathlineto{\pgfqpoint{2.246173in}{0.749418in}}%
\pgfpathlineto{\pgfqpoint{2.282298in}{0.761654in}}%
\pgfpathlineto{\pgfqpoint{2.318422in}{0.774397in}}%
\pgfpathlineto{\pgfqpoint{2.354547in}{0.787667in}}%
\pgfpathlineto{\pgfqpoint{2.390671in}{0.801482in}}%
\pgfpathlineto{\pgfqpoint{2.426796in}{0.815863in}}%
\pgfpathlineto{\pgfqpoint{2.462921in}{0.830828in}}%
\pgfpathlineto{\pgfqpoint{2.499045in}{0.846398in}}%
\pgfpathlineto{\pgfqpoint{2.535170in}{0.862592in}}%
\pgfpathlineto{\pgfqpoint{2.571294in}{0.879429in}}%
\pgfpathlineto{\pgfqpoint{2.607419in}{0.896929in}}%
\pgfpathlineto{\pgfqpoint{2.643543in}{0.915111in}}%
\pgfpathlineto{\pgfqpoint{2.679668in}{0.933995in}}%
\pgfpathlineto{\pgfqpoint{2.715793in}{0.953601in}}%
\pgfpathlineto{\pgfqpoint{2.751917in}{0.973947in}}%
\pgfpathlineto{\pgfqpoint{2.788042in}{0.995053in}}%
\pgfpathlineto{\pgfqpoint{2.824166in}{1.016940in}}%
\pgfpathlineto{\pgfqpoint{2.860291in}{1.039625in}}%
\pgfpathlineto{\pgfqpoint{2.896416in}{1.063129in}}%
\pgfpathlineto{\pgfqpoint{2.932540in}{1.087472in}}%
\pgfpathlineto{\pgfqpoint{2.968665in}{1.112672in}}%
\pgfpathlineto{\pgfqpoint{3.004789in}{1.138750in}}%
\pgfpathlineto{\pgfqpoint{3.040914in}{1.165725in}}%
\pgfpathlineto{\pgfqpoint{3.077039in}{1.193615in}}%
\pgfpathlineto{\pgfqpoint{3.113163in}{1.222442in}}%
\pgfpathlineto{\pgfqpoint{3.149288in}{1.252223in}}%
\pgfpathlineto{\pgfqpoint{3.185412in}{1.282980in}}%
\pgfpathlineto{\pgfqpoint{3.221537in}{1.314730in}}%
\pgfpathlineto{\pgfqpoint{3.257662in}{1.347495in}}%
\pgfpathlineto{\pgfqpoint{3.293786in}{1.381293in}}%
\pgfpathlineto{\pgfqpoint{3.329911in}{1.416143in}}%
\pgfpathlineto{\pgfqpoint{3.366035in}{1.452065in}}%
\pgfpathlineto{\pgfqpoint{3.402160in}{1.489080in}}%
\pgfpathlineto{\pgfqpoint{3.438285in}{1.527205in}}%
\pgfpathlineto{\pgfqpoint{3.474409in}{1.566461in}}%
\pgfpathlineto{\pgfqpoint{3.510534in}{1.606868in}}%
\pgfpathlineto{\pgfqpoint{3.546658in}{1.648444in}}%
\pgfpathlineto{\pgfqpoint{3.582783in}{1.691209in}}%
\pgfpathlineto{\pgfqpoint{3.618908in}{1.735183in}}%
\pgfpathlineto{\pgfqpoint{3.655032in}{1.780385in}}%
\pgfpathlineto{\pgfqpoint{3.691157in}{1.826835in}}%
\pgfpathlineto{\pgfqpoint{3.727281in}{1.874552in}}%
\pgfpathlineto{\pgfqpoint{3.763406in}{1.923556in}}%
\pgfpathlineto{\pgfqpoint{3.799531in}{1.973866in}}%
\pgfpathlineto{\pgfqpoint{3.835655in}{2.025501in}}%
\pgfpathlineto{\pgfqpoint{3.871780in}{2.078482in}}%
\pgfpathlineto{\pgfqpoint{3.907904in}{2.132827in}}%
\pgfpathlineto{\pgfqpoint{3.944029in}{2.188556in}}%
\pgfpathlineto{\pgfqpoint{3.980154in}{2.245689in}}%
\pgfpathlineto{\pgfqpoint{4.016278in}{2.304245in}}%
\pgfpathlineto{\pgfqpoint{4.052403in}{2.364244in}}%
\pgfpathlineto{\pgfqpoint{4.088527in}{2.425705in}}%
\pgfpathlineto{\pgfqpoint{4.124652in}{2.488647in}}%
\pgfusepath{stroke}%
\end{pgfscope}%
\begin{pgfscope}%
\pgfsetrectcap%
\pgfsetmiterjoin%
\pgfsetlinewidth{1.254687pt}%
\definecolor{currentstroke}{rgb}{0.150000,0.150000,0.150000}%
\pgfsetstrokecolor{currentstroke}%
\pgfsetdash{}{0pt}%
\pgfpathmoveto{\pgfqpoint{0.548317in}{0.386884in}}%
\pgfpathlineto{\pgfqpoint{0.548317in}{2.488647in}}%
\pgfusepath{stroke}%
\end{pgfscope}%
\begin{pgfscope}%
\pgfsetrectcap%
\pgfsetmiterjoin%
\pgfsetlinewidth{1.254687pt}%
\definecolor{currentstroke}{rgb}{0.150000,0.150000,0.150000}%
\pgfsetstrokecolor{currentstroke}%
\pgfsetdash{}{0pt}%
\pgfpathmoveto{\pgfqpoint{0.548317in}{0.386884in}}%
\pgfpathlineto{\pgfqpoint{4.124652in}{0.386884in}}%
\pgfusepath{stroke}%
\end{pgfscope}%
\begin{pgfscope}%
\pgfsetbuttcap%
\pgfsetroundjoin%
\definecolor{currentfill}{rgb}{0.400000,0.760784,0.647059}%
\pgfsetfillcolor{currentfill}%
\pgfsetlinewidth{0.000000pt}%
\definecolor{currentstroke}{rgb}{0.400000,0.760784,0.647059}%
\pgfsetstrokecolor{currentstroke}%
\pgfsetdash{}{0pt}%
\pgfsys@defobject{currentmarker}{\pgfqpoint{-0.048611in}{-0.048611in}}{\pgfqpoint{0.048611in}{0.048611in}}{%
\pgfpathmoveto{\pgfqpoint{0.000000in}{-0.048611in}}%
\pgfpathcurveto{\pgfqpoint{0.012892in}{-0.048611in}}{\pgfqpoint{0.025257in}{-0.043489in}}{\pgfqpoint{0.034373in}{-0.034373in}}%
\pgfpathcurveto{\pgfqpoint{0.043489in}{-0.025257in}}{\pgfqpoint{0.048611in}{-0.012892in}}{\pgfqpoint{0.048611in}{0.000000in}}%
\pgfpathcurveto{\pgfqpoint{0.048611in}{0.012892in}}{\pgfqpoint{0.043489in}{0.025257in}}{\pgfqpoint{0.034373in}{0.034373in}}%
\pgfpathcurveto{\pgfqpoint{0.025257in}{0.043489in}}{\pgfqpoint{0.012892in}{0.048611in}}{\pgfqpoint{0.000000in}{0.048611in}}%
\pgfpathcurveto{\pgfqpoint{-0.012892in}{0.048611in}}{\pgfqpoint{-0.025257in}{0.043489in}}{\pgfqpoint{-0.034373in}{0.034373in}}%
\pgfpathcurveto{\pgfqpoint{-0.043489in}{0.025257in}}{\pgfqpoint{-0.048611in}{0.012892in}}{\pgfqpoint{-0.048611in}{0.000000in}}%
\pgfpathcurveto{\pgfqpoint{-0.048611in}{-0.012892in}}{\pgfqpoint{-0.043489in}{-0.025257in}}{\pgfqpoint{-0.034373in}{-0.034373in}}%
\pgfpathcurveto{\pgfqpoint{-0.025257in}{-0.043489in}}{\pgfqpoint{-0.012892in}{-0.048611in}}{\pgfqpoint{0.000000in}{-0.048611in}}%
\pgfpathclose%
\pgfusepath{fill}%
}%
\begin{pgfscope}%
\pgfsys@transformshift{0.812206in}{2.311974in}%
\pgfsys@useobject{currentmarker}{}%
\end{pgfscope}%
\end{pgfscope}%
\begin{pgfscope}%
\definecolor{textcolor}{rgb}{0.150000,0.150000,0.150000}%
\pgfsetstrokecolor{textcolor}%
\pgfsetfillcolor{textcolor}%
\pgftext[x=1.062206in,y=2.263363in,left,base]{\color{textcolor}\sffamily\fontsize{10.000000}{12.000000}\selectfont samples}%
\end{pgfscope}%
\begin{pgfscope}%
\pgfsetbuttcap%
\pgfsetroundjoin%
\pgfsetlinewidth{1.756562pt}%
\definecolor{currentstroke}{rgb}{0.988235,0.552941,0.384314}%
\pgfsetstrokecolor{currentstroke}%
\pgfsetdash{{5.600000pt}{2.400000pt}}{0.000000pt}%
\pgfpathmoveto{\pgfqpoint{0.673317in}{2.115246in}}%
\pgfpathlineto{\pgfqpoint{0.951095in}{2.115246in}}%
\pgfusepath{stroke}%
\end{pgfscope}%
\begin{pgfscope}%
\definecolor{textcolor}{rgb}{0.150000,0.150000,0.150000}%
\pgfsetstrokecolor{textcolor}%
\pgfsetfillcolor{textcolor}%
\pgftext[x=1.062206in,y=2.066635in,left,base]{\color{textcolor}\sffamily\fontsize{10.000000}{12.000000}\selectfont function}%
\end{pgfscope}%
\begin{pgfscope}%
\pgfsetroundcap%
\pgfsetroundjoin%
\pgfsetlinewidth{1.756562pt}%
\definecolor{currentstroke}{rgb}{0.552941,0.627451,0.796078}%
\pgfsetstrokecolor{currentstroke}%
\pgfsetdash{}{0pt}%
\pgfpathmoveto{\pgfqpoint{0.673317in}{1.918518in}}%
\pgfpathlineto{\pgfqpoint{0.951095in}{1.918518in}}%
\pgfusepath{stroke}%
\end{pgfscope}%
\begin{pgfscope}%
\definecolor{textcolor}{rgb}{0.150000,0.150000,0.150000}%
\pgfsetstrokecolor{textcolor}%
\pgfsetfillcolor{textcolor}%
\pgftext[x=1.062206in,y=1.869907in,left,base]{\color{textcolor}\sffamily\fontsize{10.000000}{12.000000}\selectfont 3rd order fit}%
\end{pgfscope}%
\end{pgfpicture}%
\makeatother%
\endgroup%
}
			\caption{\engordnumber{3}-order polynomial fitting 10 points}
			\label{fig:polyfit3rdlots}
		\end{subfigure}
		~
		\begin{subfigure}[t]{0.49\textwidth}
			\resizebox{\linewidth}{!}{%% Creator: Matplotlib, PGF backend
%%
%% To include the figure in your LaTeX document, write
%%   \input{<filename>.pgf}
%%
%% Make sure the required packages are loaded in your preamble
%%   \usepackage{pgf}
%%
%% Figures using additional raster images can only be included by \input if
%% they are in the same directory as the main LaTeX file. For loading figures
%% from other directories you can use the `import` package
%%   \usepackage{import}
%% and then include the figures with
%%   \import{<path to file>}{<filename>.pgf}
%%
%% Matplotlib used the following preamble
%%   \usepackage[utf8x]{inputenc}
%%   \usepackage[T1]{fontenc}
%%
\begingroup%
\makeatletter%
\begin{pgfpicture}%
\pgfpathrectangle{\pgfpointorigin}{\pgfqpoint{4.296389in}{2.655314in}}%
\pgfusepath{use as bounding box, clip}%
\begin{pgfscope}%
\pgfsetbuttcap%
\pgfsetmiterjoin%
\definecolor{currentfill}{rgb}{1.000000,1.000000,1.000000}%
\pgfsetfillcolor{currentfill}%
\pgfsetlinewidth{0.000000pt}%
\definecolor{currentstroke}{rgb}{1.000000,1.000000,1.000000}%
\pgfsetstrokecolor{currentstroke}%
\pgfsetdash{}{0pt}%
\pgfpathmoveto{\pgfqpoint{0.000000in}{0.000000in}}%
\pgfpathlineto{\pgfqpoint{4.296389in}{0.000000in}}%
\pgfpathlineto{\pgfqpoint{4.296389in}{2.655314in}}%
\pgfpathlineto{\pgfqpoint{0.000000in}{2.655314in}}%
\pgfpathclose%
\pgfusepath{fill}%
\end{pgfscope}%
\begin{pgfscope}%
\pgfsetbuttcap%
\pgfsetmiterjoin%
\definecolor{currentfill}{rgb}{1.000000,1.000000,1.000000}%
\pgfsetfillcolor{currentfill}%
\pgfsetlinewidth{0.000000pt}%
\definecolor{currentstroke}{rgb}{0.000000,0.000000,0.000000}%
\pgfsetstrokecolor{currentstroke}%
\pgfsetstrokeopacity{0.000000}%
\pgfsetdash{}{0pt}%
\pgfpathmoveto{\pgfqpoint{0.548317in}{0.386884in}}%
\pgfpathlineto{\pgfqpoint{4.124652in}{0.386884in}}%
\pgfpathlineto{\pgfqpoint{4.124652in}{2.488647in}}%
\pgfpathlineto{\pgfqpoint{0.548317in}{2.488647in}}%
\pgfpathclose%
\pgfusepath{fill}%
\end{pgfscope}%
\begin{pgfscope}%
\pgfsetbuttcap%
\pgfsetroundjoin%
\definecolor{currentfill}{rgb}{0.150000,0.150000,0.150000}%
\pgfsetfillcolor{currentfill}%
\pgfsetlinewidth{1.003750pt}%
\definecolor{currentstroke}{rgb}{0.150000,0.150000,0.150000}%
\pgfsetstrokecolor{currentstroke}%
\pgfsetdash{}{0pt}%
\pgfsys@defobject{currentmarker}{\pgfqpoint{0.000000in}{-0.083333in}}{\pgfqpoint{0.000000in}{0.000000in}}{%
\pgfpathmoveto{\pgfqpoint{0.000000in}{0.000000in}}%
\pgfpathlineto{\pgfqpoint{0.000000in}{-0.083333in}}%
\pgfusepath{stroke,fill}%
}%
\begin{pgfscope}%
\pgfsys@transformshift{0.548317in}{0.386884in}%
\pgfsys@useobject{currentmarker}{}%
\end{pgfscope}%
\end{pgfscope}%
\begin{pgfscope}%
\definecolor{textcolor}{rgb}{0.150000,0.150000,0.150000}%
\pgfsetstrokecolor{textcolor}%
\pgfsetfillcolor{textcolor}%
\pgftext[x=0.548317in,y=0.206329in,,top]{\color{textcolor}\sffamily\fontsize{10.000000}{12.000000}\selectfont \(\displaystyle 0.0\)}%
\end{pgfscope}%
\begin{pgfscope}%
\pgfsetbuttcap%
\pgfsetroundjoin%
\definecolor{currentfill}{rgb}{0.150000,0.150000,0.150000}%
\pgfsetfillcolor{currentfill}%
\pgfsetlinewidth{1.003750pt}%
\definecolor{currentstroke}{rgb}{0.150000,0.150000,0.150000}%
\pgfsetstrokecolor{currentstroke}%
\pgfsetdash{}{0pt}%
\pgfsys@defobject{currentmarker}{\pgfqpoint{0.000000in}{-0.083333in}}{\pgfqpoint{0.000000in}{0.000000in}}{%
\pgfpathmoveto{\pgfqpoint{0.000000in}{0.000000in}}%
\pgfpathlineto{\pgfqpoint{0.000000in}{-0.083333in}}%
\pgfusepath{stroke,fill}%
}%
\begin{pgfscope}%
\pgfsys@transformshift{1.263584in}{0.386884in}%
\pgfsys@useobject{currentmarker}{}%
\end{pgfscope}%
\end{pgfscope}%
\begin{pgfscope}%
\definecolor{textcolor}{rgb}{0.150000,0.150000,0.150000}%
\pgfsetstrokecolor{textcolor}%
\pgfsetfillcolor{textcolor}%
\pgftext[x=1.263584in,y=0.206329in,,top]{\color{textcolor}\sffamily\fontsize{10.000000}{12.000000}\selectfont \(\displaystyle 0.2\)}%
\end{pgfscope}%
\begin{pgfscope}%
\pgfsetbuttcap%
\pgfsetroundjoin%
\definecolor{currentfill}{rgb}{0.150000,0.150000,0.150000}%
\pgfsetfillcolor{currentfill}%
\pgfsetlinewidth{1.003750pt}%
\definecolor{currentstroke}{rgb}{0.150000,0.150000,0.150000}%
\pgfsetstrokecolor{currentstroke}%
\pgfsetdash{}{0pt}%
\pgfsys@defobject{currentmarker}{\pgfqpoint{0.000000in}{-0.083333in}}{\pgfqpoint{0.000000in}{0.000000in}}{%
\pgfpathmoveto{\pgfqpoint{0.000000in}{0.000000in}}%
\pgfpathlineto{\pgfqpoint{0.000000in}{-0.083333in}}%
\pgfusepath{stroke,fill}%
}%
\begin{pgfscope}%
\pgfsys@transformshift{1.978851in}{0.386884in}%
\pgfsys@useobject{currentmarker}{}%
\end{pgfscope}%
\end{pgfscope}%
\begin{pgfscope}%
\definecolor{textcolor}{rgb}{0.150000,0.150000,0.150000}%
\pgfsetstrokecolor{textcolor}%
\pgfsetfillcolor{textcolor}%
\pgftext[x=1.978851in,y=0.206329in,,top]{\color{textcolor}\sffamily\fontsize{10.000000}{12.000000}\selectfont \(\displaystyle 0.4\)}%
\end{pgfscope}%
\begin{pgfscope}%
\pgfsetbuttcap%
\pgfsetroundjoin%
\definecolor{currentfill}{rgb}{0.150000,0.150000,0.150000}%
\pgfsetfillcolor{currentfill}%
\pgfsetlinewidth{1.003750pt}%
\definecolor{currentstroke}{rgb}{0.150000,0.150000,0.150000}%
\pgfsetstrokecolor{currentstroke}%
\pgfsetdash{}{0pt}%
\pgfsys@defobject{currentmarker}{\pgfqpoint{0.000000in}{-0.083333in}}{\pgfqpoint{0.000000in}{0.000000in}}{%
\pgfpathmoveto{\pgfqpoint{0.000000in}{0.000000in}}%
\pgfpathlineto{\pgfqpoint{0.000000in}{-0.083333in}}%
\pgfusepath{stroke,fill}%
}%
\begin{pgfscope}%
\pgfsys@transformshift{2.694118in}{0.386884in}%
\pgfsys@useobject{currentmarker}{}%
\end{pgfscope}%
\end{pgfscope}%
\begin{pgfscope}%
\definecolor{textcolor}{rgb}{0.150000,0.150000,0.150000}%
\pgfsetstrokecolor{textcolor}%
\pgfsetfillcolor{textcolor}%
\pgftext[x=2.694118in,y=0.206329in,,top]{\color{textcolor}\sffamily\fontsize{10.000000}{12.000000}\selectfont \(\displaystyle 0.6\)}%
\end{pgfscope}%
\begin{pgfscope}%
\pgfsetbuttcap%
\pgfsetroundjoin%
\definecolor{currentfill}{rgb}{0.150000,0.150000,0.150000}%
\pgfsetfillcolor{currentfill}%
\pgfsetlinewidth{1.003750pt}%
\definecolor{currentstroke}{rgb}{0.150000,0.150000,0.150000}%
\pgfsetstrokecolor{currentstroke}%
\pgfsetdash{}{0pt}%
\pgfsys@defobject{currentmarker}{\pgfqpoint{0.000000in}{-0.083333in}}{\pgfqpoint{0.000000in}{0.000000in}}{%
\pgfpathmoveto{\pgfqpoint{0.000000in}{0.000000in}}%
\pgfpathlineto{\pgfqpoint{0.000000in}{-0.083333in}}%
\pgfusepath{stroke,fill}%
}%
\begin{pgfscope}%
\pgfsys@transformshift{3.409385in}{0.386884in}%
\pgfsys@useobject{currentmarker}{}%
\end{pgfscope}%
\end{pgfscope}%
\begin{pgfscope}%
\definecolor{textcolor}{rgb}{0.150000,0.150000,0.150000}%
\pgfsetstrokecolor{textcolor}%
\pgfsetfillcolor{textcolor}%
\pgftext[x=3.409385in,y=0.206329in,,top]{\color{textcolor}\sffamily\fontsize{10.000000}{12.000000}\selectfont \(\displaystyle 0.8\)}%
\end{pgfscope}%
\begin{pgfscope}%
\pgfsetbuttcap%
\pgfsetroundjoin%
\definecolor{currentfill}{rgb}{0.150000,0.150000,0.150000}%
\pgfsetfillcolor{currentfill}%
\pgfsetlinewidth{1.003750pt}%
\definecolor{currentstroke}{rgb}{0.150000,0.150000,0.150000}%
\pgfsetstrokecolor{currentstroke}%
\pgfsetdash{}{0pt}%
\pgfsys@defobject{currentmarker}{\pgfqpoint{0.000000in}{-0.083333in}}{\pgfqpoint{0.000000in}{0.000000in}}{%
\pgfpathmoveto{\pgfqpoint{0.000000in}{0.000000in}}%
\pgfpathlineto{\pgfqpoint{0.000000in}{-0.083333in}}%
\pgfusepath{stroke,fill}%
}%
\begin{pgfscope}%
\pgfsys@transformshift{4.124652in}{0.386884in}%
\pgfsys@useobject{currentmarker}{}%
\end{pgfscope}%
\end{pgfscope}%
\begin{pgfscope}%
\definecolor{textcolor}{rgb}{0.150000,0.150000,0.150000}%
\pgfsetstrokecolor{textcolor}%
\pgfsetfillcolor{textcolor}%
\pgftext[x=4.124652in,y=0.206329in,,top]{\color{textcolor}\sffamily\fontsize{10.000000}{12.000000}\selectfont \(\displaystyle 1.0\)}%
\end{pgfscope}%
\begin{pgfscope}%
\pgfsetbuttcap%
\pgfsetroundjoin%
\definecolor{currentfill}{rgb}{0.150000,0.150000,0.150000}%
\pgfsetfillcolor{currentfill}%
\pgfsetlinewidth{1.003750pt}%
\definecolor{currentstroke}{rgb}{0.150000,0.150000,0.150000}%
\pgfsetstrokecolor{currentstroke}%
\pgfsetdash{}{0pt}%
\pgfsys@defobject{currentmarker}{\pgfqpoint{-0.083333in}{0.000000in}}{\pgfqpoint{0.000000in}{0.000000in}}{%
\pgfpathmoveto{\pgfqpoint{0.000000in}{0.000000in}}%
\pgfpathlineto{\pgfqpoint{-0.083333in}{0.000000in}}%
\pgfusepath{stroke,fill}%
}%
\begin{pgfscope}%
\pgfsys@transformshift{0.548317in}{0.386884in}%
\pgfsys@useobject{currentmarker}{}%
\end{pgfscope}%
\end{pgfscope}%
\begin{pgfscope}%
\definecolor{textcolor}{rgb}{0.150000,0.150000,0.150000}%
\pgfsetstrokecolor{textcolor}%
\pgfsetfillcolor{textcolor}%
\pgftext[x=0.082267in,y=0.336742in,left,base]{\color{textcolor}\sffamily\fontsize{10.000000}{12.000000}\selectfont \(\displaystyle -1.0\)}%
\end{pgfscope}%
\begin{pgfscope}%
\pgfsetbuttcap%
\pgfsetroundjoin%
\definecolor{currentfill}{rgb}{0.150000,0.150000,0.150000}%
\pgfsetfillcolor{currentfill}%
\pgfsetlinewidth{1.003750pt}%
\definecolor{currentstroke}{rgb}{0.150000,0.150000,0.150000}%
\pgfsetstrokecolor{currentstroke}%
\pgfsetdash{}{0pt}%
\pgfsys@defobject{currentmarker}{\pgfqpoint{-0.083333in}{0.000000in}}{\pgfqpoint{0.000000in}{0.000000in}}{%
\pgfpathmoveto{\pgfqpoint{0.000000in}{0.000000in}}%
\pgfpathlineto{\pgfqpoint{-0.083333in}{0.000000in}}%
\pgfusepath{stroke,fill}%
}%
\begin{pgfscope}%
\pgfsys@transformshift{0.548317in}{0.912325in}%
\pgfsys@useobject{currentmarker}{}%
\end{pgfscope}%
\end{pgfscope}%
\begin{pgfscope}%
\definecolor{textcolor}{rgb}{0.150000,0.150000,0.150000}%
\pgfsetstrokecolor{textcolor}%
\pgfsetfillcolor{textcolor}%
\pgftext[x=0.082267in,y=0.862183in,left,base]{\color{textcolor}\sffamily\fontsize{10.000000}{12.000000}\selectfont \(\displaystyle -0.5\)}%
\end{pgfscope}%
\begin{pgfscope}%
\pgfsetbuttcap%
\pgfsetroundjoin%
\definecolor{currentfill}{rgb}{0.150000,0.150000,0.150000}%
\pgfsetfillcolor{currentfill}%
\pgfsetlinewidth{1.003750pt}%
\definecolor{currentstroke}{rgb}{0.150000,0.150000,0.150000}%
\pgfsetstrokecolor{currentstroke}%
\pgfsetdash{}{0pt}%
\pgfsys@defobject{currentmarker}{\pgfqpoint{-0.083333in}{0.000000in}}{\pgfqpoint{0.000000in}{0.000000in}}{%
\pgfpathmoveto{\pgfqpoint{0.000000in}{0.000000in}}%
\pgfpathlineto{\pgfqpoint{-0.083333in}{0.000000in}}%
\pgfusepath{stroke,fill}%
}%
\begin{pgfscope}%
\pgfsys@transformshift{0.548317in}{1.437766in}%
\pgfsys@useobject{currentmarker}{}%
\end{pgfscope}%
\end{pgfscope}%
\begin{pgfscope}%
\definecolor{textcolor}{rgb}{0.150000,0.150000,0.150000}%
\pgfsetstrokecolor{textcolor}%
\pgfsetfillcolor{textcolor}%
\pgftext[x=0.190292in,y=1.387624in,left,base]{\color{textcolor}\sffamily\fontsize{10.000000}{12.000000}\selectfont \(\displaystyle 0.0\)}%
\end{pgfscope}%
\begin{pgfscope}%
\pgfsetbuttcap%
\pgfsetroundjoin%
\definecolor{currentfill}{rgb}{0.150000,0.150000,0.150000}%
\pgfsetfillcolor{currentfill}%
\pgfsetlinewidth{1.003750pt}%
\definecolor{currentstroke}{rgb}{0.150000,0.150000,0.150000}%
\pgfsetstrokecolor{currentstroke}%
\pgfsetdash{}{0pt}%
\pgfsys@defobject{currentmarker}{\pgfqpoint{-0.083333in}{0.000000in}}{\pgfqpoint{0.000000in}{0.000000in}}{%
\pgfpathmoveto{\pgfqpoint{0.000000in}{0.000000in}}%
\pgfpathlineto{\pgfqpoint{-0.083333in}{0.000000in}}%
\pgfusepath{stroke,fill}%
}%
\begin{pgfscope}%
\pgfsys@transformshift{0.548317in}{1.963207in}%
\pgfsys@useobject{currentmarker}{}%
\end{pgfscope}%
\end{pgfscope}%
\begin{pgfscope}%
\definecolor{textcolor}{rgb}{0.150000,0.150000,0.150000}%
\pgfsetstrokecolor{textcolor}%
\pgfsetfillcolor{textcolor}%
\pgftext[x=0.190292in,y=1.913065in,left,base]{\color{textcolor}\sffamily\fontsize{10.000000}{12.000000}\selectfont \(\displaystyle 0.5\)}%
\end{pgfscope}%
\begin{pgfscope}%
\pgfsetbuttcap%
\pgfsetroundjoin%
\definecolor{currentfill}{rgb}{0.150000,0.150000,0.150000}%
\pgfsetfillcolor{currentfill}%
\pgfsetlinewidth{1.003750pt}%
\definecolor{currentstroke}{rgb}{0.150000,0.150000,0.150000}%
\pgfsetstrokecolor{currentstroke}%
\pgfsetdash{}{0pt}%
\pgfsys@defobject{currentmarker}{\pgfqpoint{-0.083333in}{0.000000in}}{\pgfqpoint{0.000000in}{0.000000in}}{%
\pgfpathmoveto{\pgfqpoint{0.000000in}{0.000000in}}%
\pgfpathlineto{\pgfqpoint{-0.083333in}{0.000000in}}%
\pgfusepath{stroke,fill}%
}%
\begin{pgfscope}%
\pgfsys@transformshift{0.548317in}{2.488647in}%
\pgfsys@useobject{currentmarker}{}%
\end{pgfscope}%
\end{pgfscope}%
\begin{pgfscope}%
\definecolor{textcolor}{rgb}{0.150000,0.150000,0.150000}%
\pgfsetstrokecolor{textcolor}%
\pgfsetfillcolor{textcolor}%
\pgftext[x=0.190292in,y=2.438505in,left,base]{\color{textcolor}\sffamily\fontsize{10.000000}{12.000000}\selectfont \(\displaystyle 1.0\)}%
\end{pgfscope}%
\begin{pgfscope}%
\pgfpathrectangle{\pgfqpoint{0.548317in}{0.386884in}}{\pgfqpoint{3.576335in}{2.101763in}} %
\pgfusepath{clip}%
\pgfsetbuttcap%
\pgfsetroundjoin%
\definecolor{currentfill}{rgb}{0.400000,0.760784,0.647059}%
\pgfsetfillcolor{currentfill}%
\pgfsetlinewidth{0.000000pt}%
\definecolor{currentstroke}{rgb}{0.400000,0.760784,0.647059}%
\pgfsetstrokecolor{currentstroke}%
\pgfsetdash{}{0pt}%
\pgfsys@defobject{currentmarker}{\pgfqpoint{-0.048611in}{-0.048611in}}{\pgfqpoint{0.048611in}{0.048611in}}{%
\pgfpathmoveto{\pgfqpoint{0.000000in}{-0.048611in}}%
\pgfpathcurveto{\pgfqpoint{0.012892in}{-0.048611in}}{\pgfqpoint{0.025257in}{-0.043489in}}{\pgfqpoint{0.034373in}{-0.034373in}}%
\pgfpathcurveto{\pgfqpoint{0.043489in}{-0.025257in}}{\pgfqpoint{0.048611in}{-0.012892in}}{\pgfqpoint{0.048611in}{0.000000in}}%
\pgfpathcurveto{\pgfqpoint{0.048611in}{0.012892in}}{\pgfqpoint{0.043489in}{0.025257in}}{\pgfqpoint{0.034373in}{0.034373in}}%
\pgfpathcurveto{\pgfqpoint{0.025257in}{0.043489in}}{\pgfqpoint{0.012892in}{0.048611in}}{\pgfqpoint{0.000000in}{0.048611in}}%
\pgfpathcurveto{\pgfqpoint{-0.012892in}{0.048611in}}{\pgfqpoint{-0.025257in}{0.043489in}}{\pgfqpoint{-0.034373in}{0.034373in}}%
\pgfpathcurveto{\pgfqpoint{-0.043489in}{0.025257in}}{\pgfqpoint{-0.048611in}{0.012892in}}{\pgfqpoint{-0.048611in}{0.000000in}}%
\pgfpathcurveto{\pgfqpoint{-0.048611in}{-0.012892in}}{\pgfqpoint{-0.043489in}{-0.025257in}}{\pgfqpoint{-0.034373in}{-0.034373in}}%
\pgfpathcurveto{\pgfqpoint{-0.025257in}{-0.043489in}}{\pgfqpoint{-0.012892in}{-0.048611in}}{\pgfqpoint{0.000000in}{-0.048611in}}%
\pgfpathclose%
\pgfusepath{fill}%
}%
\begin{pgfscope}%
\pgfsys@transformshift{2.070523in}{0.696509in}%
\pgfsys@useobject{currentmarker}{}%
\end{pgfscope}%
\begin{pgfscope}%
\pgfsys@transformshift{3.072668in}{1.190191in}%
\pgfsys@useobject{currentmarker}{}%
\end{pgfscope}%
\begin{pgfscope}%
\pgfsys@transformshift{1.924913in}{0.659782in}%
\pgfsys@useobject{currentmarker}{}%
\end{pgfscope}%
\begin{pgfscope}%
\pgfsys@transformshift{2.346466in}{0.784652in}%
\pgfsys@useobject{currentmarker}{}%
\end{pgfscope}%
\begin{pgfscope}%
\pgfsys@transformshift{1.785447in}{0.629405in}%
\pgfsys@useobject{currentmarker}{}%
\end{pgfscope}%
\begin{pgfscope}%
\pgfsys@transformshift{2.976879in}{1.118524in}%
\pgfsys@useobject{currentmarker}{}%
\end{pgfscope}%
\begin{pgfscope}%
\pgfsys@transformshift{1.535300in}{0.583092in}%
\pgfsys@useobject{currentmarker}{}%
\end{pgfscope}%
\begin{pgfscope}%
\pgfsys@transformshift{2.445080in}{0.823363in}%
\pgfsys@useobject{currentmarker}{}%
\end{pgfscope}%
\begin{pgfscope}%
\pgfsys@transformshift{1.459375in}{0.570316in}%
\pgfsys@useobject{currentmarker}{}%
\end{pgfscope}%
\begin{pgfscope}%
\pgfsys@transformshift{2.736514in}{0.965179in}%
\pgfsys@useobject{currentmarker}{}%
\end{pgfscope}%
\end{pgfscope}%
\begin{pgfscope}%
\pgfpathrectangle{\pgfqpoint{0.548317in}{0.386884in}}{\pgfqpoint{3.576335in}{2.101763in}} %
\pgfusepath{clip}%
\pgfsetbuttcap%
\pgfsetroundjoin%
\pgfsetlinewidth{1.756562pt}%
\definecolor{currentstroke}{rgb}{0.988235,0.552941,0.384314}%
\pgfsetstrokecolor{currentstroke}%
\pgfsetdash{{5.600000pt}{2.400000pt}}{0.000000pt}%
\pgfpathmoveto{\pgfqpoint{0.548317in}{0.386884in}}%
\pgfpathlineto{\pgfqpoint{0.584442in}{0.397288in}}%
\pgfpathlineto{\pgfqpoint{0.620566in}{0.407283in}}%
\pgfpathlineto{\pgfqpoint{0.656691in}{0.416887in}}%
\pgfpathlineto{\pgfqpoint{0.692815in}{0.426121in}}%
\pgfpathlineto{\pgfqpoint{0.728940in}{0.435004in}}%
\pgfpathlineto{\pgfqpoint{0.765065in}{0.443556in}}%
\pgfpathlineto{\pgfqpoint{0.801189in}{0.451796in}}%
\pgfpathlineto{\pgfqpoint{0.837314in}{0.459743in}}%
\pgfpathlineto{\pgfqpoint{0.873438in}{0.467418in}}%
\pgfpathlineto{\pgfqpoint{0.909563in}{0.474839in}}%
\pgfpathlineto{\pgfqpoint{0.945688in}{0.482026in}}%
\pgfpathlineto{\pgfqpoint{0.981812in}{0.488999in}}%
\pgfpathlineto{\pgfqpoint{1.017937in}{0.495776in}}%
\pgfpathlineto{\pgfqpoint{1.054061in}{0.502379in}}%
\pgfpathlineto{\pgfqpoint{1.090186in}{0.508825in}}%
\pgfpathlineto{\pgfqpoint{1.126311in}{0.515135in}}%
\pgfpathlineto{\pgfqpoint{1.162435in}{0.521328in}}%
\pgfpathlineto{\pgfqpoint{1.198560in}{0.527423in}}%
\pgfpathlineto{\pgfqpoint{1.234684in}{0.533440in}}%
\pgfpathlineto{\pgfqpoint{1.270809in}{0.539399in}}%
\pgfpathlineto{\pgfqpoint{1.306933in}{0.545319in}}%
\pgfpathlineto{\pgfqpoint{1.343058in}{0.551220in}}%
\pgfpathlineto{\pgfqpoint{1.379183in}{0.557120in}}%
\pgfpathlineto{\pgfqpoint{1.415307in}{0.563040in}}%
\pgfpathlineto{\pgfqpoint{1.451432in}{0.568999in}}%
\pgfpathlineto{\pgfqpoint{1.487556in}{0.575017in}}%
\pgfpathlineto{\pgfqpoint{1.523681in}{0.581112in}}%
\pgfpathlineto{\pgfqpoint{1.559806in}{0.587305in}}%
\pgfpathlineto{\pgfqpoint{1.595930in}{0.593615in}}%
\pgfpathlineto{\pgfqpoint{1.632055in}{0.600061in}}%
\pgfpathlineto{\pgfqpoint{1.668179in}{0.606663in}}%
\pgfpathlineto{\pgfqpoint{1.704304in}{0.613441in}}%
\pgfpathlineto{\pgfqpoint{1.740429in}{0.620414in}}%
\pgfpathlineto{\pgfqpoint{1.776553in}{0.627601in}}%
\pgfpathlineto{\pgfqpoint{1.812678in}{0.635022in}}%
\pgfpathlineto{\pgfqpoint{1.848802in}{0.642696in}}%
\pgfpathlineto{\pgfqpoint{1.884927in}{0.650644in}}%
\pgfpathlineto{\pgfqpoint{1.921052in}{0.658884in}}%
\pgfpathlineto{\pgfqpoint{1.957176in}{0.667435in}}%
\pgfpathlineto{\pgfqpoint{1.993301in}{0.676318in}}%
\pgfpathlineto{\pgfqpoint{2.029425in}{0.685553in}}%
\pgfpathlineto{\pgfqpoint{2.065550in}{0.695157in}}%
\pgfpathlineto{\pgfqpoint{2.101675in}{0.705151in}}%
\pgfpathlineto{\pgfqpoint{2.137799in}{0.715555in}}%
\pgfpathlineto{\pgfqpoint{2.173924in}{0.726388in}}%
\pgfpathlineto{\pgfqpoint{2.210048in}{0.737669in}}%
\pgfpathlineto{\pgfqpoint{2.246173in}{0.749418in}}%
\pgfpathlineto{\pgfqpoint{2.282298in}{0.761654in}}%
\pgfpathlineto{\pgfqpoint{2.318422in}{0.774397in}}%
\pgfpathlineto{\pgfqpoint{2.354547in}{0.787667in}}%
\pgfpathlineto{\pgfqpoint{2.390671in}{0.801482in}}%
\pgfpathlineto{\pgfqpoint{2.426796in}{0.815863in}}%
\pgfpathlineto{\pgfqpoint{2.462921in}{0.830828in}}%
\pgfpathlineto{\pgfqpoint{2.499045in}{0.846398in}}%
\pgfpathlineto{\pgfqpoint{2.535170in}{0.862592in}}%
\pgfpathlineto{\pgfqpoint{2.571294in}{0.879429in}}%
\pgfpathlineto{\pgfqpoint{2.607419in}{0.896929in}}%
\pgfpathlineto{\pgfqpoint{2.643543in}{0.915111in}}%
\pgfpathlineto{\pgfqpoint{2.679668in}{0.933995in}}%
\pgfpathlineto{\pgfqpoint{2.715793in}{0.953601in}}%
\pgfpathlineto{\pgfqpoint{2.751917in}{0.973947in}}%
\pgfpathlineto{\pgfqpoint{2.788042in}{0.995053in}}%
\pgfpathlineto{\pgfqpoint{2.824166in}{1.016940in}}%
\pgfpathlineto{\pgfqpoint{2.860291in}{1.039625in}}%
\pgfpathlineto{\pgfqpoint{2.896416in}{1.063129in}}%
\pgfpathlineto{\pgfqpoint{2.932540in}{1.087472in}}%
\pgfpathlineto{\pgfqpoint{2.968665in}{1.112672in}}%
\pgfpathlineto{\pgfqpoint{3.004789in}{1.138750in}}%
\pgfpathlineto{\pgfqpoint{3.040914in}{1.165725in}}%
\pgfpathlineto{\pgfqpoint{3.077039in}{1.193615in}}%
\pgfpathlineto{\pgfqpoint{3.113163in}{1.222442in}}%
\pgfpathlineto{\pgfqpoint{3.149288in}{1.252223in}}%
\pgfpathlineto{\pgfqpoint{3.185412in}{1.282980in}}%
\pgfpathlineto{\pgfqpoint{3.221537in}{1.314730in}}%
\pgfpathlineto{\pgfqpoint{3.257662in}{1.347495in}}%
\pgfpathlineto{\pgfqpoint{3.293786in}{1.381293in}}%
\pgfpathlineto{\pgfqpoint{3.329911in}{1.416143in}}%
\pgfpathlineto{\pgfqpoint{3.366035in}{1.452065in}}%
\pgfpathlineto{\pgfqpoint{3.402160in}{1.489080in}}%
\pgfpathlineto{\pgfqpoint{3.438285in}{1.527205in}}%
\pgfpathlineto{\pgfqpoint{3.474409in}{1.566461in}}%
\pgfpathlineto{\pgfqpoint{3.510534in}{1.606868in}}%
\pgfpathlineto{\pgfqpoint{3.546658in}{1.648444in}}%
\pgfpathlineto{\pgfqpoint{3.582783in}{1.691209in}}%
\pgfpathlineto{\pgfqpoint{3.618908in}{1.735183in}}%
\pgfpathlineto{\pgfqpoint{3.655032in}{1.780385in}}%
\pgfpathlineto{\pgfqpoint{3.691157in}{1.826835in}}%
\pgfpathlineto{\pgfqpoint{3.727281in}{1.874552in}}%
\pgfpathlineto{\pgfqpoint{3.763406in}{1.923556in}}%
\pgfpathlineto{\pgfqpoint{3.799531in}{1.973866in}}%
\pgfpathlineto{\pgfqpoint{3.835655in}{2.025501in}}%
\pgfpathlineto{\pgfqpoint{3.871780in}{2.078482in}}%
\pgfpathlineto{\pgfqpoint{3.907904in}{2.132827in}}%
\pgfpathlineto{\pgfqpoint{3.944029in}{2.188556in}}%
\pgfpathlineto{\pgfqpoint{3.980154in}{2.245689in}}%
\pgfpathlineto{\pgfqpoint{4.016278in}{2.304245in}}%
\pgfpathlineto{\pgfqpoint{4.052403in}{2.364244in}}%
\pgfpathlineto{\pgfqpoint{4.088527in}{2.425705in}}%
\pgfpathlineto{\pgfqpoint{4.124652in}{2.488647in}}%
\pgfusepath{stroke}%
\end{pgfscope}%
\begin{pgfscope}%
\pgfpathrectangle{\pgfqpoint{0.548317in}{0.386884in}}{\pgfqpoint{3.576335in}{2.101763in}} %
\pgfusepath{clip}%
\pgfsetroundcap%
\pgfsetroundjoin%
\pgfsetlinewidth{1.756562pt}%
\definecolor{currentstroke}{rgb}{0.552941,0.627451,0.796078}%
\pgfsetstrokecolor{currentstroke}%
\pgfsetdash{}{0pt}%
\pgfpathmoveto{\pgfqpoint{0.548317in}{0.391060in}}%
\pgfpathlineto{\pgfqpoint{0.584442in}{0.400779in}}%
\pgfpathlineto{\pgfqpoint{0.620566in}{0.410181in}}%
\pgfpathlineto{\pgfqpoint{0.656691in}{0.419277in}}%
\pgfpathlineto{\pgfqpoint{0.692815in}{0.428077in}}%
\pgfpathlineto{\pgfqpoint{0.728940in}{0.436592in}}%
\pgfpathlineto{\pgfqpoint{0.765065in}{0.444834in}}%
\pgfpathlineto{\pgfqpoint{0.801189in}{0.452814in}}%
\pgfpathlineto{\pgfqpoint{0.837314in}{0.460547in}}%
\pgfpathlineto{\pgfqpoint{0.873438in}{0.468044in}}%
\pgfpathlineto{\pgfqpoint{0.909563in}{0.475322in}}%
\pgfpathlineto{\pgfqpoint{0.945688in}{0.482393in}}%
\pgfpathlineto{\pgfqpoint{0.981812in}{0.489274in}}%
\pgfpathlineto{\pgfqpoint{1.017937in}{0.495979in}}%
\pgfpathlineto{\pgfqpoint{1.054061in}{0.502525in}}%
\pgfpathlineto{\pgfqpoint{1.090186in}{0.508928in}}%
\pgfpathlineto{\pgfqpoint{1.126311in}{0.515206in}}%
\pgfpathlineto{\pgfqpoint{1.162435in}{0.521376in}}%
\pgfpathlineto{\pgfqpoint{1.198560in}{0.527454in}}%
\pgfpathlineto{\pgfqpoint{1.234684in}{0.533460in}}%
\pgfpathlineto{\pgfqpoint{1.270809in}{0.539411in}}%
\pgfpathlineto{\pgfqpoint{1.306933in}{0.545326in}}%
\pgfpathlineto{\pgfqpoint{1.343058in}{0.551223in}}%
\pgfpathlineto{\pgfqpoint{1.379183in}{0.557122in}}%
\pgfpathlineto{\pgfqpoint{1.415307in}{0.563041in}}%
\pgfpathlineto{\pgfqpoint{1.451432in}{0.568999in}}%
\pgfpathlineto{\pgfqpoint{1.487556in}{0.575016in}}%
\pgfpathlineto{\pgfqpoint{1.523681in}{0.581112in}}%
\pgfpathlineto{\pgfqpoint{1.559806in}{0.587305in}}%
\pgfpathlineto{\pgfqpoint{1.595930in}{0.593615in}}%
\pgfpathlineto{\pgfqpoint{1.632055in}{0.600061in}}%
\pgfpathlineto{\pgfqpoint{1.668179in}{0.606663in}}%
\pgfpathlineto{\pgfqpoint{1.704304in}{0.613441in}}%
\pgfpathlineto{\pgfqpoint{1.740429in}{0.620414in}}%
\pgfpathlineto{\pgfqpoint{1.776553in}{0.627601in}}%
\pgfpathlineto{\pgfqpoint{1.812678in}{0.635022in}}%
\pgfpathlineto{\pgfqpoint{1.848802in}{0.642696in}}%
\pgfpathlineto{\pgfqpoint{1.884927in}{0.650644in}}%
\pgfpathlineto{\pgfqpoint{1.921052in}{0.658884in}}%
\pgfpathlineto{\pgfqpoint{1.957176in}{0.667435in}}%
\pgfpathlineto{\pgfqpoint{1.993301in}{0.676318in}}%
\pgfpathlineto{\pgfqpoint{2.029425in}{0.685553in}}%
\pgfpathlineto{\pgfqpoint{2.065550in}{0.695157in}}%
\pgfpathlineto{\pgfqpoint{2.101675in}{0.705151in}}%
\pgfpathlineto{\pgfqpoint{2.137799in}{0.715555in}}%
\pgfpathlineto{\pgfqpoint{2.173924in}{0.726388in}}%
\pgfpathlineto{\pgfqpoint{2.210048in}{0.737669in}}%
\pgfpathlineto{\pgfqpoint{2.246173in}{0.749418in}}%
\pgfpathlineto{\pgfqpoint{2.282298in}{0.761654in}}%
\pgfpathlineto{\pgfqpoint{2.318422in}{0.774397in}}%
\pgfpathlineto{\pgfqpoint{2.354547in}{0.787667in}}%
\pgfpathlineto{\pgfqpoint{2.390671in}{0.801482in}}%
\pgfpathlineto{\pgfqpoint{2.426796in}{0.815863in}}%
\pgfpathlineto{\pgfqpoint{2.462921in}{0.830828in}}%
\pgfpathlineto{\pgfqpoint{2.499045in}{0.846398in}}%
\pgfpathlineto{\pgfqpoint{2.535170in}{0.862592in}}%
\pgfpathlineto{\pgfqpoint{2.571294in}{0.879429in}}%
\pgfpathlineto{\pgfqpoint{2.607419in}{0.896928in}}%
\pgfpathlineto{\pgfqpoint{2.643543in}{0.915111in}}%
\pgfpathlineto{\pgfqpoint{2.679668in}{0.933995in}}%
\pgfpathlineto{\pgfqpoint{2.715793in}{0.953600in}}%
\pgfpathlineto{\pgfqpoint{2.751917in}{0.973947in}}%
\pgfpathlineto{\pgfqpoint{2.788042in}{0.995055in}}%
\pgfpathlineto{\pgfqpoint{2.824166in}{1.016942in}}%
\pgfpathlineto{\pgfqpoint{2.860291in}{1.039628in}}%
\pgfpathlineto{\pgfqpoint{2.896416in}{1.063133in}}%
\pgfpathlineto{\pgfqpoint{2.932540in}{1.087475in}}%
\pgfpathlineto{\pgfqpoint{2.968665in}{1.112673in}}%
\pgfpathlineto{\pgfqpoint{3.004789in}{1.138748in}}%
\pgfpathlineto{\pgfqpoint{3.040914in}{1.165721in}}%
\pgfpathlineto{\pgfqpoint{3.077039in}{1.193616in}}%
\pgfpathlineto{\pgfqpoint{3.113163in}{1.222465in}}%
\pgfpathlineto{\pgfqpoint{3.149288in}{1.252307in}}%
\pgfpathlineto{\pgfqpoint{3.185412in}{1.283195in}}%
\pgfpathlineto{\pgfqpoint{3.221537in}{1.315210in}}%
\pgfpathlineto{\pgfqpoint{3.257662in}{1.348463in}}%
\pgfpathlineto{\pgfqpoint{3.293786in}{1.383121in}}%
\pgfpathlineto{\pgfqpoint{3.329911in}{1.419427in}}%
\pgfpathlineto{\pgfqpoint{3.366035in}{1.457730in}}%
\pgfpathlineto{\pgfqpoint{3.402160in}{1.498534in}}%
\pgfpathlineto{\pgfqpoint{3.438285in}{1.542553in}}%
\pgfpathlineto{\pgfqpoint{3.474409in}{1.590795in}}%
\pgfpathlineto{\pgfqpoint{3.510534in}{1.644658in}}%
\pgfpathlineto{\pgfqpoint{3.546658in}{1.706077in}}%
\pgfpathlineto{\pgfqpoint{3.582783in}{1.777690in}}%
\pgfpathlineto{\pgfqpoint{3.618908in}{1.863075in}}%
\pgfpathlineto{\pgfqpoint{3.655032in}{1.967039in}}%
\pgfpathlineto{\pgfqpoint{3.691157in}{2.095992in}}%
\pgfpathlineto{\pgfqpoint{3.727281in}{2.258420in}}%
\pgfpathlineto{\pgfqpoint{3.763406in}{2.465484in}}%
\pgfpathlineto{\pgfqpoint{3.768433in}{2.502536in}}%
\pgfusepath{stroke}%
\end{pgfscope}%
\begin{pgfscope}%
\pgfsetrectcap%
\pgfsetmiterjoin%
\pgfsetlinewidth{1.254687pt}%
\definecolor{currentstroke}{rgb}{0.150000,0.150000,0.150000}%
\pgfsetstrokecolor{currentstroke}%
\pgfsetdash{}{0pt}%
\pgfpathmoveto{\pgfqpoint{0.548317in}{0.386884in}}%
\pgfpathlineto{\pgfqpoint{0.548317in}{2.488647in}}%
\pgfusepath{stroke}%
\end{pgfscope}%
\begin{pgfscope}%
\pgfsetrectcap%
\pgfsetmiterjoin%
\pgfsetlinewidth{1.254687pt}%
\definecolor{currentstroke}{rgb}{0.150000,0.150000,0.150000}%
\pgfsetstrokecolor{currentstroke}%
\pgfsetdash{}{0pt}%
\pgfpathmoveto{\pgfqpoint{0.548317in}{0.386884in}}%
\pgfpathlineto{\pgfqpoint{4.124652in}{0.386884in}}%
\pgfusepath{stroke}%
\end{pgfscope}%
\begin{pgfscope}%
\pgfsetbuttcap%
\pgfsetroundjoin%
\definecolor{currentfill}{rgb}{0.400000,0.760784,0.647059}%
\pgfsetfillcolor{currentfill}%
\pgfsetlinewidth{0.000000pt}%
\definecolor{currentstroke}{rgb}{0.400000,0.760784,0.647059}%
\pgfsetstrokecolor{currentstroke}%
\pgfsetdash{}{0pt}%
\pgfsys@defobject{currentmarker}{\pgfqpoint{-0.048611in}{-0.048611in}}{\pgfqpoint{0.048611in}{0.048611in}}{%
\pgfpathmoveto{\pgfqpoint{0.000000in}{-0.048611in}}%
\pgfpathcurveto{\pgfqpoint{0.012892in}{-0.048611in}}{\pgfqpoint{0.025257in}{-0.043489in}}{\pgfqpoint{0.034373in}{-0.034373in}}%
\pgfpathcurveto{\pgfqpoint{0.043489in}{-0.025257in}}{\pgfqpoint{0.048611in}{-0.012892in}}{\pgfqpoint{0.048611in}{0.000000in}}%
\pgfpathcurveto{\pgfqpoint{0.048611in}{0.012892in}}{\pgfqpoint{0.043489in}{0.025257in}}{\pgfqpoint{0.034373in}{0.034373in}}%
\pgfpathcurveto{\pgfqpoint{0.025257in}{0.043489in}}{\pgfqpoint{0.012892in}{0.048611in}}{\pgfqpoint{0.000000in}{0.048611in}}%
\pgfpathcurveto{\pgfqpoint{-0.012892in}{0.048611in}}{\pgfqpoint{-0.025257in}{0.043489in}}{\pgfqpoint{-0.034373in}{0.034373in}}%
\pgfpathcurveto{\pgfqpoint{-0.043489in}{0.025257in}}{\pgfqpoint{-0.048611in}{0.012892in}}{\pgfqpoint{-0.048611in}{0.000000in}}%
\pgfpathcurveto{\pgfqpoint{-0.048611in}{-0.012892in}}{\pgfqpoint{-0.043489in}{-0.025257in}}{\pgfqpoint{-0.034373in}{-0.034373in}}%
\pgfpathcurveto{\pgfqpoint{-0.025257in}{-0.043489in}}{\pgfqpoint{-0.012892in}{-0.048611in}}{\pgfqpoint{0.000000in}{-0.048611in}}%
\pgfpathclose%
\pgfusepath{fill}%
}%
\begin{pgfscope}%
\pgfsys@transformshift{0.812206in}{2.311974in}%
\pgfsys@useobject{currentmarker}{}%
\end{pgfscope}%
\end{pgfscope}%
\begin{pgfscope}%
\definecolor{textcolor}{rgb}{0.150000,0.150000,0.150000}%
\pgfsetstrokecolor{textcolor}%
\pgfsetfillcolor{textcolor}%
\pgftext[x=1.062206in,y=2.263363in,left,base]{\color{textcolor}\sffamily\fontsize{10.000000}{12.000000}\selectfont samples}%
\end{pgfscope}%
\begin{pgfscope}%
\pgfsetbuttcap%
\pgfsetroundjoin%
\pgfsetlinewidth{1.756562pt}%
\definecolor{currentstroke}{rgb}{0.988235,0.552941,0.384314}%
\pgfsetstrokecolor{currentstroke}%
\pgfsetdash{{5.600000pt}{2.400000pt}}{0.000000pt}%
\pgfpathmoveto{\pgfqpoint{0.673317in}{2.115246in}}%
\pgfpathlineto{\pgfqpoint{0.951095in}{2.115246in}}%
\pgfusepath{stroke}%
\end{pgfscope}%
\begin{pgfscope}%
\definecolor{textcolor}{rgb}{0.150000,0.150000,0.150000}%
\pgfsetstrokecolor{textcolor}%
\pgfsetfillcolor{textcolor}%
\pgftext[x=1.062206in,y=2.066635in,left,base]{\color{textcolor}\sffamily\fontsize{10.000000}{12.000000}\selectfont function}%
\end{pgfscope}%
\begin{pgfscope}%
\pgfsetroundcap%
\pgfsetroundjoin%
\pgfsetlinewidth{1.756562pt}%
\definecolor{currentstroke}{rgb}{0.552941,0.627451,0.796078}%
\pgfsetstrokecolor{currentstroke}%
\pgfsetdash{}{0pt}%
\pgfpathmoveto{\pgfqpoint{0.673317in}{1.918518in}}%
\pgfpathlineto{\pgfqpoint{0.951095in}{1.918518in}}%
\pgfusepath{stroke}%
\end{pgfscope}%
\begin{pgfscope}%
\definecolor{textcolor}{rgb}{0.150000,0.150000,0.150000}%
\pgfsetstrokecolor{textcolor}%
\pgfsetfillcolor{textcolor}%
\pgftext[x=1.062206in,y=1.869907in,left,base]{\color{textcolor}\sffamily\fontsize{10.000000}{12.000000}\selectfont 20th order fit}%
\end{pgfscope}%
\end{pgfpicture}%
\makeatother%
\endgroup%
}
			\caption{\engordnumber{20}-order polynomial fitting 10 points}
			\label{fig:polyfit20thlots}
		\end{subfigure}
		\caption[Polynomial Data Fitting]{Polynomial fits of samples from a 3rd order function. Polynomials of high order, like neural networks of many parameters, easily overfit a small number of samples as compared to polynomials of a more suitable order for the sampled function. While generalization is helped by more data, the higher order polynomial still tends to overfit.}
		\label{fig:polyfits}
	\end{figure}

	\citet{denker1987large} explored the relationship of network architecture to generalization. The work was particularly motivating in the later design of convolutional neural networks~\citep{lecun1989generalization, lecun1989backpropagation}. The authors make the intuitive analogy between the affect of the size of a neural network on its generalization, and a simple least-squares polynomial fit. Fig.~\ref{fig:polyfits} shows various polynomial fits to samples from a \engordnumber{3}-order polynomial function. When using a \engordnumber{3}-order polynomial to fit even a small number of samples (Fig.~\ref{fig:polyfit3rd}), the fit extrapolates, \ie is closer to the desired function outside the range of training samples, better than when we use a \engordnumber{20}-order polynomial to fit the same data (Fig.~\ref{fig:polyfit20th}). While the number of samples can help the fit of the higher order function, even with a large number of samples the \engordnumber{20}-order polynomial fit  (Fig.~\ref{fig:polyfit20thlots}) will not extrapolate as well as the polynomial with a more appropriate lower number of parameters (Fig.~\ref{fig:polyfit3rdlots}). Similarly, a neural network with a large number of parameters may not generalize as well as a neural network with fewer more salient parameters.
	
	
	\subsection{Structural Priors}
	In practice when fitting a curve, we have little idea of what order polynomial would best fit the data. Necessarily, we must use a relatively higher order curve to fit the data. Similarly, with a neural network we rarely have knowledge of the underlying structure of the solution (but when we do, we should use it to parameterize our models appropriately~\citep{jain2016structural}), instead we must use networks with more parameters than necessary to ensure that there is enough capacity to learn the underlying, but potentially sparse solution. Over-parameterization of a model generally leads to poor generalization however, due to overfitting. To prevent this, there are a number of methods in which we can relate our prior knowledge that the model is over-parameterized to the optimization:
	
	\begin{description}
	\item[Weak Priors]
	Knowing only that our model is over-parameterized is a relatively weak prior, however we can encode this into the fit by using a regularization penalty. This restricts the model to effectively use only a small number of the parameters by adding a penalty, for example on the L2 norm of the model weights. For polynomial regression, this is called \emph{ridge regression}, while for neural networks it is called \emph{weight decay}.
	
	\item[Strong Priors]
	With more prior information on the task, \eg from the convexity of the polynomial, we may imply that a certain order polynomial is more appropriate, and restrict learning to that order. For example, given samples from a polynomial appear to be convex, we can surmise that the polynomial is likely to be of an even or \engordnumber{2}-order, and restrict our fit to be of that order. 
    \end{description}
    
    
    \section{Deep Networks}
	% deep boltzmann networks
	\citep{Krizhevsky2012}
	\citep{Simonyan2014verydeep}
	\citep{He2015}
	\citep{He2016}
    
	
	%% NOTE: Polynomial fits are a special case of linear regression where we've used a polynomial basis
	%% Use example of sin wave? We can fit it with a polynomial, but better to reparameterize into a basis
	
	\subsection{Generalization and Parameters in Neural Networks}
	\begin{figure}[tb]
		\centering
		\large
        \renewcommand{\ttdefault}{pcr}
		\begin{subfigure}[t]{0.45\textwidth}
			\begin{center}
			\texttt{00\textbf{11111}00}\\
			\texttt{00\textbf{111}0000}\\
			\texttt{0000\textbf{1}0000}
			\end{center}
			\caption{Binary sequences with one clump}
			\label{fig:oneclump}
		\end{subfigure}
		~
		\begin{subfigure}[t]{0.45\textwidth}
			\begin{center}
			\texttt{00\textbf{11}0\textbf{11}00}\\
			\texttt{00\textbf{1}0\textbf{1}0\textbf{1}00}\\
			\texttt{00\textbf{111}0\textbf{1}00}
			\end{center}
			\caption{Binary sequences with two or more clumps}
			\label{fig:polyfit20th}
		\end{subfigure}
		
        \renewcommand{\ttdefault}{lmodern}
		\caption[Two-or-more Clump Predicate]{The two-or-more clumps predicate asks for the network to classify (padded) binary input sequences as having one or two or more contiguous strings of ones.}
		\label{fig:tomclumps}
	\end{figure}
	
	\citet{denker1987large} explore the relationship between network architecture and generalization by evaluating networks for solving the \emph{two-or-more clumps} predicate. The two-or-more clumps predicate asks for the network to classify binary input sequences as having one or two or more contiguous strings of ones, some examples of which are shown in Fig.~\ref{fig:tomclumps}. 
	
	The authors illustrate some surprising properties of the generalization of fully-connected neural networks learned with backpropogation. First, a human-preferred `geometric' solution is manually hard-coded into the weights of a fully-connected network. While this weight configuration is a valid solution, and is intuitive to humans, it does not seem to be a solution that the network would ever settle upon. By using the geometric solution as an initialization, and training the network further, the authors show that the error-surface around the region is not stable,
	
\section{The Neocognitron}
	\begin{chapquote}{Kunihiko Fukushima, \textit{Neocognitron, 
				%A Self-organizing Neural Network Model
				%for a Mechanism of Pattern Recognition
				%Unaffected by Shift in Position , 
				Biol.\ Cybernetics 1980}}
		`` One of the largest and long-standing difficulties in designing a pattern-recognizing machine has been the problem how to cope with the shift in position and the distortion in shape of the input patterns. The neocognitron proposed in this paper gives a drastic solution to this difficulty.
		''
	\end{chapquote}
	
	In an era in which fully connected networks were used to learn any input type, \citet{Fuk80} showed that for structured inputs a drastically different architecture could make a big difference in generalization. The \emph{Neocognitron}~\citep{Fuk80, fukushima2013artificial} was a biologically motivated architecture, motivated by what are typically called simple and complex cells in the primary visual cortex (V1), as found by \citet{Hubel1959a}. To model simple cells; cells whose response correlated with simple oriented edges in a translation invariant manner, the neocognitron used shared weights which were connected to local image patches of the input image (and were not simply described as convolution of a filter). Complex cells were modelled by a ``blurring'' operation, what we now term more generally as \emph{pooling}. The neocognitron network consisted of alternating layers of simple and complex cells, \ie alternating convolution and pooling layers, much as seen in state of the art convolutional networks.

	While the Neocognitron was ahead of its time, and is now recognized as the first iteration of what were to become convolutional neural networks, the article's neurological focus, timing, and to some degree country of origin, meant that it was somewhat unnoticed in the mainstream field of connectionist research. In fact Yann LeCun recounts specifically his interest in Japan, rather than any particular citation, having lead to his discovery of the work\footnote{As related in a Q\&A session at the 2016 International Computer Vision Summer School (ICVSS)}.
	
	\section{Convolutional Neural Networks}
	\begin{chapquote}{Yann LeCun, \textit{Backpropagation Applied to Handwritten Zip Code Recognition, 1989}}
		``Classical work in visual pattern recognition has demonstrated the advantage of extracting local features and combining them to form higher order features. Such knowledge can be easily built into the network by forcing the hidden units to combine only local sources of information. Distinctive features of an object can appear at various location on the input image. Therefore it seems judicious to have a set of feature detectors that can detect a particular instance of a feature anywhere on the input place. Since the precise location of a feature is not relevant to the classification, we can afford to lose some position information in the process. Nevertheless, approximate position information must be preserved, to allow the next levels to detect higher order, more complex features.''
	\end{chapquote}
	
	Despite the pioneering novelty of the work on neocognitrons, it was only following the simplification and improvement of~\citet{lecun1989backpropagation,Lecun1998} in both the description of the network and its operation that it gained wider acknowledgement as a breakthrough for image recognition. In their work the local shared weights of the neocognitron are put in the context of convolution, and the averaging operation replaces with max-pooling. The application to handwritten digit recognition gave state of the art results, and would result in the \emph{LeNet5} network, still used today in commercial applications.
	
	The application of the LeNet-style CNN architecture to more complex problems, however, proved infeasible. These problems required a deeper hierarchy of representation, which implied a large number of layers. Networks with a large number of layers proved to be un-trainable due to numerous issues with the model itself, notably vanishing gradients~\citep{hochreiter1991untersuchungen}, and the lack of large datasets and computational power at the time. Convolutional neural networks fell out of favour, and were passed over in favour of the more successful paradigm of using hand-crafted local features for many tasks, and in particular the problem of object instance recognition was well addressed by such solutions. Meanwhile object class recognition remained a difficult problem, for which the best solutions were deformable parts models, also based on local features.
	
	
	\begin{chapquote}{David MacKay, \textit{A Practical Bayesian Framework for Backprop Networks, Neural Computation 1991}}
		``There are many knobs on the black box of 'backprop' (learning by back-propagation of
		errors). Generally these knobs are set by rules of thumb, trial and error, and the use of reserved test data to assess generalisation ability (or more sophisticated cross-validation).''
	\end{chapquote}
	
\end{document}
