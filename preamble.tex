% !TEX root = thesis.tex
% ******************************************************************************
% ****************************** Custom Margin *********************************

% Add `custommargin' in the document class options to use this section
% Set {innerside margin / outerside margin / topmargin / bottom margin}  and
% other page dimensions
\ifsetCustomMargin
  \RequirePackage[left=37mm,right=30mm,top=35mm,bottom=30mm]{geometry}
  \setFancyHdr % To apply fancy header after geometry package is loaded
\fi

% Add spaces between paragraphs
%\setlength{\parskip}{0.5em}
% Ragged bottom avoids extra whitespaces between paragraphs
\raggedbottom
% To remove the excess top spacing for enumeration, list and description
%\usepackage{enumitem}
%\setlist[enumerate,itemize,description]{topsep=0em}

% *****************************************************************************
% ******************* AMS Font/Symbol Packages (to not conflict with font) ****

\usepackage{amsmath, amsthm, amssymb}

% *****************************************************************************
% ******************* Fonts (like different typewriter fonts etc.)*************

% Add `customfont' in the document class option to use this section

\ifsetCustomFont
  % Set your custom font here and use `customfont' in options. Leave empty to
  % load computer modern font (default LaTeX font).
  %\RequirePackage{helvet}
  % \RequirePackage{newtxtext,newtxmath}
  \RequirePackage{lmodern}
  % \RequirePackage{tgtermes,newtxmath}

  % For use with XeLaTeX
  %  \setmainfont[
  %    Path              = ./libertine/opentype/,
  %    Extension         = .otf,
  %    UprightFont = LinLibertine_R,
  %    BoldFont = LinLibertine_RZ, % Linux Libertine O Regular Semibold
  %    ItalicFont = LinLibertine_RI,
  %    BoldItalicFont = LinLibertine_RZI, % Linux Libertine O Regular Semibold Italic
  %  ]
  %  {libertine}
  %  % load font from system font
  %  \newfontfamily\libertinesystemfont{Linux Libertine O}
\fi

% *****************************************************************************
% **************************** Custom Packages ********************************

% ************************* Algorithms and Pseudocode **************************

%\usepackage{algpseudocode}
\usepackage{listings}


% ********************Captions and Hyperreferencing / URL **********************

% Captions: This makes captions of figures use a boldfaced small font.
%\RequirePackage[small,bf]{caption}

\RequirePackage[labelsep=space,tableposition=top]{caption}
\renewcommand{\figurename}{Fig.} %to support older versions of captions.sty


% *************************** Graphics and figures *****************************

%\usepackage{rotating}
%\usepackage{wrapfig}

% Uncomment the following two lines to force Latex to place the figure.
% Use [H] when including graphics. Note 'H' instead of 'h'
%\usepackage{float}
%\restylefloat{figure}

% Subcaption package is also available in the sty folder you can use that by
% uncommenting the following line
% This is for people stuck with older versions of texlive
%\usepackage{sty/caption/subcaption}
\usepackage[labelformat=simple]{subcaption}
\renewcommand\thesubfigure{(\alph{subfigure})}

% ********************************** Tables ************************************
\usepackage{booktabs} % For professional looking tables
\usepackage{multirow}

%\usepackage{multicol}
%\usepackage{longtable}
%\usepackage{tabularx}


% *********************************** SI Units *********************************
\usepackage{siunitx} % use this package module for SI units

% ******************************* Line Spacing *********************************

% Choose linespacing as appropriate. Default is one-half line spacing as per the
% University guidelines

% \doublespacing
% \onehalfspacing
% \singlespacing


% ************************ Formatting / Footnote *******************************

% Don't break enumeration (etc.) across pages in an ugly manner (default 10000)
%\clubpenalty=500
%\widowpenalty=500

%\usepackage[perpage]{footmisc} %Range of footnote options

% ***************************** Epi-graph for inspirational quotes *************************************
\makeatletter
\renewcommand{\@chapapp}{}% Not necessary...
\newenvironment{chapquote}[2][1.5em]
  {\setlength{\@tempdima}{#1}%
   \def\chapquote@author{#2}%
   \parshape 1 \@tempdima \dimexpr\textwidth-2\@tempdima\relax%
   \itshape}
  {\par\normalfont\hfill--\ \chapquote@author\hspace*{\@tempdima}\par\bigskip}
\makeatother

% ********************************** Sub-Files ************************************
\usepackage{subfiles} % To work seamlessly with mutiple file projects

% *****************************************************************************
% *************************** Bibliography  and References ********************

\usepackage[capitalise]{cleveref} %Referencing without need to explicitly state fig /table

% Add `custombib' in the document class option to use this section
\ifuseCustomBib
%   \RequirePackage[square, sort, numbers, authoryear]{natbib} % CustomBib

% If you would like to use biblatex for your reference management, as opposed to the default `natbibpackage` pass the option `custombib` in the document class. Comment out the previous line to make sure you don't load the natbib package. Uncomment the following lines and specify the location of references.bib file
  \RequirePackage[backend=biber,
    natbib=true,
    hyperref=true,
    url=false,
    isbn=false,
    backref=true,
    style=authoryear,
    citereset=chapter,
    maxcitenames=3,
    maxbibnames=100,
    block=none]{biblatex}
	\bibliography{references} %Location of references.bib only for biblatex
	% Use author initials only for consistency in inconsistent refs
\fi
% changes the default name `Bibliography` -> `References'
\renewcommand{\bibname}{References}


% ******************************************************************************
% ************************* User Defined Commands ******************************
% ******************************************************************************

% *********** To change the name of Table of Contents / LOF and LOT ************

%\renewcommand{\contentsname}{My Table of Contents}
%\renewcommand{\listfigurename}{My List of Figures}
%\renewcommand{\listtablename}{My List of Tables}


% ********************** TOC depth and numbering depth *************************

\setcounter{secnumdepth}{2}
\setcounter{tocdepth}{2}


% ******************************* Nomenclature *********************************

% To change the name of the Nomenclature section, uncomment the following line

%\renewcommand{\nomname}{Symbols}


% ********************************* Appendix ***********************************

% The default value of both \appendixtocname and \appendixpagename is `Appendices'. These names can all be changed via:

%\renewcommand{\appendixtocname}{List of appendices}
%\renewcommand{\appendixname}{Appndx}

% *********************** Configure Draft Mode **********************************

% Uncomment to disable figures in `draft'
%\setkeys{Gin}{draft=true}  % set draft to false to enable figures in `draft'

% These options are active only during the draft mode
% Default text is "Draft"
%\SetDraftText{DRAFT}

% Default Watermark location is top. Location (top/bottom)
%\SetDraftWMPosition{bottom}

% Draft Version - default is v1.0
%\SetDraftVersion{v1.1}

% Draft Text grayscale value (should be between 0-black and 1-white)
% Default value is 0.75
%\SetDraftGrayScale{0.8}


% ******************************** Todo Notes **********************************
%% Uncomment the following lines to have todonotes.

\ifsetDraft
	\usepackage[colorinlistoftodos]{todonotes}
	\newcommand{\mynote}[1]{\todo[author=yani,size=\small,inline,color=green!40]{#1}}
\else
	\newcommand{\mynote}[1]{}
	\newcommand{\listoftodos}{}
\fi

% Example todo: \mynote{Hey! I have a note}

% **************************** Define Graphics Path **************************
\ifpdf
    \graphicspath{{Figs/PDF/}{Figs/Raster/}{Figs/Vector/}{Figs/}}
\else
    \graphicspath{{Figs/EPS/}{Figs/Raster/}{Figs/Vector/}{Figs/}}
\fi

\usepackage{tikz}
\usetikzlibrary{arrows, colorbrewer}

\definecolor{Set1-A}{RGB}{228,26,28}
\definecolor{Set1-B}{RGB}{55,126,184}
\definecolor{Set1-C}{RGB}{77,175,74}
\definecolor{Set1-D}{RGB}{152,78,163}
\definecolor{Set1-E}{RGB}{255,127,0}
\definecolor{Set1-F}{RGB}{255,255,51}
\definecolor{Set1-G}{RGB}{166,86,40}
\definecolor{Set1-H}{RGB}{247,129,191}
\definecolor{Set1-I}{RGB}{153,153,153}

\definecolor{Set2-A}{RGB}{102,194,165}
\definecolor{Set2-B}{RGB}{252,141,98}
\definecolor{Set2-C}{RGB}{141,160,203}
\definecolor{Set2-D}{RGB}{231,138,195}
\definecolor{Set2-E}{RGB}{166,216,84}
\definecolor{Set2-F}{RGB}{255,217,47}
\definecolor{Set2-G}{RGB}{229,196,148}
\definecolor{Set2-H}{RGB}{179,179,179}

\definecolor{Set3-A}{RGB}{141,211,199}
\definecolor{Set3-B}{RGB}{255,255,179}
\definecolor{Set3-C}{RGB}{190,186,218}
\definecolor{Set3-D}{RGB}{251,128,114}
\definecolor{Set3-E}{RGB}{128,177,211}
\definecolor{Set3-F}{RGB}{253,180,98}
\definecolor{Set3-G}{RGB}{179,222,105}
\definecolor{Set3-H}{RGB}{252,205,229}
\definecolor{Set3-I}{RGB}{217,217,217}
\definecolor{Set3-J}{RGB}{188,128,189}
\definecolor{Set3-K}{RGB}{204,235,197}
\definecolor{Set3-L}{RGB}{255,237,111}

% COMMENT OUT NEXT TWO LINES TO USE SANS-SERIF WITH TIKZ/PGFPLOTS
\usepackage[eulergreek]{sansmath}
\tikzstyle{every picture}+=[font=\sffamily\sansmath]

\usepackage{pgfplots, pgfplotstable}
\usepgfplotslibrary{statistics, colorbrewer}
\pgfplotsset{
  compat=newest,
  cycle list/Set1-3,
  cycle list/Set1-4,
  cycle list/Set1-5,
  cycle list/Set1-6,
  cycle list/Set1-7,
  cycle list/Set1-8,
  cycle list/PuOr-3,
  cycle list/PuOr-4,
  cycle list/PuOr-5,
  cycle list/PuOr-6,
  cycle list/PuOr-8,
  cycle list/PuOr-9,
  cycle list/OrRd-3,
  cycle list/OrRd-4,
  cycle list/OrRd-5,
  cycle list/OrRd-6,
  cycle list/OrRd-7,
  cycle list/OrRd-8,
  cycle list/OrRd-9,
}

% matplotlib parula JET replacement

\pgfplotsset{%
  colormap={parula}{%
    rgb=(0.2081,0.1663,0.5292)rgb=(0.2116,0.1898,0.5777)rgb=(0.2123,0.2138,0.627)
    rgb=(0.2081,0.2386,0.6771)rgb=(0.1959,0.2645,0.7279)rgb=(0.1707,0.2919,0.7792)
    rgb=(0.1253,0.3242,0.8303)rgb=(0.0591,0.3598,0.8683)rgb=(0.0117,0.3875,0.882)
    rgb=(0.006,0.4086,0.8828) rgb=(0.0165,0.4266,0.8786)rgb=(0.0329,0.443,0.872)
    rgb=(0.0498,0.4586,0.8641)rgb=(0.0629,0.4737,0.8554)rgb=(0.0723,0.4887,0.8467)
    rgb=(0.0779,0.504,0.8384) rgb=(0.0793,0.52,0.8312)  rgb=(0.0749,0.5375,0.8263)
    rgb=(0.0641,0.557,0.824)  rgb=(0.0488,0.5772,0.8228)rgb=(0.0343,0.5966,0.8199)
    rgb=(0.0265,0.6137,0.8135)rgb=(0.0239,0.6287,0.8038)rgb=(0.0231,0.6418,0.7913)
    rgb=(0.0228,0.6535,0.7768)rgb=(0.0267,0.6642,0.7607)rgb=(0.0384,0.6743,0.7436)
    rgb=(0.059,0.6838,0.7254) rgb=(0.0843,0.6928,0.7062)rgb=(0.1133,0.7015,0.6859)
    rgb=(0.1453,0.7098,0.6646)rgb=(0.1801,0.7177,0.6424)rgb=(0.2178,0.725,0.6193)
    rgb=(0.2586,0.7317,0.5954)rgb=(0.3022,0.7376,0.5712)rgb=(0.3482,0.7424,0.5473)
    rgb=(0.3953,0.7459,0.5244)rgb=(0.442,0.7481,0.5033) rgb=(0.4871,0.7491,0.484)
    rgb=(0.53,0.7491,0.4661)  rgb=(0.5709,0.7485,0.4494)rgb=(0.6099,0.7473,0.4337)
    rgb=(0.6473,0.7456,0.4188)rgb=(0.6834,0.7435,0.4044)rgb=(0.7184,0.7411,0.3905)
    rgb=(0.7525,0.7384,0.3768)rgb=(0.7858,0.7356,0.3633)rgb=(0.8185,0.7327,0.3498)
    rgb=(0.8507,0.7299,0.336) rgb=(0.8824,0.7274,0.3217)rgb=(0.9139,0.7258,0.3063)
    rgb=(0.945,0.7261,0.2886) rgb=(0.9739,0.7314,0.2666)rgb=(0.9938,0.7455,0.2403)
    rgb=(0.999,0.7653,0.2164) rgb=(0.9955,0.7861,0.1967)rgb=(0.988,0.8066,0.1794)
    rgb=(0.9789,0.8271,0.1633)rgb=(0.9697,0.8481,0.1475)rgb=(0.9626,0.8705,0.1309)
    rgb=(0.9589,0.8949,0.1132)rgb=(0.9598,0.9218,0.0948)rgb=(0.9661,0.9514,0.0755)
    rgb=(0.9763,0.9831,0.0538)
  }
}

%%%%%%%%%%%%%%%%%%%%%%%%%%% begin color
% https://tex.stackexchange.com/questions/64131/implementing-switch-cases#64132
\usepackage{pdftexcmds}
\makeatletter
\newcommand{\stringcases}[3]{%
\romannumeral
  \str@case{#1}#2{#1}{#3}\q@stop
}

\newcommand{\str@case}[3]{%
\ifnum\pdf@strcmp{\unexpanded{#1}}{\unexpanded{#2}}=\z@
  \expandafter\@firstoftwo
\else
  \expandafter\@secondoftwo
\fi
  {\str@case@end{#3}}
  {\str@case{#1}}%
}

\newcommand{\str@case@end}{}
\long\def\str@case@end#1#2\q@stop{\z@#1}
\makeatother


% for other plots that can't use colormaps
\newcommand{\colora}{Set1-A}
\newcommand{\colorb}{Set1-B}
\newcommand{\colorc}{Set1-C}
% https://tex.stackexchange.com/questions/170221/pgfplots-line-colors#170223
%%%%%%%%%%%%%%%%%%%%%%%%%%% define marker list
\pgfplotsset{
  % define a `cycle list' for marker
  cycle list/.define={markercycle}{
      every mark/.append style={solid,fill=\pgfkeysvalueof{/pgfplots/mark list fill}},mark=*\\
      every mark/.append style={solid,fill=\pgfkeysvalueof{/pgfplots/mark list fill}},mark=square*\\
      every mark/.append style={solid,fill=\pgfkeysvalueof{/pgfplots/mark list fill}},mark=triangle*\\
      every mark/.append style={solid,fill=\pgfkeysvalueof{/pgfplots/mark list fill}},mark=diamond*\\
  },
}
%%%%%%%%%%%%%%%%%%%%%%%%%%% end marker list
% define colour cycles to use throughout for given number of items
\newcommand*{\setplotcyclecat}[1]{%
\stringcases
  {#1}%
  {%
    {1}{cycle list name = Set1-3}%
    {2}{cycle list name = Set1-3}
    {3}{cycle list name = Set1-3}%
    {4}{cycle list name = Set1-4}%
    {5}{cycle list name = Set1-5}%
    {6}{cycle list name = Set1-6}%
    {7}{cycle list name = Set1-7}%
    {8}{cycle list name = Set1-8}%
  }%
  {cycle list name = Set1-8}%
}

% define colour cycles to use throughout for given number of items
\newcommand*{\setplotcyclediv}[1]{%
\stringcases
  {#1}%
  {%
    {1}{cycle list = {{{orrd3}}}}%
    {2}{cycle list = {{{orrd1},{orrd3},}}}
    {3}{cycle list name = PuOr-3}%
    {4}{cycle list name = PuOr-4}%
    {5}{cycle list name = PuOr-5}%
    {6}{cycle list name = PuOr-6}%
    {7}{cycle list name = PuOr-7}%
    {8}{cycle list name = PuOr-8}%
    {9}{cycle list name = PuOr-9}%
  }%
  {cycle list name = PuOr-9}%
}

% define colour cycles to use throughout for given number of items
\newcommand*{\setplotcycleseq}[1]{%
\stringcases
  {#1}%
  {%
    {1}{cycle list name = OrRd-3}%
    {2}{cycle list name = OrRd-3}%
    {3}{cycle list name = OrRd-3}%
    {4}{cycle list name = OrRd-4}%
    {5}{cycle list name = OrRd-5}%
    {6}{cycle list name = OrRd-6}%
    {7}{cycle list name = OrRd-7}%
    {8}{cycle list name = OrRd-8}%
    {9}{cycle list name = OrRd-9}%
  }%
  {cycle list name = OrRd-9}%
}

%%%%%%%%%%%%%%%%%%%%%%%%%%% end color

\usepackage{tabularx}
\usepackage{afterpage}
\usepackage{engord}
\usepackage[toc=true,acronym]{glossaries}
\usepackage[inline]{enumitem}

%% code to highlight maximal/minimal entries in column tables with pgftables
\newcommand{\findmax}[3]{
    \pgfplotstablevertcat{\datatable}{#1}
    \pgfplotstablecreatecol[
    create col/expr={%
    \pgfplotstablerow
    }]{rownumber}\datatable
    \pgfplotstablesort[sort key={#2},sort cmp={float >}]{\sorted}{\datatable}%
    \pgfplotstablegetelem{0}{rownumber}\of{\sorted}%
    \pgfmathtruncatemacro#3{\pgfplotsretval}
    \pgfplotstableclear{\datatable}
}

\newcommand{\findmin}[3]{
    \pgfplotstablevertcat{\datatable}{#1}
    \pgfplotstablecreatecol[
      create col/expr={%
    \pgfplotstablerow
    }]{rownumber}\datatable
    \pgfplotstablesort[sort key={#2},sort cmp={float <}]{\sorted}{\datatable}%
    \pgfplotstablegetelem{0}{rownumber}\of{\sorted}%
    \pgfmathtruncatemacro#3{\pgfplotsretval}
    \pgfplotstableclear{\datatable}
}

\pgfplotstableset{
    highlight col max/.code 2 args={
        \findmax{#1}{#2}{\maxval}
        \edef\setstyles{\noexpand\pgfplotstableset{
                every row \maxval\noexpand\space column #2/.style={
                    postproc cell content/.append style={
                        /pgfplots/table/@cell content/.add={$\noexpand\bf}{$}
                    },
                }
            }
        }\setstyles
    },
    highlight col min/.code 2 args={
        \findmin{#1}{#2}{\minval}
        \edef\setstyles{\noexpand\pgfplotstableset{
                every row \minval\noexpand\space column #2/.style={
                    postproc cell content/.append style={
                        /pgfplots/table/@cell content/.add={$\noexpand\bf}{$}
                    },
                }
            }
        }\setstyles
    },
    highlight row max/.code 2 args={
        \pgfmathtruncatemacro\rowindex{#2-1}
        \pgfplotstabletranspose{\transposed}{#1}
        \findmax{\transposed}{\rowindex}{\maxval}
        \edef\setstyles{\noexpand\pgfplotstableset{
                every row \rowindex\space column \maxval\noexpand/.style={
                    postproc cell content/.append style={
                        /pgfplots/table/@cell content/.add={$\noexpand\bf}{$}
                    },
                }
            }
        }\setstyles
    },
    highlight row min/.code 2 args={
        \pgfmathtruncatemacro\rowindex{#2-1}
        \pgfplotstabletranspose{\transposed}{#1}
        \findmin{\transposed}{\rowindex}{\maxval}
        \edef\setstyles{\noexpand\pgfplotstableset{
                every row \rowindex\space column \maxval\noexpand/.style={
                    postproc cell content/.append style={
                        /pgfplots/table/@cell content/.add={$\noexpand\bf}{$}
                    },
                }
            }
        }\setstyles
    },
}

\makeatletter
\long\def\pgfplotstabletypeset@opt@collectarg[#1]#2{%

    \pgfplotstable@isloadedtable{#2}%
        {\pgfplotstabletypeset@opt@[#1]{#2}}%
        {\pgfplotstabletypesetfile@opt@[#1]{#2}}%
}
\makeatother

% !TEX root = thesis.tex
\definecolor{accolornotes}{rgb}{0.7,0.3,0.2}
\newcommand{\acnote}[1]{\textcolor{accolornotes}{[{\bf #1}]}}


\definecolor{jscolornotes}{rgb}{0.3,0.7,0.2}
\newcommand{\jsnote}[1]{\textcolor{jscolornotes}{[{\bf #1}]}}


\definecolor{acgray}{rgb}{0.8,0.8,0.8}
\newcommand{\acremove}[1]{\textcolor{acgray}{[{#1}]}}

\definecolor{myred}{rgb}{0.6, 0, 0}
\definecolor{myblue}{rgb}{0.3, 0.1, 0.9}

%\definecolor{acgreen}{rgb}{0.1,0.5,0.1}
%\definecolor{acblue}{rgb}{0.1,0.1,0.5}
%\newcommand{\bl}[1]{\textcolor{acgreen}{#1}}
%\newcommand{\gr}[1]{\textcolor{acblue}{#1}}

\newcommand{\eg}{{\it e.g.}\ }
\newcommand{\ie}{{\it i.e.}\ }
\newcommand{\vs}{{vs.}\ }
\newcommand{\etal}{{et al.}\ }
%\newcommand{\cf}{{\it cf}}

\newcommand{\be}{\begin{equation}}
\newcommand{\ee}{\end{equation}}
\newcommand{\bea}{\begin{eqnarray}}
\newcommand{\eea}{\end{eqnarray}}
\newcommand{\beas}{\begin{eqnarray*}}
\newcommand{\eeas}{\end{eqnarray*}}

\usepackage{tikz}

\newcommand{\circleplus}{ 
  \mathbin{
    \mathchoice
      {\buildcircleplus{\displaystyle}}
      {\buildcircleplus{\textstyle}}
      {\buildcircleplus{\scriptstyle}}
      {\buildcircleplus{\scriptscriptstyle}}
  } 
}

\newcommand\buildcircleplus[1]{%
  \begin{tikzpicture}[baseline=(X.base), inner sep=0, outer sep=0]
    \node[draw,circle] (X)  {$#1+$};
  \end{tikzpicture}%
}


\newcommand{\reflabel}{dummy} % Dummy initial reflabel - use renewcommand at the start of each chapter

%%%
%%% Stuff for vector typesetting  ----------------------------------------
%%%
%%%    e.g. use "\v a" or "\v r" for vectors
%%%
\def\vec#1{\mathchoice{\mbox{\boldmath  $\displaystyle\bf#1$}}
{\mbox{\boldmath  $\textstyle\bf#1$}}
{\mbox{\boldmath  $\scriptstyle\bf#1$}}
{\mbox{\boldmath  $\scriptscriptstyle\bf#1$}}}
\def\v#1{\protect\vec #1}

%%%
%%% Stuff for matrix typesetting  ----------------------------------------
%%%
%%%    e.g. use "\m M" or "\m A" for matrices
%%%
\def\mat#1{\mathchoice{\mbox{\boldmath$\displaystyle\tt#1$}}
{\mbox{\boldmath$\textstyle\tt#1$}}
{\mbox{\boldmath$\scriptstyle\tt#1$}}
{\mbox{\boldmath$\scriptscriptstyle\tt#1$}}}
\def\m#1{\protect\mat #1}

%%%
%%% Stuff for bold maths typesetting  ----------------------------------------
%%%
%%%   e.g. use "\bfmu" for boldface mu symbol
%%%
\def\bfmu{\mbox{\boldmath$\mu$}}
\def\bftau{\mbox{\boldmath$\tau$}}
\def\bftheta{\mbox{\boldmath$\theta$}}
\def\bfdelta{\mbox{\boldmath$\delta$}}
\def\bfphi{\mbox{\boldmath$\phi$}}
\def\bfpsi{\mbox{\boldmath$\psi$}}
\def\bfeta{\mbox{\boldmath$\eta$}}
\def\bfnabla{\mbox{\boldmath$\nabla$}}
\def\bfGamma{\mbox{\boldmath$\Gamma$}}

%%%
%%% Stuff for argmax argmin  ----------------------------------------
%%%
\DeclareMathOperator*{\argmax}{argmax}
\DeclareMathOperator*{\argmin}{argmin}


%%%
%%% Stuff for calligraphic style  ----------------------------------------
%%%
\DeclareMathAlphabet{\mathcal}{OMS}{cmsy}{m}{n}


\newcommand{\convolution}{\mathop{\scalebox{1.5}{\raisebox{-0.2ex}{$\ast$}}}}%
\newtheorem{theorem}{Theorem}
\robustify{\gls}
\robustify{\glspl}
\robustify{\Gls}
\robustify{\Glspl}


% activate={true,nocompatibility} - activate protrusion and expansion
% final - enable microtype; use "draft" to disable
% tracking=true, kerning=true, spacing=true - activate these techniques
% factor=1100 - add 10% to the protrusion amount (default is 1000)
% stretch=10, shrink=10 - reduce stretchability/shrinkability (default is 20/20)
%\usepackage[activate={true,nocompatibility},final,tracking=true,kerning=true,spacing=true,factor=1100,stretch=10,shrink=10]{microtype}
\usepackage[final]{microtype}
\usepackage[final]{pdfpages}
