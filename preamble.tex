% !TEX root = thesis.tex
% ******************************************************************************
% ****************************** Custom Margin *********************************

% Add `custommargin' in the document class options to use this section
% Set {innerside margin / outerside margin / topmargin / bottom margin}  and
% other page dimensions
\ifsetCustomMargin
  \RequirePackage[left=37mm,right=30mm,top=35mm,bottom=30mm]{geometry}
  \setFancyHdr % To apply fancy header after geometry package is loaded
\fi

% Add spaces between paragraphs
%\setlength{\parskip}{0.5em}
% Ragged bottom avoids extra whitespaces between paragraphs
\raggedbottom
% To remove the excess top spacing for enumeration, list and description
%\usepackage{enumitem}
%\setlist[enumerate,itemize,description]{topsep=0em}


% *****************************************************************************
% ******************* AMS Font/Symbol Packages (to not conflict with font) ****

\usepackage{amsmath, amsthm, amssymb}

% *****************************************************************************
% ******************* Fonts (like different typewriter fonts etc.)*************

% Add `customfont' in the document class option to use this section

\ifsetCustomFont
  % Set your custom font here and use `customfont' in options. Leave empty to
  % load computer modern font (default LaTeX font).
  %\RequirePackage{helvet}
  \RequirePackage{newtxtext,newtxmath}
  % \RequirePackage{tgtermes,newtxmath}

  % For use with XeLaTeX
  %  \setmainfont[
  %    Path              = ./libertine/opentype/,
  %    Extension         = .otf,
  %    UprightFont = LinLibertine_R,
  %    BoldFont = LinLibertine_RZ, % Linux Libertine O Regular Semibold
  %    ItalicFont = LinLibertine_RI,
  %    BoldItalicFont = LinLibertine_RZI, % Linux Libertine O Regular Semibold Italic
  %  ]
  %  {libertine}
  %  % load font from system font
  %  \newfontfamily\libertinesystemfont{Linux Libertine O}
\fi

% *****************************************************************************
% **************************** Custom Packages ********************************

% ************************* Algorithms and Pseudocode **************************

%\usepackage{algpseudocode}


% ********************Captions and Hyperreferencing / URL **********************

% Captions: This makes captions of figures use a boldfaced small font.
%\RequirePackage[small,bf]{caption}

\RequirePackage[labelsep=space,tableposition=top]{caption}
\renewcommand{\figurename}{Fig.} %to support older versions of captions.sty


% *************************** Graphics and figures *****************************

%\usepackage{rotating}
%\usepackage{wrapfig}

% Uncomment the following two lines to force Latex to place the figure.
% Use [H] when including graphics. Note 'H' instead of 'h'
%\usepackage{float}
%\restylefloat{figure}

% Subcaption package is also available in the sty folder you can use that by
% uncommenting the following line
% This is for people stuck with older versions of texlive
%\usepackage{sty/caption/subcaption}
\usepackage[labelformat=simple]{subcaption}
\renewcommand\thesubfigure{(\alph{subfigure})}

% ********************************** Tables ************************************
\usepackage{booktabs} % For professional looking tables
\usepackage{multirow}

%\usepackage{multicol}
%\usepackage{longtable}
%\usepackage{tabularx}


% *********************************** SI Units *********************************
\usepackage{siunitx} % use this package module for SI units

% ******************************* Line Spacing *********************************

% Choose linespacing as appropriate. Default is one-half line spacing as per the
% University guidelines

% \doublespacing
% \onehalfspacing
% \singlespacing


% ************************ Formatting / Footnote *******************************

% Don't break enumeration (etc.) across pages in an ugly manner (default 10000)
%\clubpenalty=500
%\widowpenalty=500

%\usepackage[perpage]{footmisc} %Range of footnote options

% ***************************** Epi-graph for inspirational quotes *************************************
\makeatletter
\renewcommand{\@chapapp}{}% Not necessary...
\newenvironment{chapquote}[2][1.5em]
  {\setlength{\@tempdima}{#1}%
   \def\chapquote@author{#2}%
   \parshape 1 \@tempdima \dimexpr\textwidth-2\@tempdima\relax%
   \itshape}
  {\par\normalfont\hfill--\ \chapquote@author\hspace*{\@tempdima}\par\bigskip}
\makeatother

% ********************************** Sub-Files ************************************
\usepackage{subfiles} % To work seamlessly with mutiple file projects

% *****************************************************************************
% *************************** Bibliography  and References ********************

\usepackage[capitalise]{cleveref} %Referencing without need to explicitly state fig /table

% Add `custombib' in the document class option to use this section
\ifuseCustomBib
%   \RequirePackage[square, sort, numbers, authoryear]{natbib} % CustomBib

% If you would like to use biblatex for your reference management, as opposed to the default `natbibpackage` pass the option `custombib` in the document class. Comment out the previous line to make sure you don't load the natbib package. Uncomment the following lines and specify the location of references.bib file
	\RequirePackage[backend=biber, bibstyle=authoryear, citestyle=authoryear, natbib=true, giveninits=true, uniquename=init, url=false, dashed=false]{biblatex}
	\bibliography{references} %Location of references.bib only for biblatex
	% Use author initials only for consistency in inconsistent refs
\fi
% changes the default name `Bibliography` -> `References'
\renewcommand{\bibname}{References}


% ******************************************************************************
% ************************* User Defined Commands ******************************
% ******************************************************************************

% *********** To change the name of Table of Contents / LOF and LOT ************

%\renewcommand{\contentsname}{My Table of Contents}
%\renewcommand{\listfigurename}{My List of Figures}
%\renewcommand{\listtablename}{My List of Tables}


% ********************** TOC depth and numbering depth *************************

\setcounter{secnumdepth}{2}
\setcounter{tocdepth}{1}


% ******************************* Nomenclature *********************************

% To change the name of the Nomenclature section, uncomment the following line

%\renewcommand{\nomname}{Symbols}


% ********************************* Appendix ***********************************

% The default value of both \appendixtocname and \appendixpagename is `Appendices'. These names can all be changed via:

%\renewcommand{\appendixtocname}{List of appendices}
%\renewcommand{\appendixname}{Appndx}

% *********************** Configure Draft Mode **********************************

% Uncomment to disable figures in `draft'
%\setkeys{Gin}{draft=true}  % set draft to false to enable figures in `draft'

% These options are active only during the draft mode
% Default text is "Draft"
%\SetDraftText{DRAFT}

% Default Watermark location is top. Location (top/bottom)
%\SetDraftWMPosition{bottom}

% Draft Version - default is v1.0
%\SetDraftVersion{v1.1}

% Draft Text grayscale value (should be between 0-black and 1-white)
% Default value is 0.75
%\SetDraftGrayScale{0.8}


% ******************************** Todo Notes **********************************
%% Uncomment the following lines to have todonotes.

\ifsetDraft
	\usepackage[colorinlistoftodos]{todonotes}
	\newcommand{\mynote}[1]{\todo[author=yani,size=\small,inline,color=green!40]{#1}}
\else
	\newcommand{\mynote}[1]{}
	\newcommand{\listoftodos}{}
\fi

% Example todo: \mynote{Hey! I have a note}

% **************************** Define Graphics Path **************************
\ifpdf
    \graphicspath{{Figs/PDF/}{Figs/Raster/}{Figs/Vector/}{Figs/}}
\else
    \graphicspath{{Figs/EPS/}{Figs/Raster/}{Figs/Vector/}{Figs/}}
\fi

\usepackage{tikz}
\usetikzlibrary{arrows}
\usetikzlibrary{colorbrewer}
\usepackage{pgfplots, pgfplotstable}
\usepgfplotslibrary{statistics}
%\usepgfplotslibrary{groupplots}
\usepgfplotslibrary{colorbrewer}
\pgfplotsset{compat=newest}
%% code to highlight maximal/minimal entries in column tables with pgftables
\newcommand{\findmax}[3]{
    \pgfplotstablevertcat{\datatable}{#1}
    \pgfplotstablecreatecol[
    create col/expr={%
    \pgfplotstablerow
    }]{rownumber}\datatable
    \pgfplotstablesort[sort key={#2},sort cmp={float >}]{\sorted}{\datatable}%
    \pgfplotstablegetelem{0}{rownumber}\of{\sorted}%
    \pgfmathtruncatemacro#3{\pgfplotsretval}
    \pgfplotstableclear{\datatable}
}

\newcommand{\findmin}[3]{
    \pgfplotstablevertcat{\datatable}{#1}
    \pgfplotstablecreatecol[
      create col/expr={%
    \pgfplotstablerow
    }]{rownumber}\datatable
    \pgfplotstablesort[sort key={#2},sort cmp={float <}]{\sorted}{\datatable}%
    \pgfplotstablegetelem{0}{rownumber}\of{\sorted}%
    \pgfmathtruncatemacro#3{\pgfplotsretval}
    \pgfplotstableclear{\datatable}
}

\pgfplotstableset{
    highlight col max/.code 2 args={
        \findmax{#1}{#2}{\maxval}
        \edef\setstyles{\noexpand\pgfplotstableset{
                every row \maxval\noexpand\space column #2/.style={
                    postproc cell content/.append style={
                        /pgfplots/table/@cell content/.add={$\noexpand\bf}{$}
                    },
                }
            }
        }\setstyles
    },
    highlight col min/.code 2 args={
        \findmin{#1}{#2}{\minval}
        \edef\setstyles{\noexpand\pgfplotstableset{
                every row \minval\noexpand\space column #2/.style={
                    postproc cell content/.append style={
                        /pgfplots/table/@cell content/.add={$\noexpand\bf}{$}
                    },
                }
            }
        }\setstyles
    },
    highlight row max/.code 2 args={
        \pgfmathtruncatemacro\rowindex{#2-1}
        \pgfplotstabletranspose{\transposed}{#1}
        \findmax{\transposed}{\rowindex}{\maxval}
        \edef\setstyles{\noexpand\pgfplotstableset{
                every row \rowindex\space column \maxval\noexpand/.style={
                    postproc cell content/.append style={
                        /pgfplots/table/@cell content/.add={$\noexpand\bf}{$}
                    },
                }
            }
        }\setstyles
    },
    highlight row min/.code 2 args={
        \pgfmathtruncatemacro\rowindex{#2-1}
        \pgfplotstabletranspose{\transposed}{#1}
        \findmin{\transposed}{\rowindex}{\maxval}
        \edef\setstyles{\noexpand\pgfplotstableset{
                every row \rowindex\space column \maxval\noexpand/.style={
                    postproc cell content/.append style={
                        /pgfplots/table/@cell content/.add={$\noexpand\bf}{$}
                    },
                }
            }
        }\setstyles
    },
}

\makeatletter
\long\def\pgfplotstabletypeset@opt@collectarg[#1]#2{%

    \pgfplotstable@isloadedtable{#2}%
        {\pgfplotstabletypeset@opt@[#1]{#2}}%
        {\pgfplotstabletypesetfile@opt@[#1]{#2}}%
}
\makeatother

\usepackage{tabularx}
%\usepackage{mathabx}
\usepackage{afterpage}
\usepackage{engord}

\newcommand{\convolution}{\mathop{\scalebox{1.5}{\raisebox{-0.2ex}{$\ast$}}}}%

%\pgfplotsset{
%	cycle list={Set1-9}
%}
\newtheorem{theorem}{Theorem}

\usepackage[inline]{enumitem}

% !TEX root = thesis.tex
\definecolor{accolornotes}{rgb}{0.7,0.3,0.2}
\newcommand{\acnote}[1]{\textcolor{accolornotes}{[{\bf #1}]}}


\definecolor{jscolornotes}{rgb}{0.3,0.7,0.2}
\newcommand{\jsnote}[1]{\textcolor{jscolornotes}{[{\bf #1}]}}


\definecolor{acgray}{rgb}{0.8,0.8,0.8}
\newcommand{\acremove}[1]{\textcolor{acgray}{[{#1}]}}

\definecolor{myred}{rgb}{0.6, 0, 0}
\definecolor{myblue}{rgb}{0.3, 0.1, 0.9}

%\definecolor{acgreen}{rgb}{0.1,0.5,0.1}
%\definecolor{acblue}{rgb}{0.1,0.1,0.5}
%\newcommand{\bl}[1]{\textcolor{acgreen}{#1}}
%\newcommand{\gr}[1]{\textcolor{acblue}{#1}}

\newcommand{\eg}{{\it e.g.}\ }
\newcommand{\ie}{{\it i.e.}\ }
\newcommand{\vs}{{vs.}\ }
\newcommand{\etal}{{et al.}\ }
%\newcommand{\cf}{{\it cf}}

\newcommand{\be}{\begin{equation}}
\newcommand{\ee}{\end{equation}}
\newcommand{\bea}{\begin{eqnarray}}
\newcommand{\eea}{\end{eqnarray}}
\newcommand{\beas}{\begin{eqnarray*}}
\newcommand{\eeas}{\end{eqnarray*}}

\usepackage{tikz}

\newcommand{\circleplus}{ 
  \mathbin{
    \mathchoice
      {\buildcircleplus{\displaystyle}}
      {\buildcircleplus{\textstyle}}
      {\buildcircleplus{\scriptstyle}}
      {\buildcircleplus{\scriptscriptstyle}}
  } 
}

\newcommand\buildcircleplus[1]{%
  \begin{tikzpicture}[baseline=(X.base), inner sep=0, outer sep=0]
    \node[draw,circle] (X)  {$#1+$};
  \end{tikzpicture}%
}


\newcommand{\reflabel}{dummy} % Dummy initial reflabel - use renewcommand at the start of each chapter

%%%
%%% Stuff for vector typesetting  ----------------------------------------
%%%
%%%    e.g. use "\v a" or "\v r" for vectors
%%%
\def\vec#1{\mathchoice{\mbox{\boldmath  $\displaystyle\bf#1$}}
{\mbox{\boldmath  $\textstyle\bf#1$}}
{\mbox{\boldmath  $\scriptstyle\bf#1$}}
{\mbox{\boldmath  $\scriptscriptstyle\bf#1$}}}
\def\v#1{\protect\vec #1}

%%%
%%% Stuff for matrix typesetting  ----------------------------------------
%%%
%%%    e.g. use "\m M" or "\m A" for matrices
%%%
\def\mat#1{\mathchoice{\mbox{\boldmath$\displaystyle\tt#1$}}
{\mbox{\boldmath$\textstyle\tt#1$}}
{\mbox{\boldmath$\scriptstyle\tt#1$}}
{\mbox{\boldmath$\scriptscriptstyle\tt#1$}}}
\def\m#1{\protect\mat #1}

%%%
%%% Stuff for bold maths typesetting  ----------------------------------------
%%%
%%%   e.g. use "\bfmu" for boldface mu symbol
%%%
\def\bfmu{\mbox{\boldmath$\mu$}}
\def\bftau{\mbox{\boldmath$\tau$}}
\def\bftheta{\mbox{\boldmath$\theta$}}
\def\bfdelta{\mbox{\boldmath$\delta$}}
\def\bfphi{\mbox{\boldmath$\phi$}}
\def\bfpsi{\mbox{\boldmath$\psi$}}
\def\bfeta{\mbox{\boldmath$\eta$}}
\def\bfnabla{\mbox{\boldmath$\nabla$}}
\def\bfGamma{\mbox{\boldmath$\Gamma$}}

%%%
%%% Stuff for argmax argmin  ----------------------------------------
%%%
\DeclareMathOperator*{\argmax}{argmax}
\DeclareMathOperator*{\argmin}{argmin}


%%%
%%% Stuff for calligraphic style  ----------------------------------------
%%%
\DeclareMathAlphabet{\mathcal}{OMS}{cmsy}{m}{n}

