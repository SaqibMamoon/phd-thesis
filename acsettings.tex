
\definecolor{accolornotes}{rgb}{0.7,0.3,0.2}
\newcommand{\acnote}[1]{\textcolor{accolornotes}{[{\bf #1}]}}


\definecolor{jscolornotes}{rgb}{0.3,0.7,0.2}
\newcommand{\jsnote}[1]{\textcolor{jscolornotes}{[{\bf #1}]}}


\definecolor{acgray}{rgb}{0.8,0.8,0.8}
\newcommand{\acremove}[1]{\textcolor{acgray}{[{#1}]}}

\definecolor{myred}{rgb}{0.6, 0, 0}
\definecolor{myblue}{rgb}{0.3, 0.1, 0.9}

%\definecolor{acgreen}{rgb}{0.1,0.5,0.1}
%\definecolor{acblue}{rgb}{0.1,0.1,0.5}
%\newcommand{\bl}[1]{\textcolor{acgreen}{#1}}
%\newcommand{\gr}[1]{\textcolor{acblue}{#1}}

\newcommand{\eg}{{\it e.g.}\ }
\newcommand{\ie}{{\it i.e.}\ }
\newcommand{\vs}{{v.s.}\ }
\newcommand{\etal}{{et al.}\ }
%\newcommand{\cf}{{\it cf}}

\newcommand{\be}{\begin{equation}}
\newcommand{\ee}{\end{equation}}
\newcommand{\bea}{\begin{eqnarray}}
\newcommand{\eea}{\end{eqnarray}}
\newcommand{\beas}{\begin{eqnarray*}}
\newcommand{\eeas}{\end{eqnarray*}}

% Standardized section label and references
\newcommand{\chaplabel}[1]{\label{chap:#1}}
\newcommand{\chapref}[1]{\chaptername~\ref{chap:#1}} % Use mid-sentence
\newcommand{\Chapref}[1]{\chaptername~\ref{chap:#1}} % Use at start of sentence
\newcommand{\chaprefs}[2]{\chaptername s~\ref{chap:#1}~and~\ref{chap:#2}} % Use mid-sentence
\newcommand{\Chaprefs}[2]{\chaptername s~\ref{chap:#1}~and~\ref{chap:#2}} % Use at start of sentence
\newcommand{\appref}[1]{\appendixname~\ref{chap:#1}}

% Standardized section label and references
\newcommand{\seclabel}[1]{\label{sec:\reflabel-#1}}
\newcommand{\secref}[2][\reflabel]{Section~\ref{sec:#1-#2}} % Use mid-sentence
\newcommand{\Secref}[2][\reflabel]{Section~\ref{sec:#1-#2}} % Use at start of sentence
\newcommand{\secrefs}[3][\reflabel]{Sections~\ref{sec:#1-#2} and~\ref{sec:#1-#3}} % Use mid-sentence
\newcommand{\Secrefs}[3][\reflabel]{Sections~\ref{sec:#1-#2} and~\ref{sec:#1-#3}} % Use at start of sentence

% Standardized figure label and references
\newcommand{\figlabel}[2][\reflabel]{\label{fig:#1-#2}}
\newcommand{\figref}[2][\reflabel]{Figure~\ref{fig:#1-#2}} % Use mid-sentence
\newcommand{\Figref}[2][\reflabel]{Figure~\ref{fig:#1-#2}} % Use at start of sentence
\newcommand{\figrefs}[3][\reflabel]{Figures~\ref{fig:#1-#2} and~\ref{fig:#1-#3}} % Use mid-sentence
\newcommand{\Figrefs}[3][\reflabel]{Figures~\ref{fig:#1-#2} and~\ref{fig:#1-#3}} % Use at start of sentence

% Standardized table label and references
\newcommand{\tablelabel}[2][\reflabel]{\label{table:#1-#2}}
\newcommand{\tableref}[2][\reflabel]{Table~\ref{table:#1-#2}} % Use mid-sentence
\newcommand{\Tableref}[2][\reflabel]{Table~\ref{table:#1-#2}} % Use at start of sentence
\newcommand{\tablerefs}[3][\reflabel]{Tables~\ref{table:#1-#2} and~\ref{table:#1-#3}} % Use mid-sentence
\newcommand{\Tablerefs}[3][\reflabel]{Tables~\ref{table:#1-#2} and~\ref{table:#1-#3}} % Use at start of sentence

% Standardized equation label and references -- NB if you change this one, some of the uses in the text might have grammatical problems
\newcommand{\eqlabel}[1]{\label{eq:\reflabel-#1}}
\renewcommand{\eqref}[2][\reflabel]{(\ref{eq:#1-#2})} % Use mid-sentence
\newcommand{\Eqref}[2][\reflabel]{(\ref{eq:#1-#2})} % Use at start of sentence
\newcommand{\eqrefs}[3][\reflabel]{(\ref{eq:#1-#2}) and~(\ref{eq:#1-#3})} % Use mid-sentence
\newcommand{\Eqrefs}[3][\reflabel]{Equations (\ref{eq:#1-#2}) and~(\ref{eq:#1-#3})} % Use at start of sentence

% Standardized algorithm label and references
\newcommand{\alglabel}[2][\reflabel]{\label{alg:#1-#2}}
\newcommand{\algref}[2][\reflabel]{Algorithm~\ref{alg:#1-#2}} % Use mid-sentence
\newcommand{\algrefs}[3][\reflabel]{Algorithms~\ref{alg:#1-#2} and~\ref{alg:#1-#3}} % Use at start of sentence
\newcommand{\Algref}[2][\reflabel]{Algorithm~\ref{alg:#1-#2}} % Use mid-sentence
\newcommand{\Algrefs}[3][\reflabel]{Algorithm~\ref{alg:#1-#2} and~\ref{alg:#1-#3}} % Use at start of sentence

% Standardized part references
\newcommand{\PartI}{Part I\xspace}
\newcommand{\PartII}{Part II\xspace}
\newcommand{\PartIII}{Part III\xspace}

\usepackage{tikz}

\newcommand{\circleplus}{ 
  \mathbin{
    \mathchoice
      {\buildcircleplus{\displaystyle}}
      {\buildcircleplus{\textstyle}}
      {\buildcircleplus{\scriptstyle}}
      {\buildcircleplus{\scriptscriptstyle}}
  } 
}

\newcommand\buildcircleplus[1]{%
  \begin{tikzpicture}[baseline=(X.base), inner sep=0, outer sep=0]
    \node[draw,circle] (X)  {$#1+$};
  \end{tikzpicture}%
}


\newcommand{\reflabel}{dummy} % Dummy initial reflabel - use renewcommand at the start of each chapter

%%%
%%% Stuff for vector typesetting  ----------------------------------------
%%%
%%%    e.g. use "\v a" or "\v r" for vectors
%%%
\def\vec#1{\mathchoice{\mbox{\boldmath  $\displaystyle\bf#1$}}
{\mbox{\boldmath  $\textstyle\bf#1$}}
{\mbox{\boldmath  $\scriptstyle\bf#1$}}
{\mbox{\boldmath  $\scriptscriptstyle\bf#1$}}}
\def\v#1{\protect\vec #1}

%%%
%%% Stuff for matrix typesetting  ----------------------------------------
%%%
%%%    e.g. use "\m M" or "\m A" for matrices
%%%
\def\mat#1{\mathchoice{\mbox{\boldmath$\displaystyle\tt#1$}}
{\mbox{\boldmath$\textstyle\tt#1$}}
{\mbox{\boldmath$\scriptstyle\tt#1$}}
{\mbox{\boldmath$\scriptscriptstyle\tt#1$}}}
\def\m#1{\protect\mat #1}

%%%
%%% Stuff for bold maths typesetting  ----------------------------------------
%%%
%%%   e.g. use "\bfmu" for boldface mu symbol
%%%
\def\bfmu{\mbox{\boldmath$\mu$}}
\def\bftau{\mbox{\boldmath$\tau$}}
\def\bftheta{\mbox{\boldmath$\theta$}}
\def\bfdelta{\mbox{\boldmath$\delta$}}
\def\bfphi{\mbox{\boldmath$\phi$}}
\def\bfpsi{\mbox{\boldmath$\psi$}}
\def\bfeta{\mbox{\boldmath$\eta$}}
\def\bfnabla{\mbox{\boldmath$\nabla$}}
\def\bfGamma{\mbox{\boldmath$\Gamma$}}

%%%
%%% Stuff for argmax argmin  ----------------------------------------
%%%
\DeclareMathOperator*{\argmax}{argmax}
\DeclareMathOperator*{\argmin}{argmin}


%%%
%%% Stuff for calligraphic style  ----------------------------------------
%%%
\DeclareMathAlphabet{\mathcal}{OMS}{cmsy}{m}{n}
